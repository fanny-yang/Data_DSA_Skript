


\section *{Wahrscheinlichkeitstheorie}


\vspace{15pt}


1933 ver\"offentlichte der russische Mathematiker Andrey Kolmogorov sein Buch \textit{Foundations of the Theory of Probability}, in dem er die drei Axiome der Wahrscheinlichkeitstheorie aufstellte. Es existiert ein Wahrscheinlichkeitsraum, der sich aus den drei Elementen $\Omega, \mathcal{F}, \mathds{P}$ zusammensetzt.

\vspace{5pt}

\begin{enumerate}
	\item $\Omega$ ist eine Menge mit endlicher Anzahl an Elementen. $|\Omega|= n $
	\item $\mathcal{F}$ ist eine Menge der Ereignisse $\textsce $ und eine Teilmenge von $\Omega$. $\mathcal{F} \subseteq \Omega$
	\item $\mathds{P}$ bezeichnet die Sicherheit der Wahrscheinlichkeit, das Wahrscheinlichkeitsma�. Dieses weist jedem Ereignis in $\mathcal{F}$ eine reelle Zahl zu.  $\mathds{P}: \mathcal{F} \longrightarrow \mathds{R}$
	
	
\end{enumerate}

\vspace{10pt}



\subsection *{Kolmogorovs Axiome lauten:}

\vspace{5pt}

\begin{enumerate}
	\item Die Wahrscheinlichkeit eines Ereignisses ist eine positive, reelle Zahl. \\ $\mathds{P} (\mathds{E}) \geq 0$ f\"ur alle $\mathds{E} \in \mathcal{F}$.
	\item Die Menge aller Ergebnisse bezeichnet man als sicheres Ereignis, das die Wahrscheinlichkeit $1$ hat. $\mathds{P} (\Omega) = 1$
	\item Die Wahrscheinlichkeit der Vereinigung von disjunkten Ereignisssen ist gleich der Summe der Wahrscheinlichkeiten der disjunkten Ereignisse. 
	\vspace{3pt}
	\begin{equation*}
	\mathds{P} (\cup_{i=1}^m \mathds{E}_{i}) = \sum_{i=1}^n \mathds{P} (\mathds{E}_{i})$ f\"ur alle $\mathds{E}_{1},...,\mathds{E}_{m} \in \mathcal{F}
	\end{equation*}
\end{enumerate}
\vspace{5pt}

$\mathds{E}_{1},...,\mathds{E}_{m}$ ist disjunkt, falls $\mathds{E}_{i} \cap \mathds{E}_{j} = \O$ f\"ur alle $i,j= 1,2,...,m$. 
Hierbei bezeichnet $\O$ ein unm\"ogliches Ereignis $\O \in \mathcal{F}$. \\ Im Gegensatz dazu gibt es das sichere Ereignis mit ${\Omega} \in \mathcal{F}$.

\vspace{5pt}

Die Wahrscheinlichkeit wird errechnet durch $ \mathds{P} = \frac {|\mathds{E}|}{|\Omega|}$.
Alternativ definiert man: 

\vspace{5pt}


$p_{i} \geq 0$  f\"ur $i \in \Omega$, sodass $\sum_{i\in \Omega} p_{i} =1$, $p_{i} = \mathds{P} ({i})$



\vspace{10pt}

\subsection *{Konsequenzen der Axiome:}

\vspace{5pt}

\begin{enumerate}
	\item $\mathds{P} (\O) = 0$
	\item Monotonie: $A\subseteq B$ daraus folgt: $\mathds{P} (A) \leq \mathds{P} (B)$
	\item $0 \leq \mathds{P} (E) \leq 1$
	\item $ \mathds{P} (A \cup B) = \mathds{P} (A) + \mathds{P} (B) - \mathds{P} (A \cap B)$
\end{enumerate}


\vspace{10pt}

\subsection *{Zufallsvariablen}

\vspace{5pt}

In einem Wahrscheinlichkeitsraum $(\Omega, \mathcal{F}, \mathds{P})$ beschreibt die Funktion $X:\Omega \longrightarrow \mathds{R}^k$ einen k-dimensionalen Zufallsvektoren. Im Fall von $k=1$ redet man von einer Zufallsvariablen.

\vspace{10pt}

\subsection *{Erwartungswert}

\vspace{5pt}

 Der Erwartungswert einer Zufallsvariablen beschreibt die Zahl, die die Zufallsvariable im Mittel annimmt.

\vspace{3pt}

\begin{equation*}
\mathds{E} [X] = \sum_{x \in \Omega} X (i) * \mathds{P} ({i}) = \sum_{x\in\mathcal{X}} X* \mathds{P} (X=x) wobei $\mathcal{X} $
\end{equation*}
  ein Wertebereich von x ist  

\vspace{10pt}

\subsection *{Varianz}

\vspace{5pt}

Die Varianz kennzeichnet die Ausdehnung einer Wahrscheinlichkeit. Die Varianz einer Zufallsvariable X.

\vspace{3pt}

\begin{equation*}
X= \mathds{E} [X^2] - \mathds{E} [X]^2 = \mathds{E} [ (X - \mathds{E} [X] )^2 ] 
\end{equation*}

\vspace{10pt}

\subsection *{Unabh\"angigkeit von Ereignissen}

\vspace{5pt}

\begin{enumerate}
	\item Wenn f\"ur Ereignisse $E, F \in \mathcal{F} $ gilt
	\vspace{3pt}
	$\mathds{P} (E\cap F)= \mathds{P} (E) * \mathds{P} (F)$
	\vspace{3pt}
dann sind E und F unabh\"angig. 

\item Wenn f\"ur die Zufallsvariablen 
\begin{equation*}
X_{1}: \Omega_{1} \longrightarrow \mathds{R}, \mathcal{F}_{1} , X_{2}: \Omega_{2} \longrightarrow \mathds{R}, \mathcal{F}_{2}
\end{equation*}
 gilt
\vspace{3pt}
\begin{equation*}
\mathds{P} ({X_{1} \in A} \cap {X_{2} \in B} ) = \mathds{P} ({X_{1} \in A} * \mathds{P} {X_{2} \in B} )
\end{equation*}
\vspace{3pt}
f\"ur alle $A, B \subseteq \mathds{R}$
\end{enumerate}

\vspace{10pt}

\subsection *{Marginal bestimmte Wahrscheinlichkeiten}

\vspace{5pt}

\begin{equation*}
\mathds{P} ({X_{1} \in A } ) = \sum_{b \in \Omega_{2} } \mathds{P} {X_{1} \in A} \cap {X_{2} \in b})
\end{equation*}

\vspace{10pt}

\subsection *{Bedingte Wahrscheinlichkeit}

\vspace{5pt}

\begin{equation*}
\mathds{P} ({X_{1} \in A } | { X_{2} \in B } ) = \frac {\mathds{P} ( {X_{1} \in A } \cap {X_{2} \in B })} {\mathds{P} ({X_{2} \in B})}
\end{equation*}

\vspace{10pt}

\subsection *{Summe von Zufallsvariablen}

\vspace{5pt}
\begin{equation*}
X_{1}:\Omega \longrightarrow \mathds{R}, X_{2}:\Omega \longrightarrow \mathds{R}
\end{equation*}
\begin{equation*}
Y=X_{1}+X_{2}
\end{equation*}
\vspace{3pt}
Annahme: $X_{1}$ und $X_{2}$ sind unabh\"angig.\\
\begin{equation*}
{Y=l}=\cup_{i=-\infty}^{\infty} {X_{1}=1} \cap {X_{2}=l-i}
\end{equation*}
\vspace{3pt}
\begin{equation*}
\mathds{P} ({Y=l}) = \sum_{i=-\infty}^{\infty} \mathds{P} ({X_{1}=i} \cap {X_{2}=l-i}) = \sum_{i=-\intfy}^{\intfy} \mathds{P} ({X_{1} = i}) * \mathds{P} ({X_{2}=l-i})
\end{equation*}

\vspace{10pt}

\subsection *{Theorem: Liniarit\"at des Erwartungswerts}

\vspace{5pt}

Angenommen, wir h\"atten eine Funktion
\vspace{3pt}

$f: \mathds{R} \longrightarrow \mathds{R}$ , f\"ur die gilt: 
\begin{equation*}
\int_{\mathds{R}} f(x) dx = 1
\end{equation*}

\vspace{3pt}

dann k\"onnten wir eine Wahrscheinlichkeitsverteilung folgenderma"sen definieren:
\begin{equation*}
\mathds{P} ({X\in A}) = \int_{X \in A} f(x) dx
\end{equation*}


