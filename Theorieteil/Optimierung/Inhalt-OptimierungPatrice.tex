\subsection{Gradientenmethode}

Folgender Algorithmus implementiert die Gradientenmethode, die einen wichtigen 
Teil des Gebietes der Optimierung darstellt. Funktionieren tut sie wie folgt:
\begin{algorithmic}[1]

\Procedure{$\mathbf{calculateMinimum}$}{}
   \\F(x) = (x$^2$-2)$^2$
\\F'(x) = 4x$^3$-8x

\\
\\x = 10.0
\\$\lambda$ = 0.001

\For{i = 8}
 \State $\lambda$ = $\lambda$+(i*0.001)
  \For{j = 8}
    \State x = x - $\lambda$ * F'(x)
  \EndFor
\EndFor
 
 
\\Print x\EndProcedure
\Statex
\end{algorithmic}

\paragraph{Erklärung}
F(x) beschreibt die Funktion, deren globales Minimum wir finden wollen.
F'(x) ist dementsprechend die Ableitung der Funktion F(x).
Mit x=10.0 setzen wir den Schätzwert, ab dem optimiert wird. 
$\lambda$ bekommt einen niedrigen Wert deklariert, damit
man sich in kleinen Schritten dem Minimum annähern kann.
Dies wiederum geschieht in zwei For-Schleifen, deren Inhalt
 in diesem Fall je 8 mal durchlaufen wird.
 In der äußeren Schleife sorgen wir dafür, dass unser Lambda größere Werte annimmt.
 In der inneren For-Schleife wird die Schätz-Variable x mit Hilfe der Ableitung 
 und Lambda angepasst.
 Daraus folgt:
 \\
\begin{equation*}
 f(x+\lambda \nabla f(x)) \leq f(x)
 \end{equation*}

