\subsection{Gradientenmethode}

Folgend beschrieben wird Gradientenmethode, die einen wichtigen Teil des Gebietes der Optimierung darstellt.
Das Ziel der Gradientenmethode ist es, das Optimum einer Funktion zu finden. Ein 
anfänglicher Schätzwert wird dafür in Richtung des negativen Gradienten 
optimiert, man nähert sich dem Minimum der Funktion möglichst genau durch jenes 
Abstiegsverfahren an.

\begin{dsafigure}
\begin{center}
\includegraphics[width=0.4\textwidth]{\media grafik_optimierung_patrice_gradient_descent.pdf}
\label{figure:grafik_optimierung_patrice_gradient_descent}
\caption{Beispiel der Gradientenmethode}
\end{center}
\end{dsafigure}

Eine mögliche Softwareimplementierung in Pseudocode wird hier gezeigt:

\begin{algorithmic}[1]

\Procedure{$\mathbf{calculateMinimum}$}{}
   \\$F(x) = (x^2-2)^2$
\\$F'(x) = 4x^3-8x$

\\
\\$x = 10.0$
\\$\lambda = 0.001$

\For{$i = 1, \dots, m$}
 \State $\lambda = \lambda+0.001\cdot i$
  \For{$j = 1, \dots, n$}
    \State $x = x - \lambda \cdot F'(x)$
  \EndFor
\EndFor
 
 
\\Print x\EndProcedure
\Statex
\end{algorithmic}

\paragraph{Erklärung}
$F(x)$ beschreibt die Funktion, deren globales Minimum wir finden wollen. $F'(x)$ ist die Ableitung der Funktion $F(x)$. Mit $x$ deklarieren wir den Schätzwert, ab dem optimiert wird. $\lambda$ bekommt einen niedrigen Wert deklariert, damit man sich in kleinen Schritten dem Minimum annähern kann. Dies wiederum geschieht in zwei For-Schleifen, deren Inhalt in diesem Fall n bzw. m mal durchlaufen wird. In der äußeren Schleife sorgen wir dafür, dass unser Lambda größere Werte annimmt. In der inneren For-Schleife wird die Schätz-Variable $x$ mit Hilfe der Ableitung und $\lambda$ angepasst. Daraus folgt:
\begin{equation*}
 f(x+\lambda \nabla f(x)) \leq f(x)
 \end{equation*}

