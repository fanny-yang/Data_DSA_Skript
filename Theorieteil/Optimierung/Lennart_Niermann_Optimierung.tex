%Bitte noch in den Programmkopf einf�gen, damit die Geogebradarstellungen funktionieren: 
%\usetikzlibrary{arrows}
%\usepackage{pgf,tikz}
%\usepackage{mathrsfs}
\section{Optimierung}
\subsection{Grundlagen der Optimierung}
Bei der Optimierung betrachtet man vor allem Funktionen, welche Elemente eines $n$-dimensionalen Vektorraumes auf reelle Zahlen abbilden:

\begin{equation*}
f: \mathbb{R}^n \rightarrow \mathbb{R}
\end{equation*}

Beispielhafte Anwendungszwecke w�ren die Minimierung oder Maximierung der Kosten bzw. der Einnahmen eines Unternehmens. Diese w�rden dem reellen Funktionswert der Funktion entsprechen und erg�ben sich aus den Eigenschaften des Unternehmens wie Gehalt, Anzahl der Mitarbeiter, Marketing-Ausgaben usw., welche man zusammenfassend als Vektor eines $n$-dimensionalen Vektorraumes darstellen k�nnte.

\subsection{Regression}

\begin{Thm}[Methode der kleinsten Quadrate]
Beispiel f�r die L�sung eines solchen Minimierungsproblems w�re die Methode der kleinsten Quadrate. Gegeben ist hierbei eine lineare Funktion $f$ und Punkte $P_i=(x_i,y_i)$, welche n�herungsweise auf der Geraden liegen \ref{fig:LinReg}:

\begin{align*}
f(x) &= ax\qquad \wedge\qquad y_i &= f(x_i) + \epsilon_{i}
\end{align*}

%\definecolor{ffqqqq}{rgb}{1.,0.,0.}
%\definecolor{qqqqff}{rgb}{0.,0.,1.}
%\begin{dsafigure}
%\begin{tikzpicture}[line cap=round,line join=round,>=triangle 45,x=1.0cm,y=1.0cm]
%\draw[->,color=black] (-0.252729391848446,0.) -- (5.909881514665974,0.);
%\foreach \x in {,0.5,1.,1.5,2.,2.5,3.,3.5,4.,4.5,5.,5.5}
%\draw[shift={(\x,0)},color=black] (0pt,2pt) -- (0pt,-2pt);
%\draw[->,color=black] (0.,-0.118522622381023) -- (0.,4.459842973966475);
%\foreach \y in {,0.5,1.,1.5,2.,2.5,3.,3.5,4.}
%\draw[shift={(0,\y)},color=black] (2pt,0pt) -- (-2pt,0pt);
%\clip(-0.252729391848446,-0.118522622381023) rectangle (5.909881514665974,4.459842973966475);
%\draw[smooth,samples=100,domain=-0.252729391848446:5.909881514665974] plot(\x,{(\x)});
%\draw [color=ffqqqq] (1.,1.)-- (1.,1.5);
%\draw [color=ffqqqq] (2.,2.)-- (2.,1.7);
%\draw [color=ffqqqq] (3.,3.)-- (3.,3.6);
%\draw [color=ffqqqq] (4.,4.)-- (4.,3.2);
%\begin{scriptsize}
%\draw [fill=qqqqff] (1.,1.5) circle (1.5pt);
%\draw[color=qqqqff] (0.9863140708369929,1.7517884297439121) node {$P_1$};
%\draw [fill=qqqqff] (2.,1.7) circle (1.5pt);
%\draw[color=qqqqff] (1.9808094816766213,1.4790347346423591) node {$P_2$};
%\draw [fill=qqqqff] (3.,3.6) circle (1.5pt);
%\draw[color=qqqqff] (2.9753048925162497,3.8656295667809477) node {$P_3$};
%\draw [fill=qqqqff] (4.,3.2) circle (1.5pt);
%\draw[color=qqqqff] (3.95893150105162,2.9499564475114486) node {$P_4$};
%\draw[color=ffqqqq] (1.0297892800540258,1.2647282599197103) node {$\varepsilon_1$};
%\draw[color=ffqqqq] (2.230791934674561,1.9466124976735928) node {$\varepsilon_2$};
%\draw[color=ffqqqq] (3.029648904037541,3.29089856638839) node {$\varepsilon_3$};
%\draw[color=ffqqqq] (4.236085959810206,3.6902879056442353) node {$\varepsilon_4$};
%\end{scriptsize}
%\end{tikzpicture}
%\caption{Beispiel einer Linearen Reggression}
%\label{fig:LinReg}
%\end{dsafigure}


Wenn nun der Abstand $\epsilon_i$ minimiert werden soll, dann gilt:

\begin{equation*}
\min h(x)={\sum_{i=1}^n(y_i-ax_i)^2)}
\end{equation*}

 Zur Berechnung dieses Minimums gibt es nun verschiedene M�glichkeiten:

\begin{enumerate}
\item Gew�hnliche Tiefpunktberechnung \\
z.~B.$\qquad h'(a)=0\qquad \wedge\qquad h''(a)>0$
\item Gradientenabstiegsmethode \\
$a^{t+1}  =a^{t} - \lambda \cdot \partial f(a^{t})$ \\
Man findet also ein $\lambda > 0$, sodass\\
$f(a^{t+1}) < f(a^{t})  \qquad \mid \delta f(a^{(t)}) \neq 0$ \\
$g^{+}(\lambda)=f(a^+-\lambda\nabla f(a^t))$ \\
$\lambda^+=argmin(g^+(\lambda))$
\end{enumerate}

\end{Thm}

\begin{Def}[Lokales und globales Minimum]
F�r $f: D\rightarrow \mathbb{R}$ ist $x\in D$ ein lokales Minimum, wenn mindestens eine Umgebung $N$ existiert, sodass \\$\forall_{y\in N}$ gilt: $f(y)>=f(x)$ \\
$x\in D$ ist dann ein globales Minimum, falls $N=D$.
\end{Def}

\subsection{Konvexit�t}

\begin{Def}[Konvexe Funktion]
Eine Funktion $f: \mathbb{R}^n\rightarrow\mathbb{R}$ ist dann eine konvexe Funktion, falls $\forall_{x,y\in D}$ gilt:\\ 
$f(\lambda x+ (1 - \lambda)y) \leq \lambda f(x)+(a-\lambda)f(y) \qquad \mid \forall_{\lambda\in[0,1]}$ \ref{fig:konvex}

%\definecolor{ffqqqq}{rgb}{1.,0.,0.}
%\definecolor{qqqqff}{rgb}{0.,0.,1.}
%\begin{dsafigure}
%\begin{tikzpicture}[line cap=round,line join=round,>=triangle 45,x=1.0cm,y=1.0cm]
%\draw[->,color=black] (-0.252729391848446,0.) -- (5.9098815146659724,0.);
%\foreach \x in {,0.5,1.,1.5,2.,2.5,3.,3.5,4.,4.5,5.,5.5}
%\draw[shift={(\x,0)},color=black] (0pt,2pt) -- (0pt,-2pt);
%\draw[->,color=black] (0.,-0.118522622381023) -- (0.,4.459842973966475);
%\foreach \y in {,0.5,1.,1.5,2.,2.5,3.,3.5,4.}
%\draw[shift={(0,\y)},color=black] (2pt,0pt) -- (-2pt,0pt);
%\clip(-0.252729391848446,-0.118522622381023) rectangle (5.9098815146659724,4.459842973966475);
%\draw[smooth,samples=100,domain=-0.252729391848446:5.9098815146659724] plot(\x,{(\x)});
%\draw [color=ffqqqq] (1.,1.)-- (1.,1.5);
%\draw [color=ffqqqq] (2.,2.)-- (2.,1.7);
%\draw [color=ffqqqq] (3.,3.)-- (3.,3.6);
%\draw [color=ffqqqq] (4.,4.)-- (4.,3.2);
%\begin{scriptsize}
%\draw [fill=qqqqff] (1.,1.5) circle (1.5pt);
%\draw[color=qqqqff] (1.0623956869668,1.7517884297439121) node {$(x_1,y_1)$};
%\draw [fill=qqqqff] (2.,1.7) circle (1.5pt);
%\draw[color=qqqqff] (2.0568910978064285,1.4790347346423591) node {$(x_2,y_2)$};
%\draw [fill=qqqqff] (3.,3.6) circle (1.5pt);
%\draw[color=qqqqff] (3.0513865086460568,3.8656295667809477) node {$(x_3,y_3)$};
%\draw [fill=qqqqff] (4.,3.2) circle (1.5pt);
%\draw[color=qqqqff] (4.0350131171814265,2.9499564475114486) node {$(x_4,y_4)$};
%\draw[color=ffqqqq] (1.0297892800540254,1.2647282599197103) node {$\varepsilon_1$};
%\draw[color=ffqqqq] (2.23079193467456,1.9466124976735928) node {$\varepsilon_2$};
%\draw[color=ffqqqq] (3.02964890403754,3.29089856638839) node {$\varepsilon_3$};
%\draw[color=ffqqqq] (4.236085959810204,3.6902879056442353) node {$\varepsilon_4$};
%\end{scriptsize}
%\end{tikzpicture}
%\caption{Beispiel einer Konvexen Funktion}
%\label{fig:konvex}
%\end{dsafigure}



\end{Def}

\subsubsection{Beispiel zur Konvexit�t}

\begin{align*}
f(u) &=\mid\mid u\mid\mid_2\qquad\mid u\in\mathbb{R}\\
f(\lambda u+(1-\lambda)v) &=\mid\mid\lambda u+(1-\lambda)v\mid\mid_2\\
&\leq\mid\mid\lambda u\mid\mid_2+\mid\mid(1-\lambda)v\mid\mid_2\\
&\leq\lambda\mid\mid u\mid\mid_2+(1-\lambda)\mid\mid v\mid\mid_2
\end{align*}
Dies kann dank der Dreiecksungleichung und der Ausnutzung der Linearit�t der Norm gezeigt werden.