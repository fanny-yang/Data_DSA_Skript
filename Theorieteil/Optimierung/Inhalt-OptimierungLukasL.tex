\subsection{Konvexe Mengen}
Im Rahmen der Optimierung von konvexen Funktionen ist es erforderlich, den Begriff der konvexen Menge einzuführen. Um diese Thematik anschaulich darzustellen, verwenden wir zunächst verschiedene geometrische Figuren in Abbildung \ref{figure:Grafik-Optimierung_KonvexeMengen}, von denen einige konvex sowie andere wiederum nicht konvex sind.

Ist es möglich eine direkte Verbindungsstrecke zwischen zwei beliebigen Punkten zu finden, die selbst ebenfalls in der Menge liegt, so ist die Menge konvex. Formal bedeutet das:

\begin{dsafigure}
\begin{center}
\includegraphics[width=0.4\textwidth]{\media Grafik-Optimierung_KonvexeMengen.pdf}
\label{figure:Grafik-Optimierung_KonvexeMengen}
\caption{Beispiele konvexer Mengen}
\end{center}
\end{dsafigure}

\begin{Def}[Konvexe Menge]
Eine Menge $X$ heißt konvex, falls für alle $x, y \in X$ und $\lambda \in [0,1]$ gilt, dass $\lambda x + (1 - \lambda) \in X$.
\end{Def}

\subsection{Beispiele konvexer Mengen}
\begin{description}
\item[Kugel:]
Ein Beispiel für eine konvexe Menge ist die Menge $S = \{x \in \mathbb{R}^{n} \mid \lVert x \rVert \le 1\}$ der Vektoren, die eine Kugel mit dem Radius $r = 1$ und Mittelpunkt im Ursprung beschreibt.
\item[Quadrant:] Auch die Quadranten des kartesischen Koordinatensystems lassen sich durch eine konvexe Menge erfassen; Beispiel ist der erste Quadrant $L = \{ x \in \mathbb{R}^{n} \mid x_i \ge 0, i=1,\dots,n\}$.
\item[Eistüte:] Die Eistüte bzw. der quadratischen Kegel $Q = \{(x, t) \mid x\in \mathbb{R}^{n-1}, t\in \mathbb{R}, \lVert x\rVert ^{2} \le t\}$ ist ein weiteres Beispiel für eine konvexe Menge.
\item[Box:] Eine Box mit der Seitenlänge 2, deren Mittelpunkt im Ursprung liegt, ist eine konvexe Menge $B = \{x \in \mathbb{R}^{n} \mid |x_{i}| \le 1, i = 1, \dots, n\}$.
\end{description}

\subsection{Konvexe Programme}
% \paragraph{Einführung}
Ziel eines konvexen Programmes ist die Optimierung einer Funktion $f$, die vom Parameter $x$ abhängig ist. Hierbei sind sowohl die Zielfunktion als auch eventualle Nebenbedingungen konvex.

\begin{Def}[Lösung konvexer Programme]
Sei $f: \mathbb{R}^{n} \rightarrow \mathbb{R}$ eine konvexe Funktion und $X \subseteq \mathbb{R}^{n}$ konvex, so ist die Lösung des Programms
\begin{align*}
  \text{minimiere}  \quad& f(x)\\
  \text{sodass} \quad & x\in X
\end{align*}
das globale Minimum der Funktion $f : X\rightarrow \mathbb{R}$.
\end{Def}
\noindent Im Folgenden betrachten wir einige Beispiele.
\medskip

\begin{description}
\item[Lineare Programme:] Für lineare Programme gilt $f(x) = \langle c, x\rangle$ und $X = \{x\in \mathbb{R}^n \mid Ax \leq b\}$.

Ein Beispiel für ein lineares Programm wäre
\begin{align*}
  \text{minimiere}  \quad& f(x) = x_1\\
  \text{sodass} \quad & x_1\leq 1, x_2\leq 1.
\end{align*}

\item[Quadratische Programme:]
Im Fall quadratischer Programme ist die Zielfunktion eine quadratische Funktion $f(x) = x^{\rm T} Q x$. Ein mögliches Beispiel ist
\begin{align*}
  \text{minimiere}  \quad& f(x) = x_1^2 + x_2^2\\
  \text{sodass} \quad & x_1\leq 1, x_1 + x_2\leq 1.
\end{align*}
\end{description}

