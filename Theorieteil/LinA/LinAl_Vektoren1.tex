%\documentclass{article}

%\usepackage[ngerman]{babel}
%\usepackage[utf8]{inputenc}
%\usepackage{color}
%\usepackage[a4paper,lmargin={2cm},rmargin={2cm},
%tmargin={2.5cm},bmargin = {2.5cm}]{geometry}
%\usepackage{amssymb}
%\usepackage{amsmath}
%\usepackage{graphicx}
%\usepackage{multicol}
%\usepackage{amsthm}

%\setlength{\columnsep}{1cm}
%\newtheorem{defi}{Definition}
%\newtheorem*{defi*}{Definition}

%\begin{document}
%\begin{multicols*}{2}

\section{Vektoren}
In der linearen Algebra nehmen Vektoren eine zentrale Rolle ein. Ein Vektor wird in der Schule als eine Menge an Pfeilen gelehrt, die parallel, gleichgerichtet und gleich lang sind. Betrachten wir jedoch nicht nur den dreidimensionalen Raum $\mathbb{R}^3$, sondern $\mathbb{R}^n$ f\"ur beliebige $n$, so ist die Vorstellung eines Pfeils nicht immer m\"oglich. Dies bedingt die Notwendigkeit einer anderen Definition.

Ein \textit{Vektor} ist eine geordnete Aufz\"ahlung von Objekten und wird auch \textit{Tupel} $(a_1, a_2,\dots,a_n)$ genannt. Entscheidend bei Tupeln ist die Reihenfolge der Objekte, die auch mehrfach vorkommen k\"onnen. Dies steht im Kontrast zum Begriff einer Menge.
\begin{Def} Vektoraddition \\
u + v = \begin{pmatrix} u_1 \\ u_2 \\ \dots \\ u_n \end{pmatrix} + \begin{pmatrix} v_1 \\ v_2 \\ \dots \\ v_n \end{pmatrix} = \begin{pmatrix} u_1 + v_1 \\ u_1+v_2 \\ \dots \\ u_1+v_n \end{pmatrix}\\ 
\end{Def}
\begin{Def} Skalarmultiplikation\\
\lambda u = \lambda \begin{pmatrix} u_1 \\ u_2 \\ \dots \\ u_n \end{pmatrix} = \begin{pmatrix} \lambda u_1 \\ \lambda u_2 \\ \dots \\ \lambda u_n \end{pmatrix}
\end{Def}



\subsection{Vektor- \& Unterraum}
Vektoren bilden die Elemente eines \textit{Vektorraumes V}. Addieren wir zwei Vektoren eines Vektorraumes oder multiplizieren wir sie mit einem Skalar, so ist die Summe bzw. das Produkt ebenfalls ein Element des Vektorraumes.
\vspace{10pt}
\newline Eine Menge von Vektoren ist dann ein Vektorraum, wenn gilt:
\begin{enumerate}
\item $u, v \in V \text{ }\Rightarrow u+v \in V$
\item $u\in V, \lambda \in \mathbb{R} \text{ }\Rightarrow \lambda u \in V$
\end{enumerate}
\vspace{10pt}
Ein \textit{Unterraum} $U$ ist eine Teilmenge eines Vektorraumes $V$. Es gelten die oben genannten Eigenschaften eines Vektorraumes.
\vspace{10pt}
\begin{enumerate}
\item $\text{ }u, v \in U,\text{ } u+v \in U$
\item $\text{ }u \in U, \lambda \in \mathbb{R} ,\text{ }\lambda u \in U$
\end{enumerate}



\subsection{Span}
Der span$(v_1, v_2,\dots,v_k)$ ist die Menge aller Vektoren, die mit den Linearkombinationen der Basisvektoren $v_1, v_2, \dots, v_k$ darstellbar sind, d.h.
\begin{align*}
\text{span}(v_1,\dots,v_k) = \{u \in V : u = \lambda_1v_1+\dots+\lambda_kv_k\} \\
\text{ mit } \lambda_1,\dots,\lambda_k \in \mathbb{R}
\end{align*}



\subsubsection*{Theorem}
Ein Span ist immer auch ein Vektorraum. \\
\begin{proof} 
Dies kann man folgenderma\ss en beweisen. \\
Angenommen, $V \in \text{span}(v_1,\dots,v_k)$.\\
Dann wissen wir durch die Definition des Spans, dass\\
\begin{align*}
u = \sum\limits_{i=1}^{k} \lambda_iv_i \\
v = \sum\limits_{i=1}^{k} \alpha_iv_i \\      
\end{align*}
Additionsregelung bei Vektorr\"aumen
\begin{align*}
	w_1 &= u+v\\
	&=\lambda_1 v_1+\dots+\lambda_k v_k+\alpha_1 v_1+\dots+\alpha_k v_k \\
	&=(\lambda_1+\alpha_1) v_1+\dots+(\lambda_k+\alpha_k) v_k
\end{align*}
Multiplikationsregelung bei Vektorr\"aumen 
\begin{align*}
	w_2 &= \lambda_1 v_1+\dots+\lambda_k v_k\\
	\lambda w_2 &= (\lambda \cdot \lambda_1) v_1+\dots+(\lambda \cdot \lambda_k)v_k
\end{align*}
\end{proof}



\subsection{Basis \& Dimension}
Jeder Vektorraum kann durch eine Menge linear unabh\"angiger Basisvektoren dargestellt werden. Diese Menge wird \textit{Basis} eines Vektorraumes genannt. Es gibt beliebig viele unterschiedliche Basen. Die Anzahl der Basisvektoren bestimmt die \textit{Dimension p} eines Vektorraums.



\subsection{Lineare Unabh\"angigkeit}
Die Vektoren $v_1,\dots,v_k$ mit $v_i \in V$ sind \textit{linear unabh\"angig}, falls $\lambda_iv_i+\dots+\lambda_kv_k = 0$ nur f\"ur $\lambda_i=\dots=\lambda_k= 0$ gilt. 

%\end{multicols*}
%\end{document}
