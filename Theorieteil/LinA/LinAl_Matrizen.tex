
\section{Lineare Abbildungen}

In diesem Abschnitt werden wir die elementarste Form einer Funktion kennenlernen, die einen Vektorraum auf einen anderen Vektorraum abbildet. Diese Funktionen nennt man \textit{lineare Abbildungen}.
\begin{Def}
	\label{Def:Def_1}
Eine Funktion $f:U \stackrel{}{\rightarrow} V$ ist linear, wenn folgende Bedingungen gelten: \\ 
1. $f(u+v) = f(u)+f(v) \forall \ u,v \in U$\\
2. $f(\lambda v) = \lambda f(v) \ \forall \ v \in U \ \text{und} \ \lambda \in \mathbb{R}$
\end{Def}

Das besondere an solchen Abbildungen ist, dass sie von \textit{Matrizen} eindeutig definiert werden, wobei $n$ die Dimension von $U$ und $m$ die Dimension von $V$ ist. Wie Matrizen definiert sind und wie wir sie verwenden können, lernen wir in nächsten Kapitel.

\subsection{Matrix}
Eine Matrix ist eine spezielle Anordnung von $n\cdot m$ Zahlen in $n$ Zeilen und $m$ Spalten. Man kann eine Matrix aber auch als eine Zusammenfassung von $m$ Vektoren mit $n$ Komponenten ansehen, dabei bildet ein Vekotor eine Spalte. Die Komponente in der $i$-ten Zeile und $j$-ten Spalte wird als $a_{ij}$ geschrieben und die ganze Spalte mit $a_j$ notiert.\\
\begin{equation*}
A= \left(
   \begin{array}{cccc}
	  a_{11} & a_{12} & ... & a_{1m}\\
		a_{21} & a_{22} & ... & a_{2m}\\
		...   & ...   & ... & ...  \\
		a_{n1} & a_{n2} & ... & a_{nm}
	 \end{array}
	 \right) = (a_1, a_2, ..., a_m)\\
\end{equation*}
\\
Da wir nun wissen was eine Matrix ist, führen wir zwei neue Operationen ein.
\begin{enumerate}
	\item Matrix-Vektor-Multiplikation
	\item Matrix-Matrix-Multiplikation
\end{enumerate}

\subsection{Matrix-Vektor-Multiplikation}
Bei einer Matrix-Vektor-Multiplikation wird eine Matrix $A \in  \mathbb{R}^{n \times m}$ mit einem Vektor $v \in \mathbb{R}^m$ multipliziert. Dabei wird jede Zeile der Matrix mit dem Vektor skalarmultipliziert; deshalb ist es wichtig, dass die Matrix so viele Spalten wie der Vektor Dimensionen hat. Eine Multiplikation zwischen Matrizen und Vektoren, bei denen die Anzahl nicht übereinstimmt, ist nicht definiert.
\begin{align*}
	Av &= 
	\left(
   \begin{array}{cccc}
	  a_{11} & a_{12} & ... & a_{1m}\\
		a_{21} & a_{22} & ... & a_{2m}\\
		...   & ...   & ... & ...  \\
		a_{n1} & a_{n2} & ... & a_{nm}
	 \end{array}
	\right) 
	\cdot 
	\left(
	 \begin{array}{c}
	  v_1\\
		v_2\\
		...\\
		v_m
	 \end{array}
	\right)\\
	&= \left(
	 \begin{array}{c}
	 a_{11} \cdot v_1 + a_{12} \cdot v_2 + ... + a_{1m} \cdot v_m\\
	 a_{21} \cdot v_1 + a_{22} \cdot v_2 + ... + a_{2m} \cdot v_m\\
		...\\
	 a_{n1} \cdot v_1 + a_{n2} \cdot v_2 + ... + a_{nm} \cdot v_m\\
	 \end{array}
	\right)
\end{align*}

Allgemein lässt sich für die Matrix-Vektor-Multiplikation sagen
\begin{equation*}
(Av)_i = \sum_{j=1}^{n}a_{ij}v_j 
\end{equation*}

\subsection{Matrix-Matrix-Multiplikation}
Eine Matrix-Matrix-Multiplikation ist nichts anderes als mehrere Matrix-Vektor-Multiplikationen hintereinander. Haben wir zwei Matrizen $A \in \mathbb{R}^{n \times m}$ und $B \in \mathbb{R}^{m \times p}$ und berechnen $AB$ nehmen wir jede Spalte von $B$ als Vektor und multiplizieren ihn mit $A$ wie oben beschrieben. Die Vektoren die man als Ergebnisse erhält, fasst man wieder in einer Matrix zusammen mit der Dimension $\mathbb{R}^{n \times p}$. Allgemein können wir sagen:
\begin{equation*}
	(AB)_{il} = \sum_{n=i}^{m}a_{ij} \times b_{jl}
\end{equation*}

\subsubsection*{Übung}
Beweise, dass eine Komposition $h(x) = f \circ g (x)$ aus den linearen Abbildungen $f$ und $g$, auch eine lineare Abbildung ist.

\begin{proof}
Da $f$ und $g$ linear sind gelten die Bedingungen aus Definition \ref{Def:Def_1}:
\begin{align*}
	h(u+v) &= (f \circ g)(u+v)\\
	&= f(g(u+v))\\
	&= f(g(u)+g(v))\\
	&= f(g(u)) + f(g(v))\\
	&= (f\circ g)(u) + (f\circ g)(v) = h(u) + h(v)
\end{align*}
Des Weiteren gilt:
\begin{align*}
	h(\lambda v) &= (f \circ g)(\lambda v)\\
	&=f(g(\lambda v))\\
	&= f(\lambda g(v))\\
	&=\lambda f(g(v))\\
	&=\lambda (f \circ g)(v) = \lambda h(v)
\end{align*}
Da beide Bedingungen aus Definition \ref{Def:Def_1} für $h$ erfüllt sind, ist auch eine Komposition aus zwei linearen Abbildungen linear.
\end{proof}

