\section{Analysis}
  

Analysis ist ein wichtiges Teilgebiet der Mathematik. Sie ist Grundlage für Optimierung und somit auch von großer Bedeutung beim Maschinellen Lernen. \\
Im Kurs war für uns besonders die Stetigkeit und Differenzierbarkeit von ein- und mehrdimensionalen Funktionen wichtig. \\
Dazu eignen wir uns vorerst die theoretischen Grundlagen der Analysis an. \\

\subsection{Grenzwert}
Grenzwert einer Funktion L ist Grenzwert der Funktion $f(x)$ für x gegen a, wenn für alle $\epsilon > 0$ ein $\delta > 0$ gibt, sodass für alle $|x-a| < \delta$ gilt \begin{equation*}|f(x)-L|<\epsilon\end{equation*}

\subsection{Stetigkeit}
Eine Funktion $f:\mathds{R}\rightarrow \mathds{R}$ ist stetig in x, wenn $lim_{w\rightarrow x}f(w)=f(x)$ auf ihrem gesamten Definitionsbereich gilt, wenn sie stetig in x für alle $x\in D$ ist.

\subsection{Differenzierbarkeit}
Eine Funktion f:$\mathds{R}\rightarrow\mathds{R}$ ist an einer Stelle x differenzierbar, wenn eine lineare Abbildung $ l_ x (h)$ und ein $r(h)$ existieren, sodass \begin{equation*}f(x+h)=f(x)+l_{x}(h)+r(h)\end{equation*} mit dem Grenzwert \begin{equation*}lim_{h\rightarrow 0} \frac{r(h)}{h}= 0\end{equation*}gilt.

\subsection{Gradient}
Der Gradient $\nabla$ ist die Ableitung eines Vektors, wobei jeder Eintrag die partielle Ableitung der entsprechenden Komponente ist:
$f:\mathds{R}^m\rightarrow\mathds{R}$.
\begin{equation} \begin{split} \nabla f(x) = \left( \begin{array}{c}
\partial_{x1} f(x) \\
\partial_{x2} f(x) \\
... \\
\partial_{xm} f(x)
\end{array}
\right)
\end{split} \end{equation} 
Der Gradient zeigt immer in Richtung des steilsten Abstiegs einer Funktion, weshalb wir das Gradientenabstiegsverfahren verwenden können, um das Minimum einer mehrdimensionalen Funktion zu finden.
\subsection{Kettenregel}
Gegeben seien die differenzierbaren Funktionen $f:\mathds{R}^k\rightarrow\mathds{R}^n$, $g:\mathds{R}^n\rightarrow{R}^k$ und $h(x)=f(g(x))$. Daraus folgt durch die Kettenregel, dass Dh(x)=Df(g(x))*Dg(x) ist.
\subsection{Summenregel}
Die Summenregel besagt, dass die Ableitung der Funktion $h(x)=f(x)+g(x)$ \begin{equation*} D(f + g)(x)=Df(x)+Dg(x)\end{equation*} ist.
\subsection {Produktregel}
Bei zwei differenzierbaren Funktionen $f:\mathds{R}^m\rightarrow\mathds{R}$ und $g:\mathds{R}^m\rightarrow\mathds{R}$ gilt für das Produkt der Funktionen ($h(x)=f(x)*g(x)$) die Ableitung \begin{equation*}Dh(x)=f(x)*Dg(x)+Df(x)*g(x)\end{equation*}
\\
Um uns Klarheit über diesen Bereich der Mathematik zu verschaffen, l\"osen wir die folgenden Aufgaben.
\vspace{15pt}

\subsection{Aufgabe 1:} 
Aufgabenstellung:
Beweisen Sie, dass eine Funktion $f:\mathds{R} \rightarrow \mathds{R}$, welche differenzierbar an der Stelle $x \in R$ ist auch stetig an der Stelle $x \in R$ ist.\\ %Satz ist noch nicht optimal formuliert

Lösung:

Zuerst überlegen wir, welche Bedingungen bereits in der Aufgabenstellung gegeben sind. \\
Eine Funktion ist an der Stelle x differenzierbar, wenn gilt:

\begin{equation}f(x+h)=f(x)+l_x(h)+ r(h) \text{ mit } \lim\limits_{h \rightarrow 0} \frac{r(h)}{h}= 0 \end{equation} 

Diese Funktion wäre außerdem stetig, wenn $\lim\limits_{w \rightarrow x} f(w)=f(x)$.\\
Da die Funktion $f(x+h)$ äquivalent zur Funktion $f(w)$ sein soll, ersetzen wir in der Bedingung für die Stetigkeit $f(w)$ mit der gesamten Funktion von $f(x+h)$.  
Somit gilt: \\ \\
$\lim\limits_{h \rightarrow 0} f(x)+l_x(h)+r(h)=f(x)$. \\ \\

Damit die Bedingung der Stetigkeit erfüllt ist, muss der Grenzwert dieser Funktion gleich dem Funktionswert sein. Um diese Bedingung zu erfüllen, müssen wir beweisen, dass die Teile der Funktion $l_x(h)$ und $r(h)$ gegen Null gehen. Wir beginnen mit dem Teil $r(h)$.  Würde $r(h)$ nicht gegen Null gehen, so würde $\frac{r(h)}{h}$ nicht gegen Null gehen. Die Funktion $r(h)$ muss gegen Null gehen, da die Bedingung der Differenzierbarkeit gilt. Jetzt müssen wir zeigen, dass $l_x(h)$ auch gegen Null geht. Dieser Teil der Funktion kann auch als $f'(x) \cdot h$ dargestellt werden. Da wir annehmen, dass h gegen Null geht und somit ein Faktor von $l_x(h)$ Null ist, wird die Funktion Null. Damit ist bewiesen, dass $\lim\limits_{h \rightarrow 0} f(x)=f(x)$. \\

\vspace{15pt}

\subsection{Aufgabe 2:}
Bei der nachfolgenden Aufgabe soll bewiesen werden, dass es immer möglich ist für eine differenzierbare Funktion f einen Skalar $\lambda$ zu finden, damit folgende Aussage gilt: \\ \\
$(1) f(x + \lambda \nabla f(x)) \leq f(x)$ \\ \\
Dadurch lässt sich die Ungleichung (1) in folgende Form umwandeln: \\ \\
$f(x) + Df(x)h + r(h) \leq f(x)$  \\ \\
Hierbei ist h = $\lambda \nabla$ f(x) und r(h) das Restglied %Muss eventuell erklärt werden 
. \\
Das Ganze kann man noch weiter umschreiben, indem man das Skalarprodukt bildet: \\ \\
$f(x) + \lambda <\nabla f(x), \nabla f(x)> + r(h) \leq f(x)$ \\ \\
Anschließend kann man von der Regel Gebrauch machen, dass sich ein Skalarprodukt, das in der Form $<u,u>$ vorliegt, in die Auspr\'agung der Euklidischen Norm von $||u||^2$ umgewandelt werden kann. \\
Daraus folgt: \\ \\
$f(x) + \lambda \cdot || \nabla f(x)||^2 + r(h) \leq f(x)$ \\ \\
Um nun die obige Ungleichung (1) zu beweisen, muss man zeigen, dass \\ $\lambda ||\nabla f(x)||^2 + r(h) \leq 0$ ist.\\
Dafür wird $\lambda$ ausgeklammert. Anschließend erhält man:\\
Da $||\nabla f(x)||$ quadriert wird, wird dieser Term automatisch positiv. Da man den gesamten Term aber negativ haben will, wählt man $\lambda < 0.$ Schließlich kennt jeder die Regel, dass ein Produkt negativ wird, falls eine ungerade Zahl von Faktoren negativ ist. Da wir $\lambda$ negativ gewählt haben, muss \\ $||\nabla f(x)||^2$$(2) \lambda (||\nabla f(x)||^2 + \frac{r(h)}{\lambda}$ positiv sein.\\
Bei $||\nabla f(x)||^2||$ stellt dies aus den eben genannten Gr\"unden kein Problem dar. Deswegen muss nur noch gezeigt werden, dass es mindestens einen Wert von $r(h)$ gibt, der gr\"oßer oder gleich 0 ist. Ansonsten würde der Wert in der Klammer negativ werden k\"onnen und damit die unspr\"ungliche Aussage (1) widerlegen.\\
Damit wir zeigen k\"onnen, dass $r(h) >= 0$ ist, formten wir zuerst den Term (2) um:\\

$\lambda (||\nabla f(x)||^2 + \frac{r(h)}{\lambda})$ \\

$\Leftrightarrow \lambda(||\nabla f(x)||^2 + \frac{r(\lambda \nabla f(x))}{\lambda})$ \\

Bei dieser Umformung wurde h wieder mit $\lambda\nabla f(x)$ ersetzt.

$\leftrightarrow \lambda ||\nabla f(x)|| \cdot (||\nabla f(x)|| + \frac{r(\lambda \nabla f(x))}{\lambda || f(x)||})$


Durch diese Umformungen schlussfolgerten wir, dass f\"ur die Funktion $h(x)$ folgendes gelten muss:\\
(3) $h(\lambda)=||\nabla f(x)|| + \frac{r(\lambda\nabla f(x))}{\lambda||\nabla f(x)||} >=0$ \\

Da in $||\nabla f(x)$ kein $\lambda$ steht ist dieser Term konstant. Gleichzeitig wird durch das Bilden der euklidischen Norm festgelegt, dass ein Wert größe/gleich null vorliegt. \\ %der Satz ist noch nicht wiklich schön formuliert
Deswegen muss nur noch gezeigt werden, dass dder zweite Summand ebenfalls positiv werden kann. Dazu bildeten wir den Grenzwert. \\ %Die Zeitformen stimmen sehr oft nicht
$\lim\limits_{h \rightarrow 0}{\frac{r(\lambda \nabla f(x))}{\lambda ||\nabla f(x)||}} = \frac{r(\lambda \nabla f(x))}{||\lambda  \nabla f(x)||}$ \\ %hier kommt noch kein h vor. Umschreiben?

Man kann $\lambda \nabla f(x)$ wieder durch h ersetzen, woraus folgt: \\
$\lim\limits_{h \rightarrow 0} \frac{r(h)}{h}$ \\

Der Grenzwert des Restbetrages strebt gegen Null. Daraus lässt sich schließen, dass es einen Punkt gibt, ab welchem die Ungleichung der Funktion $h(\lambda)$ (3) erfüllt ist und damit auch die anfängliche Aussage (1). \\ %ist hier h(\lambda) und (3) beies nötig?
\vspace{15pt}

\subsection{Aufgabe 3:} 

In der letzten Aufgabe soll die Produktregel bewiesen werden. \\
Dafür waren zwei Funktionen gegeben: Einmal die Funktion f, die vom $\mathds{R}^m$ in den $\mathds{R}$ abbildet und die Funktion g, die ebenfalls vom $\mathds{R}^m$ in den $\mathds{R}$ abgebildet wird. \\
Es sind also die Funktion f(x) mit der Ableitung $f(x+h) = f(x) \cdot Df(x) \cdot h \cdot r_{f}(h)$ und die Funktion g(x) mit der Ableitung $g(x+h) = f(x) \cdot Dg(x) \cdot h \cdot r_{g}(h)$ gegeben. \\
Die Produktregel besagt, dass für die Funktion $h(x) = f(x) \cdot g(x)$ die Ableitung $Dh(x) = f(x) \cdot Dg(x) + Df(x) \cdot g(x)$ %Reihenfolge eventuell anpassen
 ist. \\
Damit dies bewiesen werden kann schreiben wir zuerst die Ableitung von $h(x+v)$ als Kombination aus den Ableitungen von $f(x)$ und $g(x)$ auf.

\begin{align} \begin{split} h(x+v) & = (f(x) + Df(x) \cdot v \\ & + r_{g}(h)) \cdot (g(x) \\ & + Dg(x) \cdot v + r_{g}(v) \end{split} \end{align}

Dieser Term wird ausmultipliziert und wir erhalten: 

\begin{equation}
\begin{split} h(x+v) & = f(x) \cdot g(x) \\ & + f(x)  \cdot D_{g}(x) \cdot v \\ & + f(x) \cdot r_{g}(v) \\ & + Df(x) \cdot v \cdot g(x) \\ & +  Df(x) \cdot v \cdot D_{g}(x) \cdot v  \\ & + Df(x) \cdot v \cdot r_{g}(v) \\ & +  r_{g}(v) \cdot g(x) \\ & + r_{g}(x) \cdot Dg(x) \cdot v \\ & + r_{g}(v) \cdot r_{g}(v) \end{split} \end{equation}   %Die Formel passt nicht in eine Zeile. Umbruch?

Von den ganzen Summanden müsssen letztendlich alle eliminiert werden, bis auf 
\begin{equation} f(x) \cdot g(x) + f(x) \cdot Dg(x) \cdot v + Df(x) \cdot v \cdot g(x) \end{equation} 
Dafür müssen wir zeigen, dass die übrigen Summanden gegen Null gehen. Um das zu erreichen bilden wir den Grenzwert:
%riesige Formel hier

Damit die Summanden einzeln "aufgesplittet" werden können, wenden wir die Dreiecksungleichung %Wurde die weiter vorne schon erklärt?
an:
%andere große Formel

Nun können wir von jedem Summanden einzeln den Limes bilden, um zu zeigen, dass der gesamte Grenzwert gegen Null geht. Dies ist aber nur der Fall, falls der Limes jedes Summanden gegen Null geht. Wir zeigen das exemplarisch für 

\begin{equation} \lim\limits_{v \rightarrow 0} \frac{||Df(x) \cdot r_{g}(v)||}{||v||} \end{equation}

Df(x) ist eine Matrix mit einer Zeile und m Spalten in der jeweils c steht. Daraus folgern wir:

\begin{equation} \lim\limits_{v \rightarrow 0} \frac{m \cdot c \cdot ||v|| \cdot ||r_{g}(v)||}{||v||} \end{equation}
\begin{equation}m \cdot c \cdot \lim\limits_{v \rightarrow 0} \frac{ ||v|| \cdot ||r_{g}(v)||}{||v|| = 0} \end{equation} 

Damit haben wir gezeigt, dass der Summand gegen Null geht. Das Selbe lässt sich auch mit den anderen Summanden zeigen und auch diese gehen alle gegen Null. \\
Die Produktregel ist damit bewiesen.