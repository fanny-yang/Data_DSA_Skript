\section{Einleitung}
\authors{Lukas Kamm}
\begin{quote}
\enquote{Okay, Google, wie komme ich von hier am schnellsten zum Alexanderplatz?}\\

\enquote{Geh' zu Fuß 100 Meter zur U-Bahn-Haltestelle Moritzplatz und nimm die U8 in Richtung Osloer Straße. Die nächste U-Bahn fährt in 6 Minuten.}
\end{quote}

Ein Szenario wie es vermutlich jeder Smartphone-Benutzer kennt. Die Spracheingabe wird verwendet, um seinem Gerät Fragen zu stellen: Nach dem Wetter, aktuellen Aktienkursen oder dem Weg von A nach B. Doch wie ist es möglich, Sprache zu erkennen? Sie setzt sich immerhin komplex aus verschiedenen Komponenten wie den Lauten, der Tonhöhe, Betonung und Unterschieden in der Lautstärke zusammen, verläuft im zeitlichen Kontext und variiert dazu von Sprecher zu Sprecher. Die Antwort lautet: \emph{maschinelles Lernen}. Das Prinzip von Lernalgorithmen ist es, den Computer anhand verschiedener \emph{Trainingsbeispiele} darauf zu trainieren, wie etwa das Wort \glqq Alexanderplatz\grqq klingt, um es im Umkehrschluss in einem Stück aufgezeichneter Sprache wiederzuerkennen. Diese Lernalgorithmen haben sich in den vergangenen zehn Jahren in der Spracherkennung etabliert, spielen aber auch in anderen Feldern eine bedeutende Rolle.

Ein anderes Beispiel für die Benutzer traditioneller Kommunikationsmedien: Die Deutsche Post stellt täglich rund 64 Millionen Briefe zu (\url{https://www.deutschepost.de/de/q/qualitaet_gelb.html}). Da bei dieser Menge an zu identifizierenden Postleitzahlen, Straßen und Orten der Mensch zu langsam ist, kommen dort Lernalgorithmen zur Texterkennung von gedruckter und handgeschriebener Schrift zum Einsatz, um die Anschrift der Empfänger digital zu erfassen und die Sendungen ins richtige Fach zu transportieren.

Ein verbreitetes Konzept der künstlichen Intelligenz ist das künstliche neuronale Netz. Der Mensch bedient sich hierbei an der Natur, indem die Informationsverarbeitung im Nervensystem des menschlichen Gehirns und Rückenmarks mathematisch beschrieben und auf Computern simuliert wird.

Erste Ansätze zu neuronalen Netzen gab es in den 1940er Jahren. Bald zeigten sich jedoch in der Praxis Grenzen dieser Konzepte mangels effizienter Algorithmen und der geringen Rechenleistung damaliger Rechenanlagen. Erst in den 1970er und 1980er Jahren, beispielsweise durch die Entwicklung der sogenannten \emph{Backpropagation} (Fehlerrückführung) und der wachsenden Leistungsfähigkeit von Rechnern, konnten neuronale Netze praktikabel in der Erkennung von Text, Sprache, Gesichtern, Mustern in der medizinischen Diagnostik, etwa bei einer fMRT-Aufzeichnung oder in der Zeitreihenanalyse von Wetter oder Aktien eingesetzt werden.

Im Kurs möchten wir zunächst ein Verständnis für die mathematische Modellierung verschiedener Lernalgorithmen gewinnen. Dazu eignen wir uns anfangs Grundlagen der höheren Mathematik, konkret aus den Bereichen lineare Algebra, Analysis, Wahrscheinlichkeitstheorie und der Optimierung an. Anschließend lernen wir \emph{Support Vector Machines} kennen. Ein Algorithmus, der darauf trainiert wird, ein Datum anhand verschiedener Eigenschaften richtig zu klassifizieren. Des Weiteren betrachten wir künstliche neuronale Netze und die Umsetzung und Anwendung beider in Software.

Zunächst beginnen wir mit den mathematischen Grundlagen.