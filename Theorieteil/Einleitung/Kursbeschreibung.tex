\section{Aus der Kursbeschreibung}
\authors{Philipp Moritz KL, Fanny Yang KL}
Künstliche Intelligenz erzielte in den letzten Jahren nach einer längeren Phase der Stagnation enorme Fortschritte, ermöglicht durch große Datenmengen (\enquote{Big Data}) und die erfolgreiche Implementierung tiefer neuronaler Netze (\enquote{Deep Learning}). Moderne Spracherkennungsmethoden in Smartphones, die auf dieser Technik basieren, erzielen oft Erkennungsquoten die mit denen von Menschen vergleichbar sind. Auch bei der Post wird der Großteil aller handschriftlichen Adressen vom Computer erkannt und in digitale, maschinell interpretierbare Zeichen umgewandelt.

Außer bei der Mustererkennung wird künstliche Intelligenz auch in anderen Anwendungen erfolgreich eingesetzt: Etwa wenn Amazon oder Netflix basierend auf vergangenem Konsumverhalten Filme und Produkte empfehlen, die dem eigenen Geschmack entsprechen. Oder wenn Google lernt, zwischen Spam und wichtigen E-Mails zu unterscheiden. Zu weiteren Anwendungsbeispielen zählen Logistik, Computerlinguistik, fMRI oder Genetik. 
Auch in der Entwicklung von selbstfahrenden Autos und Brain-Machine-Interfaces spielt die  künstliche Intelligenz eine tragende Rolle.

Wodurch wurde der überwältigende Erfolg dieser Technologien möglich gemacht?
Auf der einen Seite durch neue Erkenntnisse in der Theorie des maschinellen Lernens und der Optimierung. Auf der anderen Seite durch empirische Ergebnisse wie die erfolgreiche Anwendung von Backpropagation auf wesentlich größere Modelle als bisher, die nun trainiert und effektiv eingesetzt werden können.

In unserem Kurs wollen wir uns sowohl mit diesen theoretischen Konzepten als auch mit konkreten Anwendungen vertraut machen. Ziel ist es, dass die Teilnehmer die relevante Mathematik lernen und verstehen, um selbst einfache Modelle wie Support Vector Machines oder neuronale Netze zu implementieren. Hierfür beschäftigen wir uns zum Beispiel mit mehrdimensionaler Analysis, linearer Algebra, Wahrscheinlichkeitstheorie, mathematischer Statistik und konvexer Optimierung. Als Anwendungsbeispiele werden wir unsere Programme mit öffentlich zugänglichen Datensätzen, wie z.B. MNIST, trainieren und testen. 
