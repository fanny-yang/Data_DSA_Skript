\documentclass{newlayout}
%Bitte hier den enstprechenden Ort einsetzen z.B. Braunschweig und die Akademienummer
\Akademie{Braunschweig}{2015}{1}

\usepackage[english, ngerman]{babel}
\usepackage{misc}
\usepackage{multicol}
\usepackage{csquotes}

%\usepackage{amsmath}%wird automatisch durch newlayout.cls geladen
\usepackage{amssymb}
\usepackage{amsfonts}
\usepackage{tikz}
\usepackage{dsfont}
\usepackage{blindtext}
\usepackage{tipa}
\usepackage{algpseudocode}
\usepackage{url}
%\def\UrlBreaks{\do\a\do\b\do\c\do\d\do\e\do\f\do\g\do\h\do\i\do\j\do\k\do\l%
% \do\m\do\n\do\o\do\p\do\q\do\r\do\s\do\t\do\u\do\v\do\w\do\x\do\y\do\z\do\0%
% \do\1\do\2\do\3\do\4\do\5\do\6\do\7\do\8\do\9\do\-\do\_\do\/\do\%}
% \urlstyle{same}

\usetikzlibrary{arrows}

% hinzugefügt, um Fehler 'pdfTeX error (font expansion): auto expansion is only possible with scalable' zu vermeiden
\usepackage{lmodern}
\setkomafont{descriptionlabel}{\normalfont\bfseries}
\addtokomafont{paragraph}{\normalfont}
\usepackage{footnote}
\usepackage[flushmargin,hang,ragged]{footmisc}
\deffootnote{1em}{1em}{%
\textsuperscript{\thefootnotemark\ }
}
%\setlength{\abovedisplayskip}{5pt}
%\setlength{\belowdisplayskip}{5pt}

\newcommand{\media}{media/}
\newcommand{\norm}[1]{\left\lVert#1\right\rVert}

%%%%%Mathe-Definitionen
\newtheorem{Def}{Definition}
\newtheorem{Sat}{Satz}
\newtheorem{Bew}{Beweis}
\newtheorem{Lem}{Lemma}
\newtheorem{Bsp}{Beispiel}
\newtheorem{Thm}{Theorem}

\setlength\abovedisplayshortskip{0pt}
\setlength\belowdisplayshortskip{0pt}
\setlength\abovedisplayskip{3pt}
\setlength\belowdisplayskip{3pt}
%%%%Ende Mathe-Definitionen

\begin{document}
%\input{titel}
\setcounter{page}{3}

\setcounter{tocdepth}{1}
\tableofcontents

\setcounter{secnumdepth}{1}

\setcounter{page}{7}
\setcounter{chapter}{0}

%Angabe, bis zu welcher Stufe die sections im Text nummeriert werden sollen.
\settocdepth{2}

\course{2}{Learning from Data}%%% 
\begin{coursetitle}
  \centerline{Learning from Data} 
  \bigskip
  \Large \centerline{Mathematische Theorien und Methoden für das digitale Zeitalter}
  \bigskip
 \includegraphics[width=\columnwidth]{\media LearningFromData_Logo}
 \label{fig:titelbild}
  \bigskip
\end{coursetitle}
%\begin{dsafigure}
%\begin{center}
%\includegraphics[width=.9\columnwidth]{Titelbild-fehlt.png}
%\caption{meine Bildunterschrift}
%label{fig:meinbild}
%\end{center}
%\end{dsafigure}

%Einleitung
\section{Aus der Kursbeschreibung}
\authors{Philipp Moritz KL, Fanny Yang KL}

Test.

Jeder Kurs beginnt mit einem Teil "`1 Aus der Kursbeschreibung"', entnommen aus dem Programmheft.
Hier sollte nur der den Inhalt des Kurses beschreibende
Teil der Kursbeschreibung erscheinen, nicht
die Erwartungen an die Teilnehmenden.




Gegen Ende des Kurses können die Teilnehmenden je nach Interesse weitere Themen diskutieren. Es kann zum Beispiel das ${H_2}^+$-Ion als ein
\section{Einleitung}
\authors{Lukas Kamm}
\begin{quote}
\enquote{Okay, Google, wie komme ich von hier am schnellsten zum Alexanderplatz?}\\

\enquote{Geh' zu Fuß 100 Meter zur U-Bahn-Haltestelle Moritzplatz und nimm die U8 in Richtung Osloer Straße. Die nächste U-Bahn fährt in 6 Minuten.}
\end{quote}

Ein Szenario wie es vermutlich jeder Smartphone-Benutzer kennt. Die Spracheingabe wird verwendet, um seinem Gerät Fragen zu stellen: Nach dem Wetter, aktuellen Aktienkursen oder dem Weg von A nach B. Doch wie ist es möglich, Sprache zu erkennen? Sie setzt sich immerhin komplex aus verschiedenen Komponenten wie den Lauten, der Tonhöhe, Betonung und Unterschieden in der Lautstärke zusammen, verläuft im zeitlichen Kontext und variiert dazu von Sprecher zu Sprecher. Die Antwort lautet: \emph{maschinelles Lernen}. Das Prinzip von Lernalgorithmen ist es, den Computer anhand verschiedener \emph{Trainingsbeispiele} darauf zu trainieren, wie etwa das Wort \glqq Alexanderplatz\grqq klingt, um es im Umkehrschluss in einem Stück aufgezeichneter Sprache wiederzuerkennen. Diese Lernalgorithmen haben sich in den vergangenen zehn Jahren in der Spracherkennung etabliert, spielen aber auch in anderen Feldern eine bedeutende Rolle.

Ein anderes Beispiel für die Benutzer traditioneller Kommunikationsmedien: Die Deutsche Post stellt täglich rund 64 Millionen Briefe zu (\url{https://www.deutschepost.de/de/q/qualitaet_gelb.html}). Da bei dieser Menge an zu identifizierenden Postleitzahlen, Straßen und Orten der Mensch zu langsam ist, kommen dort Lernalgorithmen zur Texterkennung von gedruckter und handgeschriebener Schrift zum Einsatz, um die Anschrift der Empfänger digital zu erfassen und die Sendungen ins richtige Fach zu transportieren.

Ein verbreitetes Konzept der künstlichen Intelligenz ist das künstliche neuronale Netz. Der Mensch bedient sich hierbei an der Natur, indem die Informationsverarbeitung im Nervensystem des menschlichen Gehirns und Rückenmarks mathematisch beschrieben und auf Computern simuliert wird.

Erste Ansätze zu neuronalen Netzen gab es in den 1940er Jahren. Bald zeigten sich jedoch in der Praxis Grenzen dieser Konzepte mangels effizienter Algorithmen und der geringen Rechenleistung damaliger Rechenanlagen. Erst in den 1970er und 1980er Jahren, beispielsweise durch die Entwicklung der sogenannten \emph{Backpropagation} (Fehlerrückführung) und der wachsenden Leistungsfähigkeit von Rechnern, konnten neuronale Netze praktikabel in der Erkennung von Text, Sprache, Gesichtern, Mustern in der medizinischen Diagnostik, etwa bei einer fMRT-Aufzeichnung oder in der Zeitreihenanalyse von Wetter oder Aktien eingesetzt werden.

Im Kurs möchten wir zunächst ein Verständnis für die mathematische Modellierung verschiedener Lernalgorithmen gewinnen. Dazu eignen wir uns anfangs Grundlagen der höheren Mathematik, konkret aus den Bereichen lineare Algebra, Analysis, Wahrscheinlichkeitstheorie und der Optimierung an. Anschließend lernen wir \emph{Support Vector Machines} kennen. Ein Algorithmus, der darauf trainiert wird, ein Datum anhand verschiedener Eigenschaften richtig zu klassifizieren. Des Weiteren betrachten wir künstliche neuronale Netze und die Umsetzung und Anwendung beider in Software.

Zunächst beginnen wir mit den mathematischen Grundlagen.

%\documentclass{article}

%\usepackage[ngerman]{babel}
%\usepackage[utf8]{inputenc}
%\usepackage{color}
%\usepackage[a4paper,lmargin={2cm},rmargin={2cm},
%tmargin={2.5cm},bmargin = {2.5cm}]{geometry}
%\usepackage{amssymb}
%\usepackage{amsmath}
%\usepackage{graphicx}
%\usepackage{multicol}
%\usepackage{amsthm}

%\setlength{\columnsep}{1cm}
%\newtheorem{defi}{Definition}
%\newtheorem*{defi*}{Definition}

%\begin{document}
%\begin{multicols*}{2}

\section{Vektoren}
In der linearen Algebra nehmen Vektoren eine zentrale Rolle ein. Ein Vektor wird in der Schule als eine Menge an Pfeilen gelehrt, die parallel, gleichgerichtet und gleich lang sind. Betrachten wir jedoch nicht nur den dreidimensionalen Raum, sondern $\mathbb{R}\textsuperscript{n}$, so ist die Vorstellung eines Pfeils nicht immer m\"oglich. Dies bedingt die Notwendigkeit einer anderen Definition.
\newline
\indent Ein \textit{Vektor} ist eine geordnete Aufz\"ahlung von Objekten und wird auch \textit{Tupel} $(a_1, a_2,\dots,a_n)$ genannt. Entscheidend bei Tupeln ist die Reihenfolge der Objekte, die auch mehrfach vorkommen k\"onnen.

\subsection{Vektor- \& Unterraum}
Vektoren bilden die Elemente eines \textit{Vektorraumes V}. Addieren wir zwei Vektoren eines Vektorraumes oder multiplizieren wir sie mit einem Skalar, so ist die Summe bzw. das Produkt ebenfalls ein Element des Vektorraumes.
\vspace{10pt}
\begin{enumerate}
\item $u, v \in V \text{ }\Rightarrow u+v \in V$
\item $u\in V, \lambda \in \mathbb{R} \text{ }\Rightarrow \lambda u \in V$
\end{enumerate}
\vspace{10pt}
Ein \textit{Unterraum} $U$ ist eine Teilmenge eines Vektorraumes $V$. Es gelten f\"ur sie die oben genannten Eigenschaften eines Vektorraumes.
\vspace{10pt}
\begin{enumerate}
\item $\text{ }u, v \in U,\text{ } u+v \in U$
\item $\text{ }u \in U, \lambda \in \mathbb{R} ,\text{ }\lambda u \in U$
\end{enumerate}

\subsection{Span}
Der span$(v_1, v_2,\dots,v_k)$ ist die Menge aller Vektoren, die mit den Linearkombinationen der Basisvektoren darstellbar sind.
\begin{align*}
\text{span}(v_1,\dots,v_k) = \{u \in V : u = \lambda_1v_1+\dots+\lambda_kv_k\} \\
\text{ mit } \lambda_1,\dots,\lambda_k \in \mathbb{R}
\end{align*}

\subsection{Basis \& Dimension}
Die Menge der Basisvektoren wird \textit{Basis} eines Vektorraumes genannt. Jeder Vektorraum besitzt eine \textit{Dimension p}, die durch die Anzahl der Basisvektoren bestimmt wird.

\subsubsection*{\"Ubung}
Beweise, dass ein Span immer ein Vektorraum ist.
\begin{proof}
Wir nehmen an: 
\begin{align*}
u &\in \text{span}(v_1,\dots,v_k)\\
v &\in \text{span}(v_1,\dots,v_k)\\
u &= \lambda_1v_1+\dots+\lambda_k v_k\\
v &= \alpha_1v_1+\dots+\alpha_k v_k
\end{align*}
Additionsregelung bei Vektorr\"aumen
\begin{align*}
	w_1 &= u+v\\
	&=\lambda_1 v_1+\dots+\lambda_k v_k+\alpha_1 v_1+\dots+\alpha_k v_k \\
	&=(\lambda_1+\alpha_1) v_1+\dots+(\lambda_k+\alpha_k) v_k
\end{align*}
Multiplikationsregelung bei Vektorr\"aumen 
\begin{align*}
	w_2 &= \lambda_1 v_1+\dots+\lambda_k v_k\\
	\lambda w_2 &= (\lambda \cdot \lambda_1) v_1+\dots+(\lambda \cdot \lambda_k)v_k
\end{align*}
\end{proof}

\subsection{Lineare Unabh\"angigkeit}
Die Vektoren $v_1,\dots,v_k$ mit $v_i \in V$ sind \textit{linear unabh\"angig}, falls $\lambda_iv_i+\dots+\lambda_kv_k = 0$ nur f\"ur $\lambda_i=\dots=\lambda_k= 0$ gilt. 

%\end{multicols*}
%\end{document}

%\documentclass{article}

%\usepackage{tikz}
%\usepackage[utf8]{inputenc}
%\usepackage[ngerman]{babel}
%\usepackage{color}
%\usepackage[a4paper,lmargin={2cm},rmargin={2cm},
%tmargin={2.5cm},bmargin = {2.5cm}]{geometry}
%\usepackage{amssymb}
%\usepackage{amsmath}
%\usepackage{graphicx}
%\usepackage{multicol}
%\usepackage{amsthm}

%\newtheorem*{defi*}{Definition}
%\newtheorem*{defi2*}{Definition}

%\begin{document}
%\begin{multicols*}{2}

% \newcommand{\norm}[1]{\left\lVert#1\right\rVert}

\subsection{Norm und Skalarprodukt}
\author{Vivien Thi, Max Braun, Luca Bohn}
\begin{Def}
$\norm{v}$ ist eine Norm im Vektorraum $V$ falls
\begin{enumerate}
\item 
\begin{enumerate}
\item $\norm{v} \geq 0$ für alle $v \in V$
\item $\norm{v} = 0$ nur für $v = 0$
\end{enumerate}
\item $\norm{u+v} \leq \norm{u} + \norm{v}$ für alle $v \in V$
\item $\norm{\lambda v} = |\lambda| \norm{v}$ für alle $\lambda \in \mathbb{R}$ und $v \in V$
\end{enumerate}
\end{Def}
In Worte gefasst ergeben sich die folgenden Eigenschaften: Eine Norm kann nie negativ sein, ist die Norm 0 bedeutet das, dass der Vektor der Nullvektor ist. Wenn man zwei Vektoren
addiert, ist die Norm des resultierenden Vektors kleiner oder gleich der Summe der Normen der beiden Ausgangsvektoren. Außerdem ist die
Norm eines Vektors, der mit einem reellen Faktor multipliziert wurde gleich dem Produkt aus der Norm des Vektors und dem 
Betrag des Faktors.

\begin{Def}
$f:V\times V \rightarrow\mathbb{R}$ ist ein Skalarprodukt, wenn\\
\begin{enumerate}
\item $\left\langle  \alpha u + \beta v, w\right\rangle = \alpha \left\langle  u,w\right\rangle + \beta \left\langle  v,w\right\rangle $ (Bilinearität)
\item $\left\langle  u,v\right\rangle = \left\langle  v,u\right\rangle $ für alle $u,v \in V$
\item $\left\langle  v,v\right\rangle \geq 0$ mit $\left\langle  v,v\right\rangle = 0$ nur für $v = 0$
\end{enumerate}
\end{Def}
Wir können ein Skalarprodukt im euklidischen Raum $\mathbb{R}$ mit
\begin{equation}
\left\langle u,v\right\rangle  = u_1 v_1 + \dots + u_n v_n
\label{skal}
\end{equation}
definieren. An der ausgeschriebenen Form lässt sich überprüfen, ob es sich dabei wirklich um ein Skalarprodukt handelt. Hierbei kann  $\left\langle \alpha u + \beta v, w\right\rangle $ zu $\sum_{i=1}^n (\alpha u_i + \beta v_i)w_i$ umgeformt werden. Durch das Ausmultiplizieren der Produkte erhalten wir die Form $\sum_{i=1}^n \alpha u_i w_i + \beta v_i w_i$. Wenn wir daraus nach zwei Summen sortieren und die Summanden mit $\alpha$ von denen mit $\beta$ trennen, können wir die beiden Faktoren ausklammern und gelangen zu der aus der Definition geforderten Form $\alpha \left\langle u,w\right\rangle  + \beta \left\langle v,w\right\rangle $.
Die zweite Bedingung können wir mit dem Kommutativgesetz beweisen, wenn wir dieses auf die ausgeschriebene Form anwenden.
Für die dritte Bedingung bekommen wir $\sum_{i=1}^n v_i v_i = \sum_{i=1}^n v_i^2$. Da durch das Quadrieren keiner der Summanden kleiner als 0 werden kann und auch nur für $v_i = 0$ genau 0 werden kann, ist auch diese Bedingung erfüllt.

\begin{dsafigure}
\centering
\begin{tikzpicture}
\coordinate (A) at (0,0);
\coordinate (B) at (4,1.5);
\coordinate (C) at (1,3);
\draw[thick, ->] (A) -- (B) node[pos=0.5,sloped,below]{$v$};
\draw[thick, ->] (A) -- (C) node[pos=0.5,sloped, above]{$u$};
\draw[dashed, ->] (B) -- (C) node[pos=0.5,sloped, above]{$v-u$};
\draw (1,0.375) arc (21:75:1);
\node[] at (45:0.7) {$\theta$};
\end{tikzpicture}
\caption {Anwendung des Kosinussatzes}
\label{Kosinussatz}
\end{dsafigure}

Da grundsätzlich $\langle v,v\rangle  = \norm{v}^2$ gilt, kann über die Abbildung \ref{Kosinussatz} auch die Aussage getroffen werden, dass $\norm{v-u}^2 = \left\langle v-u,v-u\right\rangle $. Mit der Bilinearität des Skalarproduktes und der eben getroffenen Aussage können wir die Gleichung zu 
$\lVert v-u\rVert ^2 = \lVert v\rVert ^2 - 2\left\langle v,u\right\rangle  + \lVert u\rVert ^2$
umformen.
Zusätzlich folgt aus dem Kosinussatz, dass $\norm{v-u}^2 = \norm{u}^2 + \norm{v}^2 - 2 \norm{u} \norm{v} \cos\theta$. Setzt man beide Terme gleich, gelangt man über einige wenige Umformungen zu der Form $\left\langle u,v\right\rangle  = \norm{u} \norm{v} \cos\theta$, was ebenfalls eine Möglichkeit ist, das Skalarprodukt darzustellen.
Aus dieser Gleichung ergibt sich direkt die Cauchy-Schwarz Ungleichung. Da der Betrag des Kosinus sich nur zwischen 0 und 1 bewegt gilt
\begin{equation}
\left\langle u,v\right\rangle  \leq \norm{u} \norm{v}.
\label{cauchy}
\end{equation}

Eine häufig genutzte Norm ist die Euklidische Norm, welche durch $\norm{v}_2 = \sqrt{v_1^2 + \dots + v_n^2}$ definiert ist und die obrige Definition des Skalarproduktes \eqref{skal} induziert wird.

Die erste und die dritte Bedingung aus der Definition für eine Norm zeigen sich aus der ausgeschriebenen Form der Euklidischen Norm. Da alle Komponenten dabei quadriert werden und die Wurzel gezogen wird, muss die erste Bedingung stimmen. Ein Faktor $\lambda$, der ebenfalls quadriert in der Wurzel steht kann aus der Summe ausgeklammert werden und als Faktor vor die Wurzel gelangen. Der Betrag ergibt sich dabei daraus, dass $\lambda$ nach dem Vorgang nur positiv sein kann, auch wenn es zuvor negativ war.

Um zu beweisen, dass die Dreiecksungleichung $\norm{u+v} \leq\norm{u} + \norm{v}$  gilt,benötigen wir die Dreiecksungleichung für reelle Zahlen und die Cauchy-Schwarz Ungleichung. Wir gehen dabei von der quadrierten Form des linken Terms aus und erhalten
\begin{align*}
\norm{u+v}^2 &= |\left\langle u+v,u+v\right\rangle|\\
&\leq |\left\langle u,u\right\rangle| + 2|\left\langle u,v\right\rangle| + |\left\langle v,v\right\rangle|\\
&\leq \norm{v}^2 + 2\norm{u} \norm{v} + \norm{v}^2\\
&= (\norm{u} + \norm{v})^2.
\end{align*}

Man sieht sofort, dass der Ausgangsterm kleiner oder gleich dem Ergebnis ist, womit bewiesen ist, dass die Euklidische Norm eine Norm ist.

%\end{multicols*}
%\end{document}

\subsection{Lineare Abbildungen}
In diesem Abschnitt werden wir die elementarste Form einer Funktion kennenlernen, die einen Vektorraum auf einen anderen Vektorraum abbildet. Diese Funktionen nennt man \textit{lineare Abbildungen}.
\begin{Def}
	\label{Def:Def_1}
Eine Funktion $f:U \stackrel{}{\rightarrow} V$ ist linear, wenn folgende Bedingungen gelten: \\ 
1. $f(u+v) = f(u)+f(v) \ \forall \ u,v \in U$\\
2. $f(\lambda v) = \lambda f(v) \ \forall \ v \in U \ \text{und} \ \lambda \in \mathbb{R}$
\end{Def}

Das besondere an solchen Abbildungen ist, dass sie von \textit{Matrizen} eindeutig definiert werden. Wie Matrizen definiert sind und wie wir sie verwenden können, lernen wir im nächsten Kapitel. Im Folgenden gehen wir davon aus, dass $U$ die Dimension $n$ hat und $V$ die Dimension $m$. 

\subsubsection{Matrix}
Eine Matrix ist eine spezielle Anordnung von $n\cdot m$ Zahlen in $n$ Zeilen und $m$ Spalten. Man kann eine Matrix aber auch als eine Zusammenfassung von $m$ Vektoren mit $n$ Komponenten ansehen, wobei ein Vektor eine Spalte bildet. Die Komponente in der $i$-ten Zeile und $j$-ten Spalte wird als $a_{ij}$ geschrieben und die ganze Spalte mit $a_j$ notiert.\\
\begin{equation*}
A= \left(
   \begin{array}{cccc}
	  a_{11} & a_{12} & ... & a_{1m}\\
		a_{21} & a_{22} & ... & a_{2m}\\
		...   & ...   & ... & ...  \\
		a_{n1} & a_{n2} & ... & a_{nm}
	 \end{array}
	 \right) = (a_1, a_2, ..., a_m)\\
\end{equation*}
\\
Da wir nun wissen was eine Matrix ist, führen wir in dem folgenden Abschnitt zwei neue Operationen ein: die \textit{Matrix-Vektor-Multiplikation} und \textit{Matrix-Matrix-Multiplikation}.

\subsubsection{Matrix-Vektor-Multiplikation}
Bei einer Matrix-Vektor-Multiplikation wird eine Matrix $A \in  \mathbb{R}^{n \times m}$ mit einem Vektor $v \in \mathbb{R}^m$ multipliziert. Dabei bildet jede Multiplikation zwischen einer Matrixzeile und einem Vektor ein Skalarprodukt. Eine Multiplikation zwischen Matrizen und Vektoren, bei denen die Anzahl nicht übereinstimmt, ist nicht definiert. Deshalb ist es wichtig, dass die Matrix so viele Spalten wie der Vektor Dimensionen hat.
\begin{align*}
	Av &= 
	\left(
   \begin{array}{cccc}
	  a_{11} & a_{12} & ... & a_{1m}\\
		a_{21} & a_{22} & ... & a_{2m}\\
		...   & ...   & ... & ...  \\
		a_{n1} & a_{n2} & ... & a_{nm}
	 \end{array}
	\right) 
	\cdot 
	\left(
	 \begin{array}{c}
	  v_1\\
		v_2\\
		...\\
		v_m
	 \end{array}
	\right)\\
	&= \left(
	 \begin{array}{c}
	 a_{11} \cdot v_1 + a_{12} \cdot v_2 + ... + a_{1m} \cdot v_m\\
	 a_{21} \cdot v_1 + a_{22} \cdot v_2 + ... + a_{2m} \cdot v_m\\
		...\\
	 a_{n1} \cdot v_1 + a_{n2} \cdot v_2 + ... + a_{nm} \cdot v_m\\
	 \end{array}
	\right)
\end{align*}

Allgemein lässt sich die Matrix-Vektor-Multiplikation folgendermaßen berechnen

\begin{equation*}
(Av)_i = \sum_{j=1}^{n}a_{ij}v_j ,
\end{equation*}

wobei $(AV)_i$ der $i$-te Eintrag des Vektors $n$ ist, der das Ergebnis der Multiplikation ist.\\
Eine Eigenschaft der linearen Abbildung ist, dass sich jede Abbildung als ein solches Matrix-Vektor-Produkt schreiben lässt. 

\subsubsection{Matrix-Matrix-Multiplikation}
Eine Matrix-Matrix-Multiplikation ist nichts anderes als mehrere Matrix-Vektor-Multiplikationen hintereinander. Haben wir zwei Matrizen $A \in \mathbb{R}^{n \times m}$ und $B \in \mathbb{R}^{m \times p}$ und berechnen $AB$, nehmen wir jede Spalte von $B$ als Vektor und multiplizieren ihn mit $A$ wie oben beschrieben. Die Vektoren, die man als Ergebnisse erhält, fasst man wieder in einer Matrix mit der Dimension $\mathbb{R}^{n \times p}$ zusammen. Allgemein können wir die Matrix-Matrix-Multiplikation folgendermaßen berechnen
\begin{equation*}
	(AB)_{il} = \sum_{n=i}^{m}a_{ij} \cdot b_{jl}
\end{equation*}

\subsubsection{Komposition}
\begin{Thm}
Eine Komposition $h(x) = (f \circ g)(x)$ aus den linearen Abbildungen $f$ und $g$, ist auch eine lineare Abbildung.
\end{Thm}

\begin{proof}
Da $f$ und $g$ linear sind gelten die Bedingungen aus Definition \ref{Def:Def_1}:
\begin{align*}
	h(u+v) &= (f \circ g)(u+v)\\
	&= f(g(u+v))\\
	&= f(g(u)+g(v))\\
	&= f(g(u)) + f(g(v))\\
	&= (f\circ g)(u) + (f\circ g)(v) = h(u) + h(v)
\end{align*}
Des Weiteren gilt:
\begin{align*}
	h(\lambda v) &= (f \circ g)(\lambda v)\\
	&=f(g(\lambda v))\\
	&= f(\lambda g(v))\\
	&=\lambda f(g(v))\\
	&=\lambda (f \circ g)(v) = \lambda h(v)
\end{align*}
Da beide Bedingungen aus Definition \ref{Def:Def_1} für $h$ erfüllt sind, ist auch eine Komposition aus zwei linearen Abbildungen linear.
\end{proof}


\section{Analysis}
\authors{Franziska Wilfinger, Jan Fritz, Florentine Klepel}  

Analysis ist ein wichtiges Teilgebiet der Mathematik. Sie ist Grundlage für Optimierung und somit auch von großer Bedeutung für das Maschinelle Lernen. Im Kurs war für uns besonders die Stetigkeit und Differenzierbarkeit von ein- und mehrdimensionalen Funktionen wichtig.

\subsection{Stetigkeit und Differenzierbarkeit}

Wir beginnen mit einigen theoretischen Grundlagen der Analysis.
\begin{Def} Die Zahl $L$ heißt \emph{Grenzwert} der Funktion $f(x)$ für $x$ gegen $a$, wenn es für alle $\epsilon > 0$ ein $\delta > 0$ existiert, sodass für alle $|x-a| < \delta$ gilt $|f(x)-L|<\epsilon$
\end{Def}
\begin{Def}
Eine Funktion $f: D \subseteq \mathbb{R}\rightarrow \mathbb{R}$ ist \emph{stetig}, wenn $\lim_{w\rightarrow x}f(w)=f(x)$ auf ihrem gesamten Definitionsbereich gilt, das heißt sie ist stetig in $x$ für alle $x\in D$.
\end{Def}
\begin{Def}
Eine Funktion $f:\mathbb{R}\rightarrow \mathbb{R}$ ist an einer Stelle $x$ \emph{differenzierbar}, wenn eine lineare Abbildung $ l_ x (h)$ und ein Restglied $r(h)$ existieren, sodass $f(x+h)=f(x)+l_{x}(h)+r(h)$ wobei $\lim_{h\rightarrow 0} r(h)/h= 0$ gilt.
\end{Def}
Wir betrachten eine einfache Folgerung aus diesen Definitionen.


\begin{lemma}
Eine Funktion $f:\mathbb{R} \rightarrow \mathbb{R}$, welche differenzierbar an der Stelle $x \in \mathbb{R}$ ist, ist dort auch stetig.
\end{lemma}

\begin{proof}
Wir überlegen zuerst, welche Bedingungen gegeben sind. Eine Funktion ist an der Stelle $x$ differenzierbar, wenn $f(x+h)=f(x)+l_x(h)+ r(h)$  mit $\lim_{h \rightarrow 0} r(h)/h= 0$. Um die Stetigkeit nachzuweisen müssen wir zeigen, dass $\lim_{w \rightarrow x} f(w)=f(x)$. Mit $w = x+h$ ist dies äquivalent zu $\lim_{h \rightarrow 0} f(x)+l_x(h)+r(h)=f(x)$. Es genügt also zu zeigen, dass  $l_x(h)$ und $r(h)$ gegen Null gehen.
\begin{itemize}
  \item Würde $r(h)$ nicht gegen Null gehen, so würde $r(h)/h$ nicht gegen Null gehen. Widerspruch zur Differenzierbarkeit.
  \item Der lineare Term $l_x(h)$ kann auch als $f'(x) \cdot h$ dargestellt werden. Da $h$ gegen Null geht folgt $l_x(h)\rightarrow 0$.
\end{itemize}
Insgesamt folgt $\lim_{w\rightarrow x} f(w) = f(x)$.
\end{proof}

\subsection{Gradient}
\begin{Def}
Der \emph{Gradient} $\nabla f(x)$ einer Funktion $f:\mathbb{R}^n\rightarrow \mathbb{R}$ ist ein Vektor, wobei jeder Eintrag die partielle Ableitung der entsprechenden Komponente ist, also
\begin{equation*} \begin{split} \nabla f(x) = \left( \begin{array}{c}
\partial_{x_1} f(x) \\
\partial_{x_2} f(x) \\
... \\
\partial_{x_n} f(x)
\end{array}
\right).
\end{split} \end{equation*} 
\end{Def}

Der Gradient zeigt immer in Richtung des steilsten Anstiegs einer Funktion. Der negative Gradient in die Richtung des steilsten Abstiegs, weshalb wir das Gradientenabstiegsverfahren verwenden können, um das Minimum einer mehrdimensionalen Funktion zu finden. Wichtig für das Gradientenabstiegsverfahren ist das folgende Lemma

\begin{Lem}
\label{lem:gradientabstieg}
F\"ur eine differenzierbare Funktion $f$ ist es immer moeglich einen Skalar $\lambda>0$ zu finden, sodass
\begin{equation}
\label{lambda-ungleichung}
f(x - \lambda \nabla f(x)) \leq f(x)
\end{equation} 
\end{Lem}


\begin{proof}
Wir betrachten den nicht-trivialen Fall in dem $\nabla f(x) \neq 0$. 
Die zu zeigende Ungleichung \eqref{lambda-ungleichung} laesst sich umschreiben in die folgende Form
\begin{align*}
&f(x - \lambda \nabla f(x))\\
 = \, &f(x) - \lambda \langle \nabla f(x), \nabla f(x)\rangle + r(\lambda \nabla f(x)) \\
= \, &f(x) - \lambda( \Vert\nabla f(x)\Vert^2 - \frac{r(h)}{\lambda}) \overset{!}{\leq} f(x)
\end{align*}
Es ist hinreichend zu zeigen, dass ein $\lambda>0$ existiert, sodass
\begin{equation}
\label{eq:positiveterm}
\Vert\nabla f(x)\Vert^2 - \frac{r(\lambda \nabla f(x))}{\lambda} >0.
\end{equation}
  Dies ist dadurch gegeben, dass 
\begin{align*}
\lim_{\lambda\to 0} \frac{r(\lambda \nabla f(x))}{\lambda} &= \lim_{\lambda \to 0}\Vert\nabla f(x)\Vert \frac{r(\lambda \nabla f(x))}{\lambda \Vert\nabla f(x)\Vert}= 0
\end{align*}
Die letzte Gleichung folgt durch die Bedingung an das Restglied, dass $\lim_{h\to 0}\frac{r(h)}{h} = 0$. Durch die Definition eines Grenzwerts wissen wir also, dass ein $\lambda^*$ existiert, sodass $\left| \frac{r(\lambda^* \nabla f(x))}{\lambda^* \Vert\nabla f(x)\Vert} \right| \leq \Vert\nabla f(x)\Vert$. Zusammenfassend ist die Positivitaet des Terms \eqref{eq:positiveterm} damit gezeigt, d.h. es gilt fuer das gerade gewaehlte $\lambda^*$, dass
\begin{align*}
&\Vert\nabla f(x)\Vert^2 - \frac{r(\lambda^* \nabla f(x))}{\lambda^*} \\
\geq \, &\Vert\nabla f(x)\Vert \left( \Vert \nabla f(x)\Vert - \frac{r (\lambda^* \nabla f(x))}{\lambda^* \Vert \nabla f(x)\Vert} \right) \\
\geq \, &\Vert \nabla f(x)\Vert \left(\Vert\nabla f(x)\Vert - \left| \frac{r (\lambda^* \nabla f(x))}{\lambda^* \Vert \nabla f(x)\Vert}\right| \right) \geq 0. 
\end{align*}
\end{proof}




%% Das Ganze kann man noch weiter umschreiben, indem man das Skalarprodukt bildet: \\ \\
%% $f(x) + \lambda \langle \nabla f(x), \nabla f(x) \rangle + r(h) \leq f(x)$ \\ \\
%% Anschließend kann man von der Regel Gebrauch machen, dass sich ein Skalarprodukt, das in der Form $\langle u,u \rangle$ vorliegt, in die Ausprägung der Euklidischen Norm von $||u||^2$ umgewandelt werden kann. \\
%% Daraus folgt: \\ \\
%% $f(x) + \lambda \cdot || \nabla f(x)||^2 + r(h) \leq f(x)$ \\ \\
%% Um nun die obige Ungleichung (1) zu beweisen, muss man zeigen, dass \\ $\lambda ||\nabla f(x)||^2 + r(h) \leq 0$ ist.\\
%% Dafür wird $\lambda$ ausgeklammert. Anschließend erhält man:

%% \begin{equation} \lambda (||\nabla f(x)||^2 + \frac{r(h)}{\lambda} \end{equation}

%% Da $||\nabla f(x)||$ quadriert wird, wird dieser Term automatisch positiv. Da man den gesamten Term aber negativ haben will, wählt man $\lambda < 0.$ Schließlich kennt jeder die Regel, dass ein Produkt negativ wird, falls eine ungerade Zahl von Faktoren negativ ist. Da wir $\lambda$ negativ gewählt haben, muss \\ $||\nabla f(x)||^2$$(2) \lambda (||\nabla f(x)||^2 + \frac{r(h)}{\lambda}$ positiv sein.\\
%% Bei $||\nabla f(x)||^2||$ stellt dies aus den eben genannten Gr\"unden kein Problem dar. Deswegen muss nur noch gezeigt werden, dass es mindestens einen Wert von $r(h)$ gibt, der gr\"oßer oder gleich 0 ist. Ansonsten würde der Wert in der Klammer negativ werden k\"onnen und damit die unspr\"ungliche Aussage (1) widerlegen.\\
%% Damit wir zeigen k\"onnen, dass $r(h) \geq 0$ ist, formten wir zuerst den Term (2) um:\\

%% $\lambda (||\nabla f(x)||^2 + \frac{r(h)}{\lambda})$ \\

%% $\Leftrightarrow \lambda(||\nabla f(x)||^2 + \frac{r(\lambda \nabla f(x))}{\lambda})$ \\

%% Bei dieser Umformung wurde h wieder mit $\lambda\nabla f(x)$ ersetzt.

%% $\leftrightarrow \lambda ||\nabla f(x)|| \cdot (||\nabla f(x)|| + \frac{r(\lambda \nabla f(x))}{\lambda || f(x)||})$


%% Durch diese Umformungen schlussfolgerten wir, dass f\"ur die Funktion $h(x)$ folgendes gelten muss:\\
%% \begin{equation} h(\lambda)=||\nabla f(x)|| + \frac{r(\lambda\nabla f(x))}{\lambda||\nabla f(x)||} \geq 0 \end{equation}


%% Da in $||\nabla f(x)$ kein $\lambda$ steht ist dieser Term konstant. Gleichzeitig wird durch das Bilden der euklidischen Norm festgelegt, dass ein Wert größe/gleich null vorliegt. \\ %der Satz ist noch nicht wiklich schön formuliert
%% Deswegen muss nur noch gezeigt werden, dass dder zweite Summand ebenfalls positiv werden kann. Dazu bildeten wir den Grenzwert. \\ %Die Zeitformen stimmen sehr oft nicht
%% $\lim\limits_{h \rightarrow 0}{\frac{r(\lambda \nabla f(x))}{\lambda ||\nabla f(x)||}} = \frac{r(\lambda \nabla f(x))}{||\lambda  \nabla f(x)||}$ \\ %hier kommt noch kein h vor. Umschreiben?

%% Man kann $\lambda \nabla f(x)$ wieder durch h ersetzen, woraus folgt: \\
%% $\lim\limits_{h \rightarrow 0} \frac{r(h)}{h}$ \\

%% Der Grenzwert des Restbetrages strebt gegen Null. Daraus lässt sich schließen, dass es einen Punkt gibt, ab welchem die Ungleichung der Funktion $h(\lambda)$ (3) erfüllt ist und damit auch die anfängliche Aussage (1). \\ %ist hier h(\lambda) und (3) beies nötig?

\subsection{Summen-, Produkt- und Kettenregel}

Falls mehrere Funktonen miteinander verknüpft sind gelten folgende drei Regeln.
\begin{itemize}
\item Die \emph{Summenregel} besagt, dass der Gradient der Funktion $h(x)=f(x)+g(x)$ die Summe der Gradienten ist, das heißt
\begin{equation*} D(f + g)(x)=\text{D} f(x)+\text{D} g(x)\end{equation*}

\item Gegeben seien die differenzierbaren Funktionen $f:\mathbb{R}^k\rightarrow\mathbb{R}^n$, $g:\mathbb{R}^n\rightarrow\mathbb{R}^k$ und $h(x)=f(g(x))$. Die \emph{Kettenregel} besagt, dass
\begin{equation*}
\text{D}h(x) = \text{D}f(g(x)) \cdot \text{D} g(x).
\end{equation*}

\item Gegeben zwei differenzierbaren Funktionen $f:\mathbb{R}^m\rightarrow\mathbb{R}$ und $g:\mathbb{R}^m\rightarrow\mathbb{R}$, so gilt für das Produkt der Funktionen $h(x)=f(x) \cdot g(x)$ die \emph{Produktregel} \begin{equation*}\text{D}h(x)=f(x) \cdot \text{D}g(x)+\text{D}f(x) \cdot g(x).\end{equation*}
 \end{itemize}
Wir beweisen nun die Produktregel.
Da die Funktionen $f$ und $g$ differenzierbar sind gilt
\begin{align*}
  f(x+h) &= f(x) +  Df(x) h + r_{f}(h)\\
  g(x+h) &= f(x) + Dg(x) h + r_{g}(h)
\end{align*}
wobei $\lim_{h\rightarrow 0} r_f(h)/h = 0$ und $\lim_{h\rightarrow 0} r_g(h)/h = 0$.
Wir schreiben zuerst die Ableitung von $h(x+v)$ als Kombination aus den Ableitungen von $f(x)$ und $g(x)$ auf, also $h(x+v) =  f(x+v)\cdot g(x+v)$.
Das Produkt wird ausmultipliziert und wir erhalten
\begin{equation*}
\begin{split} h(x+v) & = f(x) \cdot g(x)  + f(x)  \cdot D_{g}(x) \cdot v \\ & + f(x) \cdot r_{g}(v) + Df(x) \cdot v \cdot g(x) \\ & +  Df(x) \cdot v \cdot D_{g}(x) \cdot v  + Df(x) \cdot v \cdot r_{g}(v) \\ & +  r_{g}(v) \cdot g(x) + r_{g}(x) \cdot Dg(x) \cdot v \\ & + r_{g}(v) \cdot r_{g}(v) \end{split} \end{equation*}   %Die Formel passt nicht in eine Zeile. Umbruch?

Von den ganzen Summanden müsssen letztendlich alle eliminiert werden, bis auf 
\begin{equation*} f(x) \cdot g(x) + f(x) \cdot Dg(x) \cdot v + Df(x) \cdot v \cdot g(x) \end{equation*} 
Dafür müssen wir zeigen, dass die übrigen Summanden gegen Null gehen. Um das zu erreichen bilden wir den Grenzwert
\begin{align*}
\lim \limits_{v \rightarrow 0} &||v||^{-1}(||f(x) \cdot r_{g}v + Df(x) \cdot v \cdot Dg(x) \cdot v +{}\\
&\qquad + Df(x) \cdot v \cdot r_{g}(v) r_{f}(v) \cdot g(x) +{}\\
&\qquad + r_{g}(v) \cdot Dg(x) \cdot v + r_{g}(v) \cdot r_{f}(v)||)
\end{align*}
Damit die Summanden einzeln behandelt werden können, wenden wir die Dreiecksungleichung an. Dann können wir von jedem Summanden einzeln den Limes bilden, um zu zeigen, dass der gesamte Grenzwert gegen Null geht. Dies ist aber nur der Fall, falls der Limes jedes Summanden gegen Null geht. Wir zeigen das exemplarisch für 

\begin{equation*}
\lim\limits_{v \rightarrow 0} \frac{||Df(x) \cdot r_{g}(v)||}{||v||}
\end{equation*}

$Df(x)$ ist eine Matrix mit einer Zeile und m Spalten in der jeweils ein Eintrag $\leq c$ steht. Daraus folgern wir:
\begin{align*}
&\lim\limits_{v \rightarrow 0} \frac{m \cdot c \cdot ||v|| \cdot ||r_{g}(v)||}{||v||}\\
&\qquad= m \cdot c \cdot \lim\limits_{v \rightarrow 0} \frac{ ||v|| \cdot ||r_{g}(v)||}{||v||} = 0
\end{align*} 
Damit haben wir gezeigt, dass der Summand gegen Null geht. Ähnlich behandelt man auch die anderen Summanden und auch diese gehen alle gegen Null.
Die Produktregel ist damit bewiesen. 















%Der erste Teil der Optimierung ist noch nicht bei GitHub ...
%Bitte noch in den Programmkopf einfügen, damit die Geogebradarstellungen funktionieren: 
%\usetikzlibrary{arrows}
%\usepackage{pgf,tikz}
%\usepackage{mathrsfs}
\section{Optimierung}
\subsection{Grundlagen der Optimierung}
In der Optimierung studiert man wie Funktionen $f: \mathbb{R}^n \rightarrow \mathbb{R}$, welche Elemente eines $n$-dimensionalen Vektorraumes auf reelle Zahlen abbilden, maximiert oder minimiert werden können.

Beispielhafte Anwendungszwecke wären die Minimierung oder Maximierung der Kosten bzw. der Einnahmen eines Unternehmens. Diese würden dem reellen Funktionswert der Funktion entsprechen und ergäben sich aus den Eigenschaften des Unternehmens wie Gehalt, Anzahl der Mitarbeiter, Marketing-Ausgaben usw., welche man zusammenfassend als Vektor eines $n$-dimensionalen Vektorraumes darstellen könnte.

\subsection{Regression}

\begin{Thm}[Methode der kleinsten Quadrate]
Beispiel für die Lösung eines solchen Minimierungsproblems ist die Methode der kleinsten Quadrate. Gegeben ist hierbei eine lineare Funktion $f(x) = ax$ und Punkte $P_i=(x_i,y_i)$, welche näherungsweise auf der Geraden liegen (siehe Abb. \ref{fig:LinReg}), also $y_i = f(x_i) + \epsilon_{i}$.

\definecolor{ffqqqq}{rgb}{1.,0.,0.}
\definecolor{qqqqff}{rgb}{0.,0.,1.}
\begin{dsafigure}
\begin{tikzpicture}[line cap=round,line join=round,>=triangle 45,x=1.0cm,y=1.0cm]
\draw[->,color=black] (-0.252729391848446,0.) -- (5.909881514665974,0.);
\foreach \x in {,0.5,1.,1.5,2.,2.5,3.,3.5,4.,4.5,5.,5.5}
\draw[shift={(\x,0)},color=black] (0pt,2pt) -- (0pt,-2pt);
\draw[->,color=black] (0.,-0.118522622381023) -- (0.,4.459842973966475);
\foreach \y in {,0.5,1.,1.5,2.,2.5,3.,3.5,4.}
\draw[shift={(0,\y)},color=black] (2pt,0pt) -- (-2pt,0pt);
\clip(-0.252729391848446,-0.118522622381023) rectangle (5.909881514665974,4.459842973966475);
\draw[smooth,samples=100,domain=-0.252729391848446:5.909881514665974] plot(\x,{(\x)});
\draw [color=ffqqqq] (1.,1.)-- (1.,1.5);
\draw [color=ffqqqq] (2.,2.)-- (2.,1.7);
\draw [color=ffqqqq] (3.,3.)-- (3.,3.6);
\draw [color=ffqqqq] (4.,4.)-- (4.,3.2);
\begin{scriptsize}
\draw [fill=qqqqff] (1.,1.5) circle (1.5pt);
\draw[color=qqqqff] (0.9863140708369929,1.7517884297439121) node {$P_1$};
\draw [fill=qqqqff] (2.,1.7) circle (1.5pt);
\draw[color=qqqqff] (1.9808094816766213,1.4790347346423591) node {$P_2$};
\draw [fill=qqqqff] (3.,3.6) circle (1.5pt);
\draw[color=qqqqff] (2.9753048925162497,3.8656295667809477) node {$P_3$};
\draw [fill=qqqqff] (4.,3.2) circle (1.5pt);
\draw[color=qqqqff] (3.95893150105162,2.9499564475114486) node {$P_4$};
\draw[color=ffqqqq] (1.0297892800540258,1.2647282599197103) node {$\varepsilon_1$};
\draw[color=ffqqqq] (2.230791934674561,1.9466124976735928) node {$\varepsilon_2$};
\draw[color=ffqqqq] (3.029648904037541,3.29089856638839) node {$\varepsilon_3$};
\draw[color=ffqqqq] (4.236085959810206,3.6902879056442353) node {$\varepsilon_4$};
\end{scriptsize}
\end{tikzpicture}
\caption{Beispiel einer Linearen Reggression}
\label{fig:LinReg}
\end{dsafigure}

Wenn nun der Abstand $\epsilon_i$ minimiert werden soll, dann gilt
\begin{equation*}
\min h(x)={\sum_{i=1}^n\epsilon_i} = {\sum_{i=1}^n(y_i-ax_i)^2)}
\end{equation*}
 Zur Berechnung dieses Minimums gibt es nun verschiedene Möglichkeiten:
\begin{enumerate}
\item Gewöhnliche Tiefpunktberechnung aus
$h'(a)=0$ und $h''(a)>0$
\item Gradientenabstiegsmethode mittels
$a^{t+1}  =a^{t} - \lambda \cdot \partial f(a^{t})$,
man findet also ein $\lambda > 0$, sodass
$f(a^{t+1}) < f(a^{t})$ falls $ \partial f(a^{(t)}) \neq 0$. Dies kann erreicht werden durch $\lambda=\arg\min(g(\lambda))$ wobei
$g(\lambda)=f(a^t-\lambda\nabla f(a^t))$.
\end{enumerate}

\end{Thm}

\begin{Def}[Lokales und globales Minimum]
Für $f: D\rightarrow \mathbb{R}$ ist $x\in D$ ein lokales Minimum, wenn mindestens eine Umgebung $N$ existiert, sodass $f(y)\geq f(x)$ für alle $y\in N$.
Ein globales Minimum liegt vor, falls hierbei $N=D$.
\end{Def}

\subsection{Konvexität einer Funktion}

\begin{Def}[Konvexe Funktion]
Eine Funktion $f: \mathbb{R}^n\rightarrow\mathbb{R}$ ist dann eine konvexe Funktion, falls für alle $x, y\in D$ gilt
\begin{equation*}
  f(\lambda x+ (1 - \lambda)y) \leq \lambda f(x)+(1-\lambda)f(y)
\end{equation*}
für alle $\lambda \in [0, 1]$, siehe Abb. \ref{fig:konvex}.

\begin{dsafigure}
\begin{center}
\includegraphics[width=0.3\textwidth]{\media Grafik-Optimierung_KonvexeFunktion.pdf}
\caption{Beispiel einer konvexen Funktion}
\label{fig:konvex}
\end{center}
\end{dsafigure}



\end{Def}

\subsubsection{Beispiel zur Konvexität}
Ein Beispiel für eine konvexe Funktion ist die euklidische Norm $f(u) = \lVert u\rVert_2$ für $u\in\mathbb{R}$. Diese ist konvex, weil
\begin{align*}
f(\lambda u+(1-\lambda)v) &=\lVert\lambda u+(1-\lambda)v\rVert_2\\
&\leq\lVert\lambda u\rVert_2+\lVert(1-\lambda)v\rVert_2\\
&\leq\lambda\lVert u\rVert_2+(1-\lambda)\lVert v\rVert_2
\end{align*}
wie aus der Dreiecksungleichung und der Linearität der Norm folgt.

\subsection{Hier fehlt eine Überschrift!}

\begin{algorithmic}[1]

\Procedure{$\mathbf{calculateMinimum}$}{}
   \\F(x) = (x$^2$-2)$^2$
\\F'(x) = 4x$^3$-8x

\\
\\x = 10.0
\\$\lambda$ = 0.001

 \For{i = 8}
 	\State $\lambda$ = $\lambda$+(i*0.001)
\For{j = 8}
\State x = x - $\lambda$ * F'(x)
\EndFor
 \EndFor
 
 \\Print x\EndProcedure
\Statex
\end{algorithmic}
\subsection{Konvexe Mengen}

\paragraph{Einführung}

Im Rahmen der Optimierung von konvexen Funktionen ist es erforderlich, den Begriff der konvexen Menge einzuführen. Um diese Thematik anschaulich darzustellen, verwenden wir zunächst verschiedene geometrische Figuren in Abbildung \ref{figure:Grafik-Optimierung_KonvexeMengen}, von denen einige konvex sowie andere wiederum nicht konvex sind. \\
Ist es eine direkte Verbindungsstrecke zwischen zwei beliebigen zu finden, die selbst ebenfalls in der Menge liegt, so ist die Menge konvex.

\begin{dsafigure}
\begin{center}
\includegraphics[width=0.4\textwidth]{\media Grafik-Optimierung_KonvexeMengen.pdf}
\label{figure:Grafik-Optimierung_KonvexeMengen}
\caption{Beispiele konvexer Mengen}
\end{center}
\end{dsafigure}

\begin{Def}[Konvexe Menge]

Eine Menge $X$ heißt konvex, falls für alle $x, y \in X$ und $\lambda \in \mathbb{R}$ und $\lambda \in [0,1]$ gilt:

\begin{equation*}
\lambda x + (1 - \lambda) \in X
\end{equation*}

\end{Def}

\subsection{Beispiele konvexer Mengen}

\paragraph{Kugel}

Ein Beispiel für eine konvexe Menge ist die Menge $S$ der Vektoren, die eine Kugel mit dem Radius $r = 1$, deren Mittelpunkt im Ursprung liegt, beschreibt:

\begin{equation*}
S = \{x \in \mathbb{R}^{n} | \lVert x \rVert \le 1\}
\end{equation*}

\paragraph{Quadrant}

Auch die Quadranten des kartesischen Koordinatensystems lassen sich durch eine konvexe Menge erfassen. So gilt z.~B. für den ersten Quadranten:

\begin{equation*}
L = \{ x \in \mathbb{R}^{n} | x_1 \ge 0, x_2 \ge 0, ..., x_n \ge 0\}
\end{equation*}

\paragraph{Eistüte}

Bei einer Eistüte bzw. einem quadratischen Kegel handelt es sich um ein weiteres Beispiel für eine konvexe Menge, da sie folgendermaßen für $x \in \mathbb{R}^{n-1}$ und $t \in \mathbb{R}$ beschrieben werden kann:

\begin{equation*}
Q = \{(x, t) \in \mathbb{R}^{n} | \lVert x\rVert ^{2} \le t\}
\end{equation*}

\paragraph{Box}

Die Menge von Vektoren in $B \subseteq \mathbb{R}$ bilden eine konvexe Menge in Form einer Box mit der Seitenlänge 2, deren Mittelpunkt im Ursprung liegt:

\begin{equation*}
B = \{x \in \mathbb{R}^{n} | |x_{i}| \le 1, i = 1, ..., n\}
\end{equation*}

\subsection{Konvexe Programme}

\paragraph{Einführung}

Bei einem konvexen Programm handelt es sich um die Optimierung einer Funktion $f$, die vom Parameter $x$ abhängig ist. Wir versuchen diese Funktion unter Berücksichtigung von Nebenbedingungen zu minimieren. Hierbei sind sowohl die Zielfunktion als auch die Menge der Punkte, die die Nebenbedingungen erfüllt, konvex.

\begin{Def}[Konvexes Programm]
Sei $f: \mathbb{R}^{n} \rightarrow \mathbb{R}$ eine konvexe Funktion und $X \subseteq \mathbb{R}^{n}$, so ist das konvexe Optimierungsproblem bzw. Programm die Minimierung der Funktion $f(x)$ unter der Nebenbedingung, sodass $x \in X$.
\end{Def}

\paragraph{Lineare Programme}

Lineare Programme lassen sich durch lineare Funktionen beschreiben, z.~B. $f(x) = \langle c, x \rangle$. Darüber hinaus sind eine lineare Abbildungsmatrix $A$, z.~B. $A = \begin{pmatrix}1 & 0 \\ 0 & 1 \end{pmatrix}$, sowie ein Vektor $b$ und ein weiterer Vektor $c$, z.~B. $c = \begin{pmatrix}1 \\ 0 \end{pmatrix}$, gegeben. $x$ ist dann zulässig, wenn $x \in X$ mit $X: Ax \le b$ d.~h. $X = \{x \in \mathbb{R}^{n} | Ax \le b\}$ erfüllt ist. \\
Ein Beispiel für ein lineares Programm wäre:

\begin{equation*}
f(x) = \langle \begin{pmatrix}1 \\ 0 \end{pmatrix}, \begin{pmatrix}x_{1} \\ x_{2} \end{pmatrix} \rangle = x_{1}
\end{equation*}

Für die Nebenbedingungen gilt:

\begin{equation*}
\begin{pmatrix}1 & 0 \\ 0 & 1 \end{pmatrix}
\end{equation*}

Für den Wertebereich des linearen Programms folgt: $x_{1} \le 1$, $x_{2} \le 1$.

\paragraph{Quadratische Programme}

Im Vergleich zu einem linearen Programm weist z.~B. eine Kostenfunktion im quadratischen Programm eine quadratische Form auf. \\
Beispielsweise sei ein quadratisches Programm $f(x) = x^{T}Qx$ mit $Q = \begin{pmatrix}1 & 0 \\ 0 & 1 \end{pmatrix}$. $f$ gilt es nun unter der Bedingung $x \in X$ für $X: Ax \le b$ zu minmieren:

\begin{align*}
f(x) &= x^{T}Qx \\
&= \begin{pmatrix}x_{1}, x_{2}\end{pmatrix} \begin{pmatrix}1 & 0 \\ 0 & 1\end{pmatrix} \begin{pmatrix}x_{1} \\ x_{2}\end{pmatrix} \\
&= \begin{pmatrix}x_{1}, x_{2}\end{pmatrix} \begin{pmatrix}x_{1} \\ x_{2}\end{pmatrix} \\
&= x_{1}^{2} + x_{2}^{2}
\end{align*}

\section{Wahrscheinlichkeitstheorie}
<<<<<<< HEAD


\vspace{15pt}


1933 ver\"offentlichte der russische Mathematiker Andrey Kolmogorov sein Buch \textit{Foundations of the Theory of Probability}, in dem er die drei Axiome der Wahrscheinlichkeitstheorie aufstellte. Es existiert ein Wahrscheinlichkeitsraum, der sich aus den drei Elementen $\Omega$, $\mathcal{F}$, $\mathbb{P}$ zusammensetzt. Wir betrachten im Folgenden die speziellen Wahrscheinlichkeitsr\"aume mit folgenden Eigenschaften:
=======
\authors{Katharina Krause, Annika Scheug, Moritz Hollenberg}
1933 ver\"offentlichte der russische Mathematiker Andrey Kolmogorov sein Buch \textit{Foundations of the Theory of Probability}, in dem er die drei Axiome der Wahrscheinlichkeitstheorie aufstellte. Es existiert ein Wahrscheinlichkeitsraum, der sich aus den drei Elementen $\Omega$, $\mathcal{F}$, $\mathbb{P}$ zusammensetzt.
>>>>>>> 70aebc3b96f943eec61bf519bf64e708ab927277

\vspace{5pt}

\begin{enumerate}
	\item $\Omega$ ist eine Menge mit einer endlichen Anzahl an Elementen.
	\item $\mathcal{F}$ ist die Menge aller Ereignisse E, also Teilmengen von $\Omega$. Das hei"st f\"ur alle $E \in \mathcal{F}$ gilt $E \subseteq \Omega$.
	\item $\mathbb{P}: \mathcal{F} \rightarrow \mathbb{R}$ bezeichnet die Wahrscheinlichkeitsverteilung. Diese weist jedem Element in $\mathcal{F}$ (also Ereignis) eine reelle Zahl zu.  
	
	
\end{enumerate}

\vspace{10pt}


Die Wahrscheinlichkeit eines Ereignisses $E$ fuer unsere speziellen Wahrscheinlichkeitsr\"aume wird errechnet durch $ \mathbb{P}(E) = \frac{|E|}{|\Omega|}$.
Alternativ definieren wir $p_{i} = \mathbb{P} (\{i\})$ mit $p_{i} \geq 0$  
f\"ur $i \in \Omega$, f\"ur die gilt $\sum_{i\in \Omega} p_{i} =1$. Dann ist $\mathbb{P}(E) = \sum_{i\in E} \mathbb{P}(\{i\})$. 


\begin{Def}[Kolmogorovs Axiome]

\vspace{5pt}

\begin{enumerate}
	\item Die Wahrscheinlichkeit eines Ereignisses ist eine positive, reelle Zahl, f\"ur die gilt \\ $\mathbb{P} (E) \geq 0$ f\"ur alle $E \in \mathcal{F}$.
	\item Die Menge aller Ergebnisse bezeichnet man als sicheres Ereignis, das die Wahrscheinlichkeit $\mathbb{P} (\Omega) = 1$ hat.
	\item Die Wahrscheinlichkeit der Vereinigung von disjunkten Ereignisssen ist gleich der Summe der Wahrscheinlichkeiten der disjunkten Ereignisse. 
	\vspace{3pt}
	\begin{equation*}
	\mathbb{P} (\cup_{i=1}^m E_{i}) = \sum_{i=1}^n \mathbb{P} (E_{i}) \, \text{f\"ur alle } \, E_{1},...,E_{m} \in \mathcal{F}
	\end{equation*}
\end{enumerate}
\vspace{5pt}

$E_{1},...,E_{m}$ sind disjunkt, falls $E_{i} \cap E_{j} = \emptyset$ f\"ur alle $i,j= 1,2,...,m$. 
Hierbei bezeichnet $\emptyset$ ein unm\"ogliches Ereignis $\emptyset \in \mathcal{F}$ fuer welches gilt $P(\emptyset) = 0$.
\end{Def}


\vspace{10pt}

Aus Kolmogorovs Axiomen lassen sich direkt folgende \textit{Konsequenzen} ableiten:


\vspace{5pt}

\begin{enumerate}
	\item $\mathbb{P} (\emptyset) = 0$
	\item Monotonie: Wenn immer $A\subseteq B$ dann: $\mathbb{P} (A) \leq \mathbb{P} (B)$
	\item $0 \leq \mathbb{P} (E) \leq 1$   f\"ur alle $E \in \mathcal{F}$
	\item $ \mathbb{P} (A \cup B) = \mathbb{P} (A) + \mathbb{P} (B) - \mathbb{P} (A \cap B)$ f\"ur alle $A,B \in \mathcal(F)$
\end{enumerate}


\vspace{10pt}

\begin{Def}[Zufallsvektoren und -variablen]
In einem Wahrscheinlichkeitsraum $(\Omega, \mathcal{F}, \mathbb{P})$ k\"onnen wir eine Funktion $X:\Omega \rightarrow \mathbb{R}^k$  definieren, die dann einen k-dimensionalen Zufallsvektoren beschreibt. Im Fall von $k=1$ sprechen wir von einer Zufallsvariablen.
\end{Def}

Bei einem W\"urfelwurf w\"are eine moegliche Zufallsvariable die Augenzahl des W\"urfels. Wir schreiben $\Omega =  \{1,2,3,4,5,6\}$. 
Ein weiteres Beispiel ist die Einteilung in \glqq gerade\grqq{} und \glqq ungerade\grqq{} Zahlen. Hier nimmt die Variable $1$ an, wenn die Augenzahl gerade ist und $0$, wenn die Zahl ungerade ist.

\vspace{10pt}

\paragraph {Summe von Zufallsvariablen}
Gegeben seien zwei unabhängige Zufallsvariablen
\begin{equation*}
X_{1}:\Omega \rightarrow \mathbb{R} \text{ und }  X_{2}:\Omega \rightarrow \mathbb{R}.
\end{equation*}
Wir berechnen nun die Verteilung der Summe $Y=X_{1}+X_{2}$. Daf\"ur beobachten wir,  dass
\begin{equation*}
\{Y=l\}=\bigcup_{i=-\infty}^{\infty} \{X_{1}=1\} \cap \{X_{2}=l-i\}
\end{equation*}
sodass die Verteilung direkt berechnet werden kann durch
\begin{align}
\label{eq:zvsumme}
&\mathbb{P} (\{Y=l\}) \\
=\, &\sum_{i=-\infty}^{\infty} \mathbb{P} (\{X_{1}=i\} \cap \{X_{2}=l-i\})\nonumber\\
=\, &\sum_{i=-\infty}^{\infty} \mathbb{P} (\{X_{1} = i\})  \mathbb{P} (\{X_{2}=l-i\}) \nonumber. 
\end{align}


\subsection{Statistiken}

\begin{Def}[Erwartungswert]
 Der Erwartungswert einer Zufallsvariablen beschreibt die Zahl, die die Zufallsvariable im Mittel annimmt. Sie wird definiert durch
\begin{equation*}
\mathbb{E} [X] = \sum_{i \in \Omega} X (i) \mathbb{P} (\{i\}) = \sum_{x\in\mathcal{X}} x \, \mathbb{P} (\{X=x\}) \text{,}
\end{equation*}
wobei $\mathcal{X}$ ein Wertebereich von $X$ ist.
\end{Def}

Wirft man also beispielsweise mit zwei W\"urfeln und addiert deren Augenzahl, sagt der Erwartungswert, welche Zahl nach vielen Versuchen am ehesten erwartet werden kann. Durch Formel~\eqref{eq:zvsumme} ist dies bei zwei W\"urfeln $7$.


Folgender spezieller Erwartungswert kennzeichnet die Ausdehnung einer Wahrscheinlichkeit. 


\begin{Def}[Varianz]
Die Varianz einer Zufallsvariable $X$ ist definiert durch
\begin{equation}
\label{eq:Varianz}
X= \mathbb{E} [X^2] - \mathbb{E} [X]^2 = \mathbb{E} [ (X - \mathbb{E} [X] )^2 ].
\end{equation}

\end{Def}

Berechnet man den Erwartungswert eines einfachen W\"urfelwurfes, so erh\"alt man $3.5$. Die Varianz eines Wurfes liegt durch Formel~\eqref{eq:Varianz} bei $2.92$. Die hohe Streuung suggeriert, dass die Wahrscheinlichkeit f\"ur jede Seite des W\"urfels relativ hoch ist.

\begin{Def}[Marginale und bedingte Wahrscheinlichkeiten]
\label{def:Wkeits}
Es seien $X_{1}, X_{2}$ Zufallsvariablen auf dem Wahrscheinlichkeitsraum $\Omega_{1}\times \Omega_{2}$, dann ist die \emph{marginale} Wahrscheinlichkeit von $X_{1}$ definiert durch 

\begin{equation*}
\mathbb{P} (\{X_{1} \in A \} ) = \sum_{b \in \Omega_{2} } \mathbb{P} (\{X_{1} \in A\} \cap \{X_{2} = b\})
\end{equation*}

Eine Wahrscheinlichkeit ist \emph{bedingt}, wenn die Wahrscheinlichkeit des Eintretens von einem Ereignis $A$ davon abhängt, ob auch Ereignis $B$ eintritt. 
Diese ist bestimmt durch:

\begin{equation*}
\mathbb{P} (\{X_{1} \in A \} | \{ X_{2} \in B \} ) = \frac {\mathbb{P} ( \{X_{1} \in A \} \cap \{X_{2} \in B \})} {\mathbb{P} (\{X_{2} \in B\})}
\end{equation*}
\end{Def}

\begin{Def}[Unabh\"angigkeit von Ereignissen]

\begin{enumerate}
	\item Wenn f\"ur Ereignisse $E, F \in \mathcal{F} $ gilt
	\begin{equation} 
	\mathbb{P} (E\cap F)= \mathbb{P} (E)  \mathbb{P} (F),
	\end{equation}
        dann sind E und F unabhängig. 
\item Zwei Zufallsvariablen $X_1 : \Omega_1 \to \mathbb{R}, X_2:\Omega_2 \to \mathbb{R}$
%% \begin{equation*}
%% %X_{1}: \Omega_{1} \rightarrow \mathbb{R} \: \text{ und } \: X_{2}: \Omega_{2} \rightarrow \mathbb{R}
%% X_{1}: \Omega_{1} \rightarrow \mathbb{R} : \text{ und } : X_{2}: \Omega_{2} \rightarrow \mathbb{R}
%% \end{equation*}
sind unabhängig, falls gilt, dass für alle $A, B \subseteq \mathbb{R}$
\begin{equation*}
\mathbb{P} (\{X_{1} \in A\} \cap \{X_{2} \in B\} ) = \mathbb{P} (\{X_{1} \in A\})  \mathbb{P} \{X_{2} \in B\} )
\end{equation*}
\end{enumerate}

Aus obiger Definition von Unabhaengigkeit und bedingter Wahrscheinlichkeit in Definition \ref{def:Wkeits} folgt sofort, dass zwei Zufallsvariablen  genau dann unabh\"angig sind wenn
\begin{equation*}
P(\{X_1 \in A\} | \{X_2 \in B\}) = P(\{X_1\in A\})
\end{equation*}
f\"ur alle $A,B \subset \R$. 

\end{Def}


\begin{Thm}[Linearit\"at des Erwartungswerts]
\label{thm:explinearity}
Für die Summe beliebiger Zufallsvariablen $X_{1}, \dots, X_{n}}$ gilt
\begin{equation*}
\mathbb{E} \left[\sum_{i=1}^{n} a_{i} X_{i}\right] = \sum_{i=1}^{n} a_{i} (\mathbb{E} [X_{i}] )
\end{equation*}
\end{Thm}

Man sagt auch dass der Erwartungswert linear ist, aehnlich wie bei einer linearen Abbildung die Funktion der Summe ist gleich die Summe der Funktionen. F\"ur den Beweis von Theorem~\ref{thm:explinearity} benutzt man vor allem die Verteilung der Summen von Zufallsvariablen~\eqref{eq:zvsumme}.

\paragraph{Wahrscheinlichkeitsdichten}

Gegeben eine Funktion $f: \mathbb{R} \rightarrow \mathbb{R}$ f\"ur die gilt 
\begin{equation*}
\int_{\mathbb{R}} f(x) dx = 1,
\end{equation*}
k\"onnen wir eine Wahrscheinlichkeitsverteilung folgenderma"sen definieren:
\begin{equation*}
\mathbb{P} (\{X\in A\}) = \int_{X \in A} f(x) dx. 
\end{equation*}


\subsection{Maximum-Likelihood-Methode}

Die Maximum-Likelihood-Methode wird in der Statistik angewandt, um mithilfe einiger i.i.d. (unabhaengig und identisch verteilt) Datenpunkte auf die ihnen zu Grunde liegende Gesamtverteilung schlie"sen zu k\"onnen. Die Methode kann auf parametrisierte Verteilungen angewandt werden, wie z.B. die Gaussverteilung die ihr Maximum (und Erwartungswert) an der Stelle $\theta$ hat. Wenn man $2$ Datenpunkte hat, l\"asst sich die Gaussverteilungs-Kurve, bei der $\sigma$ sicher ist, beliebig verschieben, sodass beide Datenpunkte dem Maximum $\theta$ m\"oglichst nah sind.
Die Wahrscheinlichkeit, dass wir unter der Verteilung mit Parameter $\theta$ den Datenpunkt $x$ ziehen, ist $p(x,\theta)$. 

\begin{equation*}
\hat{\theta}_{ML} = \underset{\theta}{\operatorname{argmax}} \,\, p (x_{1},...,x_{r};\theta)
\end{equation*}

\noindent hei"st Maximum-Likelihood-Sch\"atzer (von $\theta$) für $x_{1},...,x_{n}$ gezogen aus $p(x_{1},...,x_{r};\theta$). Da die Datenpunkte $x_1, \dots, x_n$ i.i.d. verteilt sind, kann man die gemeinsame Verteilung faktorisieren und erh\"alt
\begin{equation*}
p (x_{1},...,x_{r};\theta) = \prod_{i=1}^r p(x_i;\theta).
\end{equation*}

Verwendet man auf diese Formel den Logarithmus, wird das Produkt in eine Summe umgewandelt. Dadurch l\"asst sich die Form sehr gut optimieren bzw. maximieren. 


\paragraph{Lineare Regression} 
Man betrachte das Modell
\begin{equation*} 
y_{i} = ax_{i}+\epsilon_{i} \text{ mit } y_{i}, x_{i} \in \mathbb{R}
\end{equation*}
fuer gegebene $x_i$, wobei man annimmt dass das Rauschen $\epsilon_i \overset{i.i.d.}{\sim} \mathcal{N}(0,\sigma^2)$ normal verteilt ist. Hier ist der zu bestimmende Parameter $\theta = a$. Fuer diesen konkrete Fall erh\"alt man bei der Maximum-Likelihood Methode fuer die Sch\"atzung von $a$ dasselbe Optimierungsprogramm wie dasjenige in der \textit{Linearen Regression}, und zwar 
Maximiert man nun die Likelihood Funktion \"uber $a$, d.h.
%% \begin{equation*} 
%% \underset{a}{\operatorname{max}} \,\, p (y_{1},...,y_{n};a)
%% \end{equation*}
%% so kommt man zu folgendem Ergebnis:
\begin{align*}
\hat{a}_{ML} &= \underset{a}{\operatorname{argmax}} -\frac{1}{2} \sum_{i=1}^{n} (y_{i} -ax_{i})^{2}\\
&= \underset{a}{\operatorname{argmin}} \sum_{i=1}^n (y_i - ax_i)^2
\end{align*}
Dies ist \"aquivalent zur \textit{Methode der kleinsten Quadrate} die im Optimierungsabschnitt~\ref{sec:regression} eingef\"uhrt wurde.


\section{Machine Learning}
\author{Farhadiba Mohammed, Dennis Kempf, David Steinmann}
Machine Learning ist ein Konzept, bei dem der Computer durch Algorithmen aus Daten lernt, gewisse Probleme selbstständig zu lösen. Dabei erkennt der Computer nach einer gewissen Lernphase Gesetzmä\ss igkeiten, durch die er dann in der Lage ist, mit neuen, ihm unbekannte Datensätze umzugehen. Dabei werden verschiedene Modelle genutzt, doch zunächst einmal sollten wir uns mit den grundsätzlichen Begriffen Klassifikation und Regression vertraut machen.

\subsection{Klassifikation und Regression}
\author{David Steinmann}
Bei Klassifikation und Regression handelt es sich um zwei verscheidene Möglichkeiten zur Problemlösung im Bereich Machine Learning.

\paragraph{Klassifikation.}
Bei der Klassifikation sorgt der Algorithmus am Ende dafür, dass der Datensatz vom Computer in verschiedene Klassen eingeteilt wird. Die Klassen müssen vorher vom Mensch festgelegt werden. Der Computer klassifiziert die Datensätze dann anhand der Attribute (sogenannten \emph{features}) des entsprechenden Datensatzes. Die klassifizierten Datensätze enthalten dann meistens eine weitere Dimension, in der die Klasse anhand einer Zahl gespeichert ist.

\paragraph{Regression.}
Bei der Regression geht es darum, den entsprechenden Datensätzen am Ende einen festen Wert zuzuordnen. Der Unterschied zur Klassifikation besteht darin, dass der zugeordnete Wert nicht für eine Klasse steht, sondern eine konkrete reelle Zahl beinhaltet. Ein Beispiel wäre der Preis von einer Wohnung. Diese wurde mit vielen verscheidenen Attributen in die Datenbank eingespeichert (zum Beispiel Größe, Lage, ob sie von einem) der Endwert wäre dann zum Beispiel der konkrete Preis und nicht eine Klassifizierung in \glqq teuer\grqq , \glqq mittelteuer\grqq , \glqq billig\grqq . 

\subsection{Support Vector Machines}
\author{David Steinmann}
Support Vector Machines oder kurz SVMs sind ein Konzept zum Lösen von Machine Learning Problemen.
Dabei werden für alle Datensätze zuerst gewisse Features festgelegt. Diese Features entsprechen den Attributen des Datensatzes. Je nach Problem kann es unterschiedlich viele Features geben, doch allgemein kann man sagen, dass, je mehr Features es sind die Genauigkeit des Programms genauer wird.

Diese Features werden normalerweise in einem Vektor für den entsprechenden Datensatz gespeichert.
Dieser Vektor wird $x_{i}$ genannt wobei $i = 1, ..., n$ den Datensatz beschreibt mit  $x \in  \mathbb{R}^m$.
Zusätzlich hat jeder Datensatz noch einen weiteren Wert, welcher das >>Ergebnis<< des Datensatzes beschreibt, der $y_{i} \in \mathbb{R}$ oder auch >>Label<< genannt wird.

Die Features und das Label zusammen beschreiben einen Punkt im n-dimensionalen Koordinatensystem, wobei $n = m + 1$, da zu den $m$ Dimensionen von $x$ noch die eine Dimension von $y$ dazukommt.

Durch die Features ist es dann möglich den Datensätzen einen bestimmten Ort im n-dimensionalen Koordinatensystem zuzuordnen. Diese können dann auf Grund ihrer räumlichen Anordnung klassifiziert werden. Dabei soll der Abstand von beiden Gruppen von Punkten zu dem Trennobjekt maximal sein. Dadurch kann die Klassifizierung optimal durchgeführt werden. Andernfalls würden schon geringe Abweichungen von den Test-Datensätzen reichen, das ein Datensatz falsch klassifiziert würde (Siehe Abb. \ref{SVM1}). 


\begin{dsafigure}
\begin{center}
	\label{SVM1}
	\includegraphics[width=0.35\textwidth]{\media Figure_SimpleSVM}
	\caption{Da der Abstand zwischen den unterschiedlich klassifizierten Datensätzen maximiert werden soll, gilt die durchgezogene und nicht die gestrichelte Linie als Trennelement.}
	\end{center}
\end{dsafigure}


Um die Trennlinie zu optimieren, gibt es auch eine mathematische Beschreibung. Ebenen kann man auch in der Form $\langle x, w \rangle = b $ beschreiben. Daraus folgt, dass das Optimierungsproblem wie folgt geschreiben werden kann.

\begin{align*}
	\underset{w,b}{\operatorname{max}} \quad & \frac{2}{\norm{w}} \\
	\operatorname{sodass} \quad &y_{i}(\langle w,x_{i} \rangle -b) \geq 1 \quad \forall  i = 1,...,n ; \\
	&x \in \mathbb{R}^n; \quad	y \in \{1;-1\}
	 %\label{SVMProblemMax}
\end{align*}

Dabei wird der Abstand zwischen den innersten Ebenen durch die beiden Datensätze ($\frac{2}{\norm{w}}$) (Siehe Abb. \ref{SVM2}).

Unter der Vorraussetzung, dass alle vorhandenen Datensätze richtig klassifiziert sind. Dieses Problem kann 
man aber genauso schreiben als:

\begin{align*}
		\underset{w,b}{\operatorname{minimiere}} \quad & \norm{w} \\
		\operatorname{sodass} \quad &y_{i}(\langle w,x_{i} \rangle -b) \geq 1 \quad \forall i = 1,...,n ; \\
		&x \in \mathbb{R}^n; \quad y \in \{1; -1\}
		%\label{SVMProblemMin}
\end{align*}

Klassifikation:

\begin{align*}
	y_{i} =& \{ 1 \quad \operatorname{falls} \langle w,x_{i} \rangle - b \geq 0\} \\
		& \{ -1 \operatorname{falls} \langle w,x_{i} \rangle - b \leq 0\}
	%\label{KlassifikationSVM}
\end{align*}

Nachdem das Problem optimiert wurde, ist der Computer in der Lage, weitere Daten einzuordnen, vorrausgesetzt, es ist richtig optimiert.
Dennoch kann es bei dieser Art von SVMs zu einigen Problemen kommen.\\

\begin{dsafigure}
	\begin{center}
		\includegraphics[width=0.5\textwidth]{\media Figure_SVM.pdf}
		\caption{Links und rechts zur Trenngeraden befinden sich die parallelen Grenzen (gestrichelte Geraden). Ziel der Optimierung ist es, den Abstand zwischen den Grenzen zu maximieren, um den Normalenvektor $w$ zu bestimmen.}
		\label{SVM2}
	\end{center}
\end{dsafigure}

\begin{itemize}
	\item Wenn die Datensätze nicht linear separierbar sind, d.h., es ist nicht möglich die beiden Datensätze mit einem linearen Element zu trennen.
	
	\item Wenn die Daten nicht alle korrekt klassifiziert sind oder es Daten gibt, die in dem >>Bereich<< der anderen Seite liegen.
\end{itemize}

\paragraph{Soft Margin SVMs}
Die Soft Margin SVMs können mit beiden Problemen umgehen.
Sie sind im Prinzip wie herkömmliche SVMs aufgebaut, nur dass sie eine gewisse Fehlertoleranz besitzen. Diese kommt durch eine Veränderung an der Formel zustande:

\begin{align*}
	\underset{w,b}{\operatorname{minimiere}} \quad & \norm{w} + C\sum_{i}{z_{i}} \quad \forall i = 1,...,n \\
	\operatorname{sodass} \quad & y_{i}(\langle w,x_{i} \rangle - b) \geq 1 - z_{i} \quad \forall i = 1,...,n; \\ 
	&x \in \mathbb{R}^n; \quad y \in \{1; -1\} \quad z_{i} \geq 0 \\
	%\label{eq: SoftMarginSVMs}
\end{align*}

Dabei steht $z_{i}$ für die Grösse des Fehlers des Punktes $x_{i}$ und C ist eine Konstante, deren grösse bestimmt, wie stark die Fehler gewichtet werden, da die Konstante mit der Summe der Fehler multipliziert wird. Da die gesamte Gleichung aber minimiert werden soll sorgt ein hohes C dafür, dass das Problem schwerer optimiert werden kann. Die Konstante muss vom Mensch selbst gewählt werden, je nachdem, wie schwer Fehler gewichtet werden sollen.
Um die optimale Größe von C für das entsprechende Problem zu finden wird ein System namens Crossvalidation genutzt. Bei diesem System wird der Algorithmus mehrmals auf den gleichen Testdatensatz angewendet, C dabei aber variiert. Damit kann man herausfinden, bei welchem Wert von C der Algorithmus optimal funktioniert.


\section{Neuronale Netze}
\author {Farhadiba Mohammed}

\subsection{Einführung}

Inspiriert von unserem Verständnis, wie das menschliche Gehirn lernt, benutzen Neuronale Netze Lernalgorithmen, welche besonders für praktische Anwendungen geeignet sind.
Dazu zählen Spracherkennung, Objekterkennung in Bildern und die Fähigkeit individuell passende Produkte vorzuschlagen, die dem Kunden gefallen könnten. 
Ein Neuronales Netz wird von mehreren Schichten aufgebaut. Ausgangsschicht ist dabei, die Datenschicht, auf die ein oder mehrere hidden layer folgen. Als Ausgabewert erhält man schließlich einen Vektor, welcher die Wahrscheinlichkeitsverteilung darstellt. Mit anderen Worten, wie wahrscheinlich es ist, dass das Ausgangsobjekt zu einer bestimmten Klasse gehört.

\begin{dsafigure}
	\begin{center}
		\includegraphics[width=0.35\textwidth]{\media Figure_NN.pdf}
		\caption{Ein vollständig verbundenes neuronales Netzwerk mit $i$ Eingängen und $k$ Ausgängen, bestehend aus $n$ Schichten mit jeweils $m$ \glqq Neuronen\grqq .}

		\label{FigNN}
	\end{center}
\end{dsafigure}

\subsection{Deep Learning}

Mit Deep Learning beschreibt man die Neuronalen Netze, die über mehr als einen >>versteckte Schicht<< verfügen. Damit ist gemeint, dass sich zwischen Ein- und Ausgabeschicht weitere Schichten befinden. 

Um die Anzahl an Parametern zu vergrößern, werden diese auch über nicht-lineare Funktionen miteinander verknüpft. Dadurch kann eine höhere Genauigkeit erzielt werden. Jedoch kann eine zu hohe Anzahl an Parametern auch dazu führen, dass das Netzwerk >>overfitted<<, d.h., zu sehr an den Trainingsdatensatz angepasst, wird. Dann kann das Netzwerk neue, fremde Daten nicht mehr korrekt klassifizieren. 

Wie wir in der Abbildung \ref{FigNN} sehen können ist ein typisches Fully connected Neuronales Netz abgebildet. Als Basis findet man unten die Datenschicht mit ihren Eingabewerten in Form eines Vektors $(x_1 \dots x_i)$. Diese Werte werden nun mit einem jeweiligen Faktor $w$ multipliziert. Hierbei handelt es sich um ein Skalarprodukt. Das Ergebnis wird nun im ersten hidden layer abgespeichert. Darauf wird eine nicht lineare Funktion angewendet und das Ganze wird erneut mit einem Faktor $w$ multipliziert. Dieser Vorgang wiederholt sich so lange bis uns schließlich ein Vektor mit seinen Komponenten $(p_1 \dots p_k)$ zurückgegeben wird, welcher als Wahrscheinlichkeitsverteilung interpretiert werden kann. Anhand eines konkreten Beispiels würde es folgendes bedeuten:
Nehmen wir an wir haben ein Bild und wollen ermitteln zu welcher Klasse Haus, Tisch oder Stuhl das daurauf abgebildete Objekt gehört. Unsere inputs wären demnach die Pixel des Bildes. Diese durchlaufen nun das Neuronale Netz und wir erhalten eine Wahrscheinlichkeitsverteilung als Rückgabewert, welche uns mitteilt, dass es am wahrscheinlichsten ist, dass das abgebildete Objekt ein Objekt der Klasse Stuhl ist. Demnach ist die Klassifizierung vollzogen.

\begin{align*}
a_j^{(k)} = f^{(k)} (\langle w_{j}^{(k)}, a_i^{(k-1)}\rangle) = f^{(k)} (\sum_{i=1}^{m^{(k-1)}} w_{ji}^{(k)},a_i^{(k-1)} )
\end{align*}

\subsection{Convolutional Neural Networks}

\begin{dsafigure}
	\begin{center}
		\includegraphics[width=0.5 \textwidth]{\media cnn.png}
		\caption{Ein Convolutional Neural Network (CNN) (dt.: >>faltendes neurales Netzwerk<<) %mit vier Eingängen ($x_1, ..., x_4$) und zwei Ausgängen ($p_1, p_2$). Dazwischen befindet sich eine Schicht aus drei >>Neuronen<<, die als Filter wirkt.}
		}
		\label{FigConvNN}
	\end{center}
\end{dsafigure}


Bei den Convolutional Networks handelt es sich um ein Neuronales Netz mit einer vereinfachten Struktur. Diese kommt dadurch zu Stande, dass nun kein fully connected Neuronales Netzwerk mehr vorliegt. 

\begin{dsafigure}
	\begin{center}
		\includegraphics[width=0.4 \textwidth]{\media ConvolutionalNN.png}
		\caption{Ein Convolutional NN mit einer Schicht aus neun >>Neuronen<< dazwischen, die als Filter wirkt. }
		\label{FigConvNN}
	\end{center}
\end{dsafigure}

Der existierende Filter {w} wird auf die Eingabematrix angewendet. Dabei wird das Skalarprodukt berechnet und in der Endmatrix als Ergebnis festgehalten. Daraufhin wird der Filter immer um eine Stelle weiter nach rechts in der Eingabematrix verschoben und das neue Skalarprodukt berechnet bis das Ende der Eingabematrix erreicht wurde.

\subsection{SVMs als Neuronale Netze}

Wir haben bereits SVMs kennengelernt, die durch das folgende Optimierungsproblem beschrieben werden:

\begin{align*}
\underset{b,w,z}{\text{minimiere}} &\qquad\norm{w} + C  \sum_{i=1}^{n} z_i\\
\text{sodass} &\qquad y_i (\langle w,x \rangle - b) \geq 1 - z_i \text { mit } z_i \geq 0
\end{align*}

Diese Nebenbedingung lässt sich umschreiben als:


\begin{align*}
&y_i (\langle w,x \rangle - b) \geq 1 - z_i \text { mit } z_i \geq 0 \\
\Rightarrow &z_i \geq 1- y_i (\langle w,x \rangle - b) \text{ mit } z_i \geq 0
\end{align*}

Da über $z_i$ minimiert wird und $C$ positiv ist, nehmen die Variablen $C$ immer die Grenzen an.

\begin{align*}
\min &\norm{w} + C \cdot \sum_{i=1}^{n} \max (0, 1-y_i (\langle w,x_i\rangle-b) \\
 = &\norm{w} + C \sum_{i=1}^{n} \ell ( 1-y_i (\langle w,x_i\rangle)-b) \\
=&\norm{w} + C \sum_{i=1}^{n} \ell (a_i)
\end{align*}

Charakteristisch für dieses Neuronale Netz ist, dass es nur einen Layer besitzt. Im folgenden wird verdeutlicht, wie man diese auch als Neuronales Netz ausdrücken kann. 

\subsection{Lineare Regression}

Die lineare Regression kann auch in Form von Neuronalen Netzen ausgedrückt werden. 


\begin{align*}
\min\limits_{w,b} \ & \frac{1} {2} \sum_{i=1}^{n} ((\langle w,x_i\rangle- b) -y_i)^2 \\
&a_i = \langle w,x_i\rangle -b-y_i
\end{align*}

\subsection{MNIST}

\begin{dsafigure}
	\begin{center}
		\includegraphics[width=0.4 \textwidth]{\media mnistExamples.png}
		\caption{mnist Example }
		\label{FigConvNN}
	\end{center}
\end{dsafigure}

Das MNIST bietet Daten zur freien Verfügung, damit man seine SVMs und Neuronalen Netze mit den Daten trainieren kann. Mit diesen MNIST Trainingsets haben wir uns im Kurs beschäftigt, um unsere maschinellen Lernalgorithmen zu trainieren, damit sie am Ende Vorhersagen treffen können.





\section{Finanzanalyse}
\author{Dennis Kempf, Moritz Hollenberg, Patrice Becker}

Die Finanzanalyse versucht künftige Kursverläufe vorherzusagen, um Gewinne im Finanzhandel zu erzielen.

\subsection{Formen der Finanzanalyse}
\author{Dennis Kempf}

Die drei grundlegenden Formen der Finanzanalyse bilden die \emph{Fundamentalanalyse}, \emph{Technische Analyse} und die \emph{Sentimentanalyse}. 

Die Fundamental- und die Technische Analyse gehen davon aus, dass nicht alle verfügbaren Informationen in den Kursen verarbeitet sind. Es sei somit möglich, schneller als andere Marktteilnehmer zu agieren und daraus Profite zu erzielen.

\subsubsection{Fundamentalanalyse}
\author{Dennis Kempf}

Bei der Fundamentalanalyse werden sogenannte \emph{Fundamentaldaten} analysiert, um den wirklichen Wert eines Finanzproduktes zu bestimmen. Zu diesen Daten zählen unter anderem:\\ 

\begin{itemize}
	\item Kurs-Gewinn-Verhältnis
	\item Gesamtkapitalrendite
	\item Eigenkapitalquote
	\item Bruttoinlandsprodukt
	\item Einzelhandelsverkäufe
\end{itemize}

\subsubsection{Technische Analyse}
\author{Dennis Kempf}
\label{sssec:TechnischeAnalyse}

Im Gegensatz zur Fundamentalanalyse werden bei der Technischen Analyse \emph{Charts}, d.h. Abbildungen von Kursverläufen analysiert. Dazu werden sowohl reine Preisverläufe, als auch von diesen abgeleitete \emph{Indikatoren} berücksichtigt. Zu diesen Indikatoren gehören unter anderem:\\

\begin{itemize}
	\item Moving Average
	\item Bollinger Bands
	\item Stochastic Oscillator
	\item Relative Strength Index
	\item Fractals
\end{itemize}

\begin{dsafigure}
	\begin{center}
		\includegraphics[width=0.5\textwidth]
		{\media EURUSDH1.png}
		\caption{Ein \emph{Chart} (eine Kerze $\hat =$ eine Stunde) des EUR/USD Währungspaares mit den genannten Beispielindikatoren aus \ref{sssec:TechnischeAnalyse}}
		\label{fig:Beispielchart}
	\end{center}
\end{dsafigure}

\subsection{Sentimentanalyse}
\author{Dennis Kempf}

Die Sentimentanalyse befasst sich mit der Stimmung von Investoren, um daraus zu schließen, ob eine \emph{bullische} Phase (steigender Trend) oder eine \emph{bärische} Phase (fallender Trend) bevorsteht. Dazu können Mittel wie etwa \emph{Meinungsumfragen} oder die Analyse von \emph{Börsenbriefen} eingesetzt werden.

Lorem ipsum...
\section{Künstliche Intelligenz}

Computer begegnen uns täglich im Alltag. Die Wissenschaft, die sich mit ihnen beschäftigt, wächst exponentiell. Können Computer so intelligent wie Menschen sein? Wer übernimmt die Verantwortung, wenn es soweit kommt?

Künstliche Intelligenz (KI) bezeichnet den Versuch, spezifische, menschliche Verhaltensweisen, vor allem die Intelligenz, maschinell nachzuahmen. Im besten Fall den Menschen in seinen Fähigkeiten zu übertreffen. Hierbei verbindet die KI computerwissenschaftliche und kognitionswissenschaftliche Forschung.
Es geht nicht nur um rationale und rationelle Fertigkeiten des Menschen, sondern auch um die Nachahmung menschlichen Empfindens, Erkennens und Wahrnehmens.
Wenn man den Begriff KI näher definiert, muss man zwischen zwei verschiedenen Arten unterscheiden: der starken und der schwachen KI. Die schwache KI wurde für bestimmte Anwendungsdomänen entwickelt. Ein Computer probiert, Intelligenz zu simulieren. Im Gegensatz dazu verleiht die starke KI Computern intellektuelle Fähigkeiten, wie z.B. logisches Denken, Treffen von Entscheidungen bei Unsicherheit, Planen, Lernen, Kommunikation in natürlicher Sprache. Für ein als gelungen angesehenes Ergebnis soll eine starke KI all diese Fertigkeiten miteinander verbinden. Er soll natürliche Kreativität und Emotionen nachbilden, gleichzeitig Bewusstsein bzw. Selbstbewusstsein entwickeln.
Die KI wird in unterschiedlichsten Bereichen angewendet. Viele davon begegnen uns regelmäßig im Alltag. Grundsätzlich lassen sich die Anwendungsbereiche in drei Kerndisziplinen aufteilen. Zuerst zu nennen sind Expertensysteme, die für Maschinen in Medizin, Forschung, Militär, Ausbildung und Ingenieurwissenschaften angewandt werden. Außerdem gibt es die Verarbeitung natürlicher Sprachen, welche sich mit dem Übersetzen, Erkennen und Wiedergeben auseinandersetzt. Die 3. große Kerndisziplin ist die Robotik, die den Menschen z.B. in der Industrie teilweise ersetzt.
Die Geburtsstunde der KI liegt in dem 1947 von Alan Turing veröffentlichten Aufsatz „Intelligent Machinery“. Nach ihm ist ein Test zum Nachweis starker künstlicher Intelligenz benannt, dem Turing Test. Bei diesem Test sind zwei Menschen und eine Maschine beteiligt. In einem Raum befindet sich eine Person, die nur über Tastatur und Bildschirm entweder mit einer Maschine oder einem Menschen kommuniziert, ohne zu wissen, wann sie mit wem redet. Das Ziel ist es dies durch geschicktes Fragen herauszufinden. Wenn die Wahrscheinlichkeit der korrekten Identifikation bei 50\% liegt, dann gilt die Versuchsperson als durch den Computer getäuscht und der Computer hat den Turing-Test bestanden.  Es besteht eine große Diskussion darüber, inwiefern dieser Test sinnvoll ist. Dafür spricht, dass in vielerlei Hinsicht eine menschliche Intelligenz getestet werden kann, da eine Vielzahl von Wissensbereichen einbezogen werden. Es geht dabei nicht nur um Wissen über die Welt, sondern vielmehr auch um soziale und emotionale Reaktionen bis hin zu religiösen Einstellungen. Ein weiterer positiver Punkt ist, dass dem Computer bei dem Test Flexibilität und Anpassungsfähigkeit abverlangt werden, schließlich muss er mit dem plötzlichen Wechsel zwischen Situationen z.B. dem Wechsel der Unterhaltung über das Wetter zu Witzen auseinandersetzen. Jedoch gab es von Beginn des Tests an Schwierigkeiten. Es ist möglich Computerprogramme genauso zu programmieren, dass sie auf die meisten Fragen antworten können. Wenn sie keine Antwort kennen, dann antworten sie mit Gegenfragen oder wenden Ablenkungsmanöver an. Damit sind wir bei einem grundsätzlichen Problem der Definition des Begriffes der künstlichen Intelligenz. Was bedeutet eigentlich Intelligenz? Wie kann man sie nachweisen? 
Der Duden definiert Intelligenz als
\begin{enumerate} \item Fähigkeit abstrakt und vernünftig zu denken und daraus zweckvolles Handeln abzuleiten;
\item Grundlage der Intellektualität. Dazu gehören wissenschaftliche, künstlerische, religiöse, literarische und journalistische Tätigkeiten. 
\end{enumerate}
Dass der erste Punkt bereits verwirklichbar ist, zeigte schon im Jahre 1997 der Supercomputer von IBM, der im Schachduell gegen den Weltmeister Garry Kasparov gewann. Schon damals war die Maschine dazu fähig 200 Positionen pro Sekunde zu berechnen. Der Begriff der Intellektualität wurde damals noch nicht betrachtet. 2014 ist die Wissenschaft weiter fortgeschritten. Das Google-Programm GoogleNet beschreibt präzise in ganzen Sätzen, was auf Fotos zu sehen ist.  Der Nahrungsmittelkonzern Nestlé hat 1000 sprechende Roboter namens Pepper in seinen Kaffeeläden in Japan als Verkäufer eingesetzt.  Auch hier ist die erste Bedingung der Definition von Intelligenz erfüllt: sowohl das Google-Programm, als auch der Roboter benötigt die Fähigkeit abstrakt zu denken und anschließend zweckvoll zu handeln, um die gestellte Aufgabe zu erfüllen. Doch erfüllt solch ein Google-Programm auch die 2. Eigenschaft? Kann man es als intellektuell bezeichnen, weil es Kunst erkennt, interpretiert und literarisch auswertet? Ein weiteres Beispiel, auch von Google entwickelt, ist das Programm \glqq Deep Dream\grqq. Dieses Programm ist dazu fähig, bereits vorhandene Bilder zu verändern, sodass sie einen neuen Eindruck vermitteln bzw. bestimmte Stimmungen wiederzugeben.  An dieser Stelle liegt die Frage nahe, inwiefern der Computer selbst als Künstler oder Schöpfer angesehen werden kann und damit auch intellektuelle Eigenschaften besitzt. Am Anfang von kreativer Schöpfung liegen Ideen, die ein Computer in dieser Form nicht haben kann. Dem Computer werden Daten vorgegeben, die er so verarbeitet, dass etwas Neues daraus entstehen kann. Jedoch ist es durch die Computer möglich, die Anzahl der Kunstwerke in kurzer Zeit stark zu vergrößern. Damit ist das eigentliche Ziel der KI-Forschung Vorgehensweisen der Informationsaufnahme und Informationsverarbeitung zu entwerfen, die menschlichem Problemlösungsverhalten näher kommen, und daraus Methoden zur qualitativen Verbesserung und Anreicherung herkömmlicher Systeme der Informatik abzuleiten, noch nicht erfüllt. Zu den bisher erreichten Teilzielen gehören vor allem die Entwicklung von Programmsystemen auf höherem Abstraktionsniveau und damit die Erarbeitung neuer Software-Techniken sowie verbesserter Formen der Mensch-Maschine-Kommunikation. Beispielsweise ist es Computern bereits möglich Straftaten auf Videoaufnahmen zu erkennen. Es ist diesem Programm möglich sich sehr gut an Situationen anzupassen. Außerdem ermöglicht es eine schnellere Problemlösung als es dem Menschen möglich ist. Durch die intelligenten Maschinen kommt es zu einer Zeitersparnis für den Menschen, da die Maschine selbstständig 24 Stunden die Aufsicht übernimmt. Dadurch wird die Leistungsfähigkeit stark erhöht, um nur einige Vorteile solcher Programme, welche durch künstliche Intelligenz funktionieren, zu nennen. Gleichzeitig gilt es aber nicht zu unterschätzen, dass der Mensch dadurch seine Verantwortung auf den Computer überträgt. Es kann immer sein, dass ein Computer eine Straftat \glqq übersieht \grqq, da er diese noch nicht erlernt hat. Der Mensch wird dem Computer vertrauen und nur stichprobenartige Tests zur Analyse des Programmes und dessen Fehler durchführen. Dieses Programm kann somit von Straftätern manipuliert werden. Ist es nicht besser deshalb Menschen diese Arbeit zu überlassen, auch unter Rücksichtnahme auf die vielen Arbeitsplätze, die durch die Programme der künstlichen Intelligenz immer weiter ersetzt werden. Am schwersten überwiegt jedoch das Argument des Kontrollverlustes. Die Menschen sind überfordert mit der sie überflutenden Datenmenge. Der Mensch muss der Chef der Maschinen bleiben, alles andere führt zu psychischen Krankheiten. Viele von uns kennen die Unsicherheit in Anbetracht all der Informationen, die über soziale Netzwerke ohne unser Wissen verbreitet werden. Können wir durch die Möglichkeit des Internets noch wissen, wer was über uns weiß? Die ständige Erreichbarkeit und die unendlichen Optionen überfordern die menschliche Psyche. Hinzu kommt die wachsende Abhängigkeit von den Geräten, die im täglichen Alltag nicht mehr wegzudenken sind. Ein Tag ohne Handy, Computer und Fernsehen – ist das noch möglich? Noch abhängiger sind wir von den lebensrettenden Maschinen der Medizintechnik, welche auf künstlicher Intelligenz basieren. Spätestens hier wird deutlich, dass das Leben mancher Menschen an Maschinen hängt. Dieses Beispiel zeigt, dass KIs eine große Bereicherung für das menschliche Leben sein können. So können sie genauer und objektiver Situationen analysieren. Auf der anderen Seite muss gesagt werden, dass dazu aber vorher das Einspeisen von Daten nötig ist. Der Computer führt lediglich einen Abgleich der momentanen Situation mit den vorgegebenen Daten ab und entscheidet sich für die größte Übereinstimmung. Dies ist keine Art von eigentlichem Denken. Weiterhin stellt es eine schwierige Aufgabe dar, dem Computer emotionale Intelligenz nachzuweisen. Da dem Menschen unklar ist, wie genau unsere Gefühle funktionieren, ist es fraglich ob diese auf ein Programm übertragen werden können. Wenn man jetzt davon ausgeht, dass es trotz dessen möglich wäre, würde dies zum effizienteren Handeln führen. Doch schlussendlich stellt sich die Frage, wer verantwortlich gemacht werden sollte, sollten KIs Fehlentscheidungen treffen und somit Menschen gefährden oder sogar umbringen? Wer ist dann schuldig – der Programmierer, die Gesellschaft, der Mensch selbst? Auf all diese Fragen gibt es keine eindeutige Antwort. Allein der menschliche Verstand kann die Richtigkeit der erarbeiteten Ergebnisse durch künstliche Intelligenzen beurteilen und überprüfen und somit auch verantworten.\\
Zusammenfassend lässt sich sagen, dass der Computer bereits eine starke Entwicklung hinter sich hat. 
Wir waren beeindruckt von dem Computer, der es schaffte, den Schachweltmeister zu besiegen und sind mitunter nicht minder amüsiert darüber, an was für profanen Aufgaben der Computer scheitert.  Denn nur wir können beurteilen, was richtig ist und was falsch. Der Maschine mag es (noch?) an Bewusstsein mangeln. Uns Menschen allerdings mangelt es an der Erkenntnis, wie überlegen wir ihr immer noch sind. Selbst wenn die Maschine schneller \glqq denkt\grqq  als wir: Was auch immer sie kann, einer von uns hat es ihr gegeben. Wir haben der Maschine die Mittel in die Hand gegeben, damit sie lernen kann. Damit tragen wir als Gesellschaft, jeder Einzelne von uns, eine immense Verantwortung, mit der wir nicht spielen dürfen. Jeder Schritt, den die Wissenschaft macht, hat Auswirkungen auf die Nachwelt. Die Frage, ob der Wissenschaft Grenzen gesetzt werden dürfen, wird viel diskutiert. Wenn sich jedoch jeder der Tragweite seines Handelns bewusst ist, dann kommt es nicht zu einem Kontrollverlust der Menschen gegenüber den Maschinen und die Frage nach einem Schuldigen, nach jemandem, dem man die Verantwortung im Falle einer Eskalation zuweisen wird, erübrigt sich. Unser Appell: Übernehmt Verantwortung!

\section{Einleitung}

Täglich werden in Deutschland rund 64 Millionen Briefe verschickt. Doch habt ihr euch schon einmal gefragt, wie die ganzen Briefe so schnell verarbeitet werden können? Die Vorstellung von mehreren tausend Mitarbeitern, die sich nur mit der Sortierung der Post beschäftigen, erscheint schon aufgrund des Aufwandes unwahrscheinich. Eine bessere Variante stellt die Benutzung von Schrifterkennungsalgorithmen dar, deren sich die Post seit längerer Zeit bedient. Dabei scannt ein Computer die auf den Umschlägen geschriebenen Adressen ein, wertet diese aus und sortiert sie entsprechend. \\
Zu diesem Thema gab es zwei Projektgruppen, die sich dem Problem auf verschiedene Weise näherten. Eine Gruppe nutzte SVMs, die andere ein neuronales Netzwerk.


\section{Schrifterkennung mit SVMs}

\paragraph{Aufbau des Programms}

Da es beim maschinellen Lernen essenziell ist über ein entsprechend großes Datenset zu verfügen, haben wir unseren Algorithmus zunächst einmal mit 60.000 Bildern ausgestattet. 

\begin{verbatim}
feature, hog_image = feats.hog(sub_pic,
 orientations = 8,pixels_per_cell = (4,4),
 cells_per_block = (1,1),visualise = true)
\end{verbatim}

Im ersten Schritt generieren wir die Features sämtlicher Bilder. Features sind z.B. Muster in einem Bild, wie Kanten und Farben oder allgemein gesagt, die Eigenschaften eines Objektes. Wie wir an dem Beispielbild mit dem Kameramann (Abb. \ref{Kameramann}) sehen können, wurden durch den Algorithmus die Kanten hervorgehoben und alles andere wurde ignoriert (Abb. \ref{Kameramann_hogfeat}). Dieses Featurebild kann als Vektor dargestellt werden und die SVM benutz solche Vektoren, um die Unterschiede der Trainingsdaten zu ermitteln. Danach kann die SVM neue Objekte, hier sind es Bilder, den richtigen Gruppen zuzuordnen.\\ 
Bei dem Erstellen der Features gibt es mehrere Einstellungsmöglichkeiten, beispielsweise zur Variation der Pixelanzahl in einer Zelle (\textit{pixels\_per\_cell}) und der Genauigkeit zur Erkennung von Konturen (\textit{orientations}). Dabei gilt: Je kleiner die Zellen sind und je höher die Genauigkeit ist, desto länger braucht der Algorithmus zum berechnen, aber desto präziser ist das Endergebnis. Allerdings muss man auch beachten, dass die Zellengröße nicht zu klein wird, da in sehr kleinen Zellen die Muster eventuell nicht für das Programm erkennbar sind. 

\begin{dsafigure}
\begin{center}
	\includegraphics[width=0.35\textwidth]{\media Kameramann.png}
	\caption{Schwarz-Weiß Bild eines Kameramannes.}
	\label{Kameramann}
\end{center}
\end{dsafigure}

\begin{dsafigure}
\begin{center}
	\includegraphics[width=0.35\textwidth]{\media Kameramann_hogfeat.png}
	\caption{Feature-Bild von Abb.\ref{Kameramann}.}
	\label{Kameramann_hogfeat}
\end{center}
\end{dsafigure}

\begin{verbatim}
clf[:fit](features_trainset', 
 label_train_data[:])
 ...
prob = clf[:predict_proba]
 (features_testset')
\end{verbatim}

Anschließend wird die SVM mit den Bildern trainiert und ist somit einsatzbereit.Um nun die Genauigkeit der SVM zu testen, erzeugen wir ein Testset, dessen Features ebenfalls über den bestehenden Algorithmus generiert werden. Getestet wird die SVM, indem wir das Testset an die SVM übergeben und sie uns aufgrund der vorhandenen Werte eine Schätzung abgibt, was am wahrscheinlichsten auf dem Bild abgebildet ist. Die Ergebnisse werden dann überprüft und die Genauigkeit wird errechnet.

\begin{dsafigure}
\begin{center}
	\includegraphics[width=0.35\textwidth]{\media Zahlen_input.png}
	\caption{Ausschnitt des generierten Bildes, das die SVM analysieren soll.}
	\label{Zahlen_input}
\end{center}
\end{dsafigure}

\begin{dsafigure}
\begin{center}
	\includegraphics[width=0.35\textwidth]{\media Zahlen_hogfeat.png}
	\caption{Feature-Bild des generierten Bildes.}
	\label{Zahlen_hogfeat}
\end{center}
\end{dsafigure}

Daraufhin erzeugen wir ein Bild mit 100 Zahlen Abb. \ref{Zahlen_input}, die jeweils in einer $38 \times 38$ Pixelzelle zufällig verteilt werden. Dieses Bild wird in eine Matrix umgewandelt und ein Algorithmus iteriert über die Matrix, sodass die SVM bestimmen kann, welche 100 Zahlen abgebildet sind.




\section{Zeichenerkennung mit neuronalen Netzen}
\authors{Luca Bohn, Lukas Lück, Lukas Kamm}
Nun betrachten wir neuronale Netze im Zusammenhang mit Schrifterkennung. Convolutional Neural Networks eignen sich besonders für die Verarbeitung von Bildern. Eine \emph{Convolution} ist im konkreten Anwendungsfall mit Bildern die Zusammenfassung der Pixel in einem kleinen Bereich, die in ein neues Bild gespeichert werden. Dadurch können beispielsweise kleine Rotationen des Objekts auf einem Foto, das erkannt werden soll, kompensiert werden. Außerdem lassen sich im CNN \emph{Feature Maps} vielseitig anwenden (näheres in Abschnitt \ref{ml:cnn}).
Ein funktionsfähiges neuronales Netz zu implementieren, bedeutet einen großen Zeitaufwand und viel Feinoptimierung. Da uns diese Zeit nicht zur Verfügung steht und wir ein größeres Interesse an der Anwendung haben, nutzen wir Caffe. Bei Caffe handelt es sich um eine Software des Berkeley Vision and Learning Centers. Caffe bietet eine Art Bausatz für neuronale Netze. Man kann verschiedene Filter auswählen und sich so sein eigenes neuronales Netz erstellen. Caffe ist in der Lage, Bilder sehr schnell auszuwerten, nach Herstellerangaben benötigt es je Bild 4 ms in der Trainingsphase und 1 ms in der Testphase.
Als Datensatz für die Zeichen nutzen wir MNIST. Dies ist ein freier Satz aus 28 $\times$ 28 px großen Bildern mit den arabischen Ziffern in Graustufe. Der Testsatz enthält 60.000, der Trainingssatz 10.000 Grafiken. Diese vorgefertigte Zeichendatenbank eignet sich vor allem für Testumgebungen im Bereich des maschinellen Lernens und das Testen von Algorithmen neuronaler Netze. Aufgrund dessen ist sie optimal für beide Schrifterkennungs-Teams geeignet.
\begin{dsafigure}
\begin{center}
	\includegraphics[width=0.40\textwidth]{\media NNFilterMNIST.png}
	\caption{Filter (sog. Feature Maps), spezialisiert für MNIST}
	\label{NNFilterMNIST}
\end{center}
\end{dsafigure}

\subsection{Training}
Voraussetzung für das maschinelle Lernen mit neuronalen Netzen ist das Anlernen mit Hilfe von Trainingsdaten. Hierzu trainieren wir Caffe mit allen 60.000 Grafiken aus dem MNIST-Datensatz.

\subsection{Testen}
Unser Ziel ist es, aus einem Bild mit Handschriftcharakter arabische Ziffern richtig zu erkennen. Hierzu erzeugen wir ein Bild mit $10 \times 10$ zufällig gewählten Ziffern, die jeweils um wenige Pixel von ihrer Zentrumsposition abweichen, um den Handschriftcharakter zu simulieren (vgl. Ausschnitt in Abb. \ref{Zahlen_input}). Anschließend iterieren wir pixelweise, ebenso wie die SVM-Projektgruppe, über das Bild und extrahieren wieder 28 $\times$ 28 px große Grafiken und übermitteln sie an das neuronale Netz.
\section{Vergleich SVM und neuronales Netzwerk}

\subsection{Trainingsdauer}

Allgemein sind SVMs schneller zu trainieren als neuronale Netze, da SVMs nur eine Schicht haben während neuronale Netze aus mehreren Schichten bestehen, in denen die Informationen verarbeitet werden, und deshalb natürlich länger zum Berechnen brauchen. In unserem Projekt waren allerdings anfangs die neuronalen Netze wesentlich schneller, was aber an der schlechteren Implementierung der SVM lag und sie dadurch nicht optimal genutzt wurde. Auch mussten wir bei der SVM zuerst die Features sämtlicher Bilder berechnen, bevor wir sie trainieren konnten. Dies war bei dem neuronalen Netz nicht notwendig, weshalb es insgesamt wesentlich schneller war, denn dieser Vorgang dauert am längsten.

\subsection{Genauigkeit}
Im direkten Vergleich der SVM und des neuronalen Netzes 
\section{Aufgaben Wahrscheinlichkeitstheorie}
\subsection{Übung 1}
Gezeigt werden soll, dass die Varainz:\\
 $\mathbb{E} [X^2] - (\mathbb{E} [X])^2$ auch als $\mathbb{E} [(X- \mathbb{E} [X] )^2 ]$ geschrieben werden kann.	

\begin{align*}
&\mathbb{E} [(X- \mathbb{E} [X] )^2 ] \\
&\quad= \mathbb{E} [X^2 - 2X \mathbb{E} [X] + \mathbb{E} [X]^2 \\
&\quad= \mathbb{E} [X^2] - \mathbb{E} [2X]  \cdot \mathbb{E} [X] + \mathbb{E} [\mathbb{E} [X] \cdot \mathbb{E} [X]]\\
&\quad= \mathbb{E} [X^2] - 2\mathbb{E} [X] \cdot \mathbb{E} [X] +  \mathbb{E} [\mathbb{E} [X] \cdot \mathbb{E} [X] \cdot 1]\\
&\quad= \mathbb{E} [X^2] - 2(\mathbb{E} [X])^2 + (\mathbb{E} [X])^2 \cdot \mathbb{E} [1]\\
&\quad= \mathbb{E} [X^2] - (\mathbb{E} [X])^2
\end{align*}


\subsection{Aufgabe Nr. 2}
Aufgabenteil 1:\\
Zu zeigen ist, dass eine Dichtefunktion $\mathbb{P}(\{X\in A\}):=\int_{x\in A}f(x)dx$ eine Wahrscheinlichkeitsverteilung ist und infolge dessen alle nötigen Axiome erfüllt.\\

\begin{enumerate}
\item Aus der Definition einer Dichtefunktion ergibt sich, dass $f(x)\geq 0$ woraus folgt, dass \begin{equation*}\int_{x\in A}f(x)dx\geq 0 \end{equation*}
\item Aus der Definition einer Dichtefunktion ergibt sich, dass \begin{equation*}\int_{-\infty}^{+\infty}f(x)dx= 1 \end{equation*} woraus folgt, dass $\mathbb{P}(A)=1$
\item Für alle für $i,j\in 1,...,n$ disjunkte $E_i,E_j$ gilt:
\begin{align*}
\mathbb{P}(\bigcup_{i=1}^{n} E_i)&=\int_{x\in \bigcup_{i=1}^{n}E_i}{f(x)dx}\\
&=\sum_{i=1}^{n}{\int_{x\in E_i}{f(x)dx}}\\
&=\sum_{i=1}^{n}{\mathbb{P}(E_i)}
\end{align*}
\end{enumerate}
Aufgabenteil 2:\\
Wir sollen zeigen, dass, wenn eine Dichtefunktion faktorisiert wird die daraus entstehenden Zufallsvektoren unabhängig sind.\\
Gegeben ist:
\begin{enumerate}
\item \begin{align*}&\mathbb{P}(\{(X_{1}, ... , X_{p}) \in (A_{1} \times ... \times A_{p})\} )\\ &= \quad \int_{X \in A}{f(x)dx}\end{align*}
\item \begin{equation*}f(x) = \prod_{i=1}^{p}{f(x_{i})}\end{equation*}
\end{enumerate}
Daraus folgt:
\begin{align*}
&\mathbb{P}(\{(X_{1}, ... , X_{p}) \in (A_{1} \times ... \times A_{p})\} ) = \int_{X \in A}{f(x)dx}\\
&= \int_{X \in A}{\prod_{i=1}^{p}{f(x_{i})}} = \prod_{i=1}^{p}{\int_{X_{i} \in A_{i}}{f(x_{i})}}\\
&= \prod_{i=1}^{p}{\mathbb{P} (\{X_{i} \in A_{i}\})}
\end{align*}

\subsection{Aufgabe Nr. 3}
Wie lässt sich die Wahrscheinlichkeit darstellen dass bei einer Abfolge von $n$ unabhängigen Ja- oder nein Experimenten $k$-mal Ja herauskommt?\\
$\mathbb{P}(\{X=K\})$ bezeichnet die Wahrscheinlichkeit, dass nach $n$-Ja, Nein Entscheidungen genau $k$-mal "`Ja"' und $n-k$-mal "`Nein"' geantwortet wurde.\\
Die Wahrscheinlichkeit für einen Durchlauf bestehend aus $n$-Entscheidungen, welche das oben genannte kriterium erfüllen und eine fest definierte Reinfolge haben lässt sich wiefolgt beschreiben:\\
$p^k\cdot (1-p)^{n-k}$, falls $p$ die Wahrscheinlichkeit für das eintreten eines "`Ja"'- Ereignisses beschreibt. Da alle möglichen Kombinationen, von gegebenen "`Ja"', "`Nein"' Ereignissen durch $\dbinom{n}{k}$ gegeben ist gilt für $\mathbb{P}(\{X=K\})=\dbinom{n}{k}p^k\cdot (1-p)^{n-k}$
%%% \documentclass{newlayout}
%% %Bitte hier den enstprechenden Ort einsetzen z.B. Braunschweig und die Akademienummer
%% \Akademie{Braunschweig}{2015}{1}

%% \usepackage[english]{babel}
%% \usepackage{misc}
%% \usepackage{multicol}

%% \usepackage[utf8]{inputenc}

%% %\usepackage{amsmath}%wird automatisch durch newlayout.cls geladen
%% \usepackage{amsfonts}

%% \usepackage{blindtext}

%% \usepackage{url}
%% \def\UrlBreaks{\do\a\do\b\do\c\do\d\do\e\do\f\do\g\do\h\do\i\do\j\do\k\do\l%
%% \do\m\do\n\do\o\do\p\do\q\do\r\do\s\do\t\do\u\do\v\do\w\do\x\do\y\do\z\do\0%
%% \do\1\do\2\do\3\do\4\do\5\do\6\do\7\do\8\do\9\do\-\do\_\do\/\do\%}
%% \urlstyle{same}

%% % hinzugef�gt, um Fehler 'pdfTeX error (font expansion): auto expansion is only possible with scalable' zu vermeiden
%% \usepackage{lmodern}
%% \setkomafont{descriptionlabel}{\normalfont\bfseries}
%% \addtokomafont{paragraph}{\normalfont}
%% \usepackage{footnote}
%% \usepackage[flushmargin,hang,ragged]{footmisc}
%% \deffootnote{1em}{1em}{%
%% \textsuperscript{\thefootnotemark\ }
%% }
%% %\setlength{\abovedisplayskip}{5pt}
%% %\setlength{\belowdisplayskip}{5pt}


%% %%%%%Mathe-Definitionen
%% \newtheorem{Def}{Definition}
%% \newtheorem{Sat}{Satz}
%% \newtheorem{Bew}{Beweis}
%% \newtheorem{Thm}{Theorem}

%% \setlength\abovedisplayshortskip{0pt}
%% \setlength\belowdisplayshortskip{0pt}
%% \setlength\abovedisplayskip{3pt}
%% \setlength\belowdisplayskip{3pt}
%% %%%%Ende Mathe-Definitionen

%% \begin{document}

%%  %   \input{titel}
%%  \setcounter{page}{3}

%% \setcounter{tocdepth}{1}
%%  \tableofcontents

%%    \setcounter{secnumdepth}{1}


%% \setcounter{page}{1}
%% \setcounter{chapter}{0}
\section{Dating mit Julia}
\authors{Jan Fritz,  Vivien Thi, Franziska Wilfinger}
Bei der Auswahl des Projekts war uns schnell bewusst, dass wir etwas programmieren wollen, um die im Kurs erlernten mathematischen Theorien anzuwenden. Schon w\"ahrend der Vorlesung zum Thema SVM wurde ersichtlich, welche M\"oglichkeiten ein Programm, das auf diesem Konzept beruht, bietet und so entschieden wir uns, ein solches Programm als Projekt zu schreiben. Als Anwendungungsgebiet w\"ahlten wir die Partnersuche. Das im Folgenden vorgestellte Programm ist in der Lage, auszugeben, ob eine Person (\enquote{Testkandidat}) eine andere Person (\enquote{Proband}) \"au\ss erlich anspricht, sofern diese Person zuvor Portr\"atbilder bewertet hat.

\paragraph{Beschreibung.}
Das Program ist so aufgebaut, dass der Proband zuerst mehr als zweihundert Personen mit \enquote{hot} oder \enquote{not} bewertet. Dabei erfahren wir etwas \"uber die Anspr\"uche beziehungsweise Vorlieben der Person. Anschlie\ss end wird eine neue Person in das Programm eingegeben. Die Merkmale dieser neuen Person werden mit denen der Trainingsdaten verglichen.
Im Anschluss gibt der Computer aus, ob diese neue Person den Vorlieben des Probanden entspricht.

\paragraph{Datensatz.}
Als Erstes ben\"otigten wir einen gro\ss en Datensatz, also eine gro\ss e Menge an Bildern, die sp\"ater zum Trainieren des Programs auf die Vorlieben einer Person genutzt werden sollten. Hierbei gilt, dass ein größerer Datensatz zu besseren Ergebnissen führt. Da das Program f\"ur ein Klientel von Frauen im Alter von etwa 15--25 Jahren geschrieben wurde, suchten wir 212 Bilder von Männern im Alter von 15--50 Jahren aus, wobei die große Mehrheit sich im Altersbereich zwischen 20 und 30 Jahren befindet. Bei der Auswahl des Trainingsdatensatzes wurde darauf geachtet, dass die H\"aufigkeit der Merkmale im Datensatz die H\"aufigkeit der Merkmale in der Weltbev\"olkerung widerspiegelt. So sind Personen aus allen Kontinenten vertreten, mehr Braunhaarige als Blonde, mehr Blonde als Rothaarige, und so weiter. Diese Bilder wurden nach $13$ Features klassifiziert. Diese sind das Alter, die Ethnie, die Gesichtsform, die Hautfarbe, die Haarfarbe, die Frisur, die Gesichtsbehaarung, die Augenbrauen, die Augenfarbe, die Lippen, die Nase und die eventuell vorhandene Brille und der K\"orperschmuck. Im Anhang ist diese Klassifikation am Beispiel einiger Kandidaten gezeigt.

\paragraph{Featurization.}
Als Featurization bezeichnet man die Konvertierung von Merkmalen (zum Beispiel Augenfarbe) in eine Repr\"asentation, die vom Computer verarbeitet werden kann. Wir haben die Merkmale in Einheitsvektoren (\enquote{one hot} Vektoren) konvertiert, damit kein Feature bevorzugt wird. F\"ur die Augenfarbe haben wir zum Beispiel
\begin{itemize}
  \item braun entspricht $(1, 0, 0)$
  \item blau entspricht $(0, 1, 0)$
  \item gr\"un entspricht $(0, 0, 1)$
\end{itemize}
Die Komponente, die der jeweiligen Merkmalsauspr\"agung (hier braun, blau und gr\"un) entspricht, ist hier eins, alle anderen null. Man kann auch zwei Auspr\"agungen eines Merkmals aufzeigen, indem man beispielsweise 0.5 und 0.5 in die jeweiligen Komponenten schreibt. Wichtig ist hierbei, dass die Einträge des Vektors sich zu eins summieren. Am Schluss h\"angt man die Vektoren aller Merkmale hintereinander und bekommt so einen neuen Vektor, der die jeweilige Person repr\"asentiert.



\paragraph{Programm.}
Die Features der Trainingsdaten speichern wir in einer Matrix $X$ ab und die labels \enquote{hot} or \enquote{not} in einem Vektor $y$ mit Komponenten $+1$ und $-1$. Anhand dieser Daten trainieren wir dann eine SVM. Zur Testzeit wird eine Person aus dem Pool der Testpersonen ausgew\"ahlt. Wir planten, die Eigenschaften der Teilnehmer der DSA anhand von Bildern zu bestimmen und dann als Testdatensatz zu verwenden. Leider konnte aufgrund von Zeitmangel das Projekt nicht fertiggestellt werden.







% \end{document}








%\section*{Literaturverzeichnis}

\begin{thebibliography}{99}

\bibitem{Bengio-et-al-2015-Book}
Yoshua Bengio, Ian~J. Goodfellow, and Aaron Courville.
\newblock Deep learning.
\newblock Book in preparation for MIT Press
  (\url{http://www.iro.umontreal.ca/~bengioy/dlbook)}, 2015.

\bibitem{Boyd:2004:CO:993483}
Stephen Boyd and Lieven Vandenberghe.
\newblock {\em Convex Optimization}.
\newblock Cambridge University Press, New York, NY, USA, 2004.

\bibitem{Grant06disciplinedconvex}
Michael Grant, Stephen Boyd, and Yinyu Ye.
\newblock Disciplined convex programming.
\newblock In {\em Global Optimization: From Theory to Implementation, Nonconvex
  Optimization and Its Application Series}, pages 155--210. Springer, 2006.

\bibitem{NIPS2012_4824}
Alex Krizhevsky, Ilya Sutskever, and Geoffrey~E. Hinton.
\newblock Imagenet classification with deep convolutional neural networks.
\newblock In F.~Pereira, C.J.C. Burges, L.~Bottou, and K.Q. Weinberger,
  editors, {\em Advances in Neural Information Processing Systems 25}, pages
  1097--1105. Curran Associates, Inc., 2012.

\bibitem{LectureA}
Andrew Ng.
\newblock {\em Machine Learning}.
\newblock Stanford University and Coursera
  (\url{http://www.coursera.org/course/ml}).

\bibitem{nielsen}
Michael Nielsen.
\newblock {\em Neural Networks and Deep Learning}.
\newblock \url{http://neuralnetworksanddeeplearning.com/}, 2014.

\bibitem{heavyball}
Ben Recht.
\newblock {\em Optimization}.
\newblock University of Wisconsin-Madison
  (\url{http://pages.cs.wisc.edu/~brecht/cs726docs/HeavyBallLinear.pdf}).

\bibitem{strang09}
Gilbert Strang.
\newblock {\em Introduction to Linear Algebra}.
\newblock Wellesley-Cambridge Press, Wellesley, MA, 2009.

\bibitem{Moritz_Yang15}
Philipp~Moritz und Fanny~Yang.
\newblock A gentle introduction of mathematical machine learning.
\newblock Unpublished lecture notes for summer course "Learning from data", 8
  2015.

\end{thebibliography}


%\nocite{*} % alle referenzen anzeigen, sogar wenn sie nicht im text zitiert sind
%\bibliography{lit}{}
%\bibliographystyle{plain}

\nocite{*}
\bibliographystyle{plain}
\bibliography{lit}
\end{document}
