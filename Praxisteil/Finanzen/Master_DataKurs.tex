\documentclass{newlayout}
%Bitte hier den enstprechenden Ort einsetzen z.B. Braunschweig und die Akademienummer
\Akademie{Braunschweig}{2015}{1}

\usepackage[english]{babel}
\usepackage{misc}
\usepackage{multicol}
\usepackage[utf8]{inputenc}

%\usepackage{amsmath}%wird automatisch durch newlayout.cls geladen
\usepackage{amsfonts}

\usepackage{blindtext}

\usepackage{url}
\def\UrlBreaks{\do\a\do\b\do\c\do\d\do\e\do\f\do\g\do\h\do\i\do\j\do\k\do\l%
\do\m\do\n\do\o\do\p\do\q\do\r\do\s\do\t\do\u\do\v\do\w\do\x\do\y\do\z\do\0%
\do\1\do\2\do\3\do\4\do\5\do\6\do\7\do\8\do\9\do\-\do\_\do\/\do\%}
\urlstyle{same}

\usepackage{lmodern}
\setkomafont{descriptionlabel}{\normalfont\bfseries}
\addtokomafont{paragraph}{\normalfont}
\usepackage{footnote}
\usepackage[flushmargin,hang,ragged]{footmisc}
\deffootnote{1em}{1em}{%
\textsuperscript{\thefootnotemark\ }
}
%\setlength{\abovedisplayskip}{5pt}
%\setlength{\belowdisplayskip}{5pt}

\newcommand{\media}{../../media/}
\newcommand{\norm}[1]{\left\lVert#1\right\rVert}

%%%%%Mathe-Definitionen
\newtheorem{Def}{Definition}
\newtheorem{Sat}{Satz}
\newtheorem{Bew}{Beweis}
\newtheorem{Thm}{Theorem}

\setlength\abovedisplayshortskip{0pt}
\setlength\belowdisplayshortskip{0pt}
\setlength\abovedisplayskip{3pt}
\setlength\belowdisplayskip{3pt}
%%%%Ende Mathe-Definitionen

\begin{document}

 \setcounter{page}{3}

\setcounter{tocdepth}{1}
 \tableofcontents

   \setcounter{secnumdepth}{1}


\setcounter{page}{7}
\setcounter{chapter}{0}

%Angabe, bis zu welcher Stufe die sections im Text nummeriert werden sollen.
      \settocdepth{2}



\course{2}{Machine Learning} 
\begin{coursetitle}
  \centerline{Machine Learning} 
  \bigskip
  \Large \centerline{SVMs und Neuronale Netzwerke}
  \bigskip
  \bigskip
\end{coursetitle}

\section{\enquote{Earning from Data} -- Finanzanalyse}
\author{Dennis Kempf, Moritz Hollenberg, Patrice Becker}

Die Finanzanalyse versucht, künftige Kursverläufe vorherzusagen, um Gewinne im Finanzhandel zu erzielen. In unserem Projekt versuchen wir dies durch den Einsatz von \emph{Machine Learning} zu erreichen.

\subsection{Formen der Finanzanalyse}
\author{Dennis Kempf}

Die drei grundlegenden Formen der Finanzanalyse bilden die \emph{Fundamentalanalyse}, \emph{Technische Analyse} und die \emph{Sentimentanalyse}. 

Die Fundamental- und die Technische Analyse gehen davon aus, dass nicht alle verfügbaren Informationen in den Kursen verarbeitet sind. Es sei somit möglich, schneller als andere Marktteilnehmer zu agieren und daraus Profite zu erzielen.

\subsubsection{Fundamentalanalyse}
\author{Dennis Kempf}

Bei der Fundamentalanalyse werden sogenannte \emph{Fundamentaldaten} analysiert, um den wirklichen Wert eines Finanzproduktes zu bestimmen. Zu diesen Daten zählen unter anderem
\begin{itemize}
	\item Kurs-Gewinn-Verhältnis,
	\item Gesamtkapitalrendite,
	\item Eigenkapitalquote,
	\item Bruttoinlandsprodukt und
	\item Einzelhandelsverkäufe.
\end{itemize}

\subsubsection{Technische Analyse}
\author{Dennis Kempf}
\label{sssec:TechnischeAnalyse}

Im Gegensatz zur Fundamentalanalyse werden bei der Technischen Analyse \emph{Charts}, das heißt Abbildungen von Kursverläufen analysiert. Dazu werden sowohl reine Preisverläufe, als auch von diesen abgeleitete \emph{Indikatoren} berücksichtigt. Zu diesen Indikatoren gehören unter anderem:
\begin{itemize}
	\item Moving Average,
	\item Bollinger Bands,
	\item Stochastic Oscillator,
	\item Relative Strength Index und
	\item Fractals.
\end{itemize}

\begin{dsafigure}
	\begin{center}
		\includegraphics[width=0.5\textwidth]
		{\media Finances_Chart.png}
		\caption{Ein \emph{Chart} (eine Kerze $\hat =$ eine Stunde) des EUR/USD Währungspaares mit den genannten Beispielindikatoren (siehe  \ref{sssec:TechnischeAnalyse})}
		\label{fig:BeispielChart}
	\end{center}
\end{dsafigure}

\subsubsection{Sentimentanalyse}
\author{Dennis Kempf}

Die Sentimentanalyse befasst sich mit der Stimmung von Investoren, um daraus zu schließen, ob eine \emph{bullische Phase} (steigender Trend) oder eine \emph{bärische Phase} (fallender Trend) bevorsteht. Dazu können Mittel wie etwa \emph{Meinungsumfragen} oder die Analyse von \emph{Börsenbriefen} eingesetzt werden.

\subsection{Einsatz von Machine Learning}
\author{Dennis Kempf}

Unser Projekt befasst sich ausschließlich mit der Technischen Analyse (siehe \ref{sssec:TechnischeAnalyse}). Der klassische Ansatz würde die Aufgabe der Analyse einem Menschen oder einem selbstgeschriebenen Algorithmus überlassen. 

Wir hingegen versuchen diese Aufgabe auf ein \emph{neuronales Netzwerk} zu übertragen. Hierbei spielt die Auswahl der Features und der verwendeten Parameter wie z.\,B. Schichtenanzahl des Netzwerks eine sehr wichtige Rolle.

\subsubsection{Verwendete Features}
\author{Dennis Kempf}
\label{sssec:Features}

Wir haben uns für die Verwendung des \emph{Stochastic Oscillator} und des \emph{Relative Strength Index} (jeweils mit Standardparametern) als Features entschieden. Es handelt sich dabei um zwei der bekanntesten Indikatoren für die technische Analyse.

\paragraph{Stochastic Oscillator.}

Der Stochastic Oscillator ist ein von Dr. George Lane in den 1950er Jahren entwickelter Indikator. Dieser versucht, bevorstehende Trendwenden anhand von Unterstützungs- und Widerstandslinien zu erkennen. Er setzt sich aus zwei Elementen zusammen: Der \%K- und der \%D-Kurve, wobei \%D ein gleitender Durchschnitt von \%K ist. 

Wir betrachten jedoch nur die \%K Kurve für unsere Features. Diese lässt sich wie folgt berechnen:
\begin{align*}
	\text{\%K}&=100\cdot \frac{P-T_5}{H_5-T_5}\quad \text{mit}\\
	P&=\text{Aktueller Schlusskurs}\\
	T_5&=\text{Tief der letzten fünf Perioden}\\
	H_5&=\text{Hoch der letzten fünf Perioden}
\end{align*}

\begin{dsafigure}
	\begin{center}
		\includegraphics[width=0.5\textwidth]
		{\media Finances_Stochastic}
		\caption{Beispielchart des EUR/USD mit dem Stochastic Oscillator (Standardparameter) im unteren Fenster (Blau: \%K-Kurve, Rot: \%D-Kurve)}
		\label{fig:BeispielChartStochastic}
	\end{center}
\end{dsafigure}

\paragraph{Relative Strength Index.} Der Relative Strength Index (\enquote{Relative Stärke Index}) ist ein von J. Welles Wilder Ende der 1970er Jahre entwickelter Indikator. Wie auch der Stochastic Oscillator versucht dieser, bevorstehende Trendwenden zu erkennen. Hierzu analysiert dieser Indikator Kursbewegungen entsprechend ihrer Richtung und Größe.

\begin{dsafigure}
	\begin{center}
		\includegraphics[width=0.5\textwidth]
		{\media Finances_RSI}
		\caption{Beispielchart des EUR/USD mit dem Relative Strength Index (Standardparameter) im unteren Fenster}
		\label{fig:BeispielChartRSI}
	\end{center}
\end{dsafigure}

\subsubsection{Verwendete Datensätze}
\author{Dennis Kempf}
\label{sssec:Datensaetze}

Für unsere Datensätze haben wir jeweils 65535 gelabelte Featurevektoren für verschiedene Währungspaare gesammelt. Jeder Featurevektor setzt sich aus jeweils vier Werten des Stochastic Oscillators und des Relative Strength Indexes zusammen. Somit sind diese insgesamt 8 Elemente lang. Bei unseren Labels unterscheiden wir lediglich, ob der Kurs in der auf die Features folgenden Stunde gestiegen oder gefallen ist.

\begin{dsafigure}
	\begin{center}
		\includegraphics[width=0.5\textwidth]
		{\media Finances_Datensatz}
		\caption{Schematischer Aufbau unserer Featurevektoren}
		\label{fig:DatensatzSchema}
	\end{center}
\end{dsafigure}

Insgesamt haben wir jeweils mehr als zehn Jahre an Features und Labels für den Währungspaare CHF/JPY, EUR/AUD, EUR/CAD, EUR/CHF, EUR/USD und GBP/JPY zusammengestellt.



%\nocite{*} % alle referenzen anzeigen, sogar wenn sie nicht im text zitiert sind
%\bibliography{lit}{}
\bibliographystyle{plain}
\end{document}
