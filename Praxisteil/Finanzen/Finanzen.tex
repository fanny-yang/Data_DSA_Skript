\section{Finanzanalyse}
\author{Dennis Kempf, Moritz Hollenberg, Patrice Becker}

Die Finanzanalyse versucht künftige Kursverläufe vorherzusagen, um Gewinne im Finanzhandel zu erzielen.

\subsection{Formen der Finanzanalyse}
\author{Dennis Kempf}

Die drei grundlegenden Formen der Finanzanalyse bilden die \emph{Fundamentalanalyse}, \emph{Technische Analyse} und die \emph{Sentimentanalyse}. 

Die Fundamental- und die Technische Analyse gehen davon aus, dass nicht alle verfügbaren Informationen in den Kursen verarbeitet sind. Es sei somit möglich, schneller als andere Marktteilnehmer zu agieren und daraus Profite zu erzielen.

\subsubsection{Fundamentalanalyse}
\author{Dennis Kempf}

Bei der Fundamentalanalyse werden sogenannte \emph{Fundamentaldaten} analysiert, um den wirklichen Wert eines Finanzproduktes zu bestimmen. Zu diesen Daten zählen unter anderem:\\ 

\begin{itemize}
	\item Kurs-Gewinn-Verhältnis
	\item Gesamtkapitalrendite
	\item Eigenkapitalquote
	\item Bruttoinlandsprodukt
	\item Einzelhandelsverkäufe
\end{itemize}

\subsubsection{Technische Analyse}
\author{Dennis Kempf}
\label{sssec:TechnischeAnalyse}

Im Gegensatz zur Fundamentalanalyse werden bei der Technischen Analyse \emph{Charts}, d.h. Abbildungen von Kursverläufen analysiert. Dazu werden sowohl reine Preisverläufe, als auch von diesen abgeleitete \emph{Indikatoren} berücksichtigt. Zu diesen Indikatoren gehören unter anderem:\\

\begin{itemize}
	\item Moving Average
	\item Bollinger Bands
	\item Stochastic Oscillator
	\item Relative Strength Index
	\item Fractals
\end{itemize}

\begin{dsafigure}
	\begin{center}
		\includegraphics[width=0.5\textwidth]
		{\media EURUSDH1.png}
		\caption{Ein \emph{Chart} (eine Kerze $\hat =$ eine Stunde) des EUR/USD Währungspaares mit den genannten Beispielindikatoren aus \ref{sssec:TechnischeAnalyse}}
		\label{fig:Beispielchart}
	\end{center}
\end{dsafigure}

\subsubsection{Sentimentanalyse}
\author{Dennis Kempf}

Die Sentimentanalyse befasst sich mit der Stimmung von Investoren, um daraus zu schließen, ob eine \emph{bullische} Phase (steigender Trend) oder eine \emph{bärische} Phase (fallender Trend) bevorsteht. Dazu können Mittel wie etwa \emph{Meinungsumfragen} oder die Analyse von \emph{Börsenbriefen} eingesetzt werden.

\subsection{Einsatz von Machine Learning}