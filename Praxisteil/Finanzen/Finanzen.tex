\section{>>Earning from Data<< --- Finanzanalyse}
\author{Dennis Kempf, Moritz Hollenberg, Patrice Becker}

Die Finanzanalyse versucht künftige Kursverläufe vorherzusagen, um Gewinne im Finanzhandel zu erzielen. In unserem Projekt versuchen wir dies durch den Einsatz von \emph{Machine Learning} zu erreichen.

\subsection{Formen der Finanzanalyse}
\author{Dennis Kempf}

Die drei grundlegenden Formen der Finanzanalyse bilden die \emph{Fundamentalanalyse}, \emph{Technische Analyse} und die \emph{Sentimentanalyse}. 

Die Fundamental- und die Technische Analyse gehen davon aus, dass nicht alle verfügbaren Informationen in den Kursen verarbeitet sind. Es sei somit möglich, schneller als andere Marktteilnehmer zu agieren und daraus Profite zu erzielen.

\subsubsection{Fundamentalanalyse}
\author{Dennis Kempf}

Bei der Fundamentalanalyse werden sogenannte \emph{Fundamentaldaten} analysiert, um den wirklichen Wert eines Finanzproduktes zu bestimmen. Zu diesen Daten zählen unter anderem:\\ 

\begin{itemize}
	\item Kurs-Gewinn-Verhältnis
	\item Gesamtkapitalrendite
	\item Eigenkapitalquote
	\item Bruttoinlandsprodukt
	\item Einzelhandelsverkäufe
\end{itemize}

\subsubsection{Technische Analyse}
\author{Dennis Kempf}
\label{sssec:TechnischeAnalyse}

Im Gegensatz zur Fundamentalanalyse werden bei der Technischen Analyse \emph{Charts}, d.h. Abbildungen von Kursverläufen analysiert. Dazu werden sowohl reine Preisverläufe, als auch von diesen abgeleitete \emph{Indikatoren} berücksichtigt. Zu diesen Indikatoren gehören unter anderem:\\

\begin{itemize}
	\item Moving Average
	\item Bollinger Bands
	\item Stochastic Oscillator
	\item Relative Strength Index
	\item Fractals
\end{itemize}

\begin{dsafigure}
	\begin{center}
		\includegraphics[width=0.5\textwidth]
		{\media Finances_Chart.png}
		\caption{Ein \emph{Chart} (eine Kerze $\hat =$ eine Stunde) des EUR/USD Währungspaares mit den genannten Beispielindikatoren (siehe  \ref{sssec:TechnischeAnalyse})}
		\label{fig:Beispielchart}
	\end{center}
\end{dsafigure}

\subsubsection{Sentimentanalyse}
\author{Dennis Kempf}

Die Sentimentanalyse befasst sich mit der Stimmung von Investoren, um daraus zu schließen, ob eine \emph{bullische Phase} (steigender Trend) oder eine \emph{bärische Phase} (fallender Trend) bevorsteht. Dazu können Mittel wie etwa \emph{Meinungsumfragen} oder die Analyse von \emph{Börsenbriefen} eingesetzt werden.

\subsection{Einsatz von Machine Learning}
\author{Dennis Kempf}

Unser Projekt befasst sich ausschließlich mit der Technischen Analyse (siehe \ref{sssec:TechnischeAnalyse}). Der klassische Ansatz würde die Aufgabe der Analyse einem Menschen oder einem selbstgeschriebenen Algorithmus überlassen. 

Wir hingegen versuchen diese Aufgabe auf ein \emph{neuronales Netzwerk} zu übertragen. Hierbei spielt die Auswahl der Features und der verwendeten Parameter wie z.B. Schichtenanzahl des Netzwerks eine sehr wichtige Rolle.

\subsubsection{Verwendete Features}
\author{Dennis Kempf}
\label{sssec:Features}

Wir haben uns für die Verwendung des \emph{Stochastic Oscillator} und des \emph{Relative Strength Index} (jeweils mit Standardparametern) als Features entschieden. Jeder Featurevektor setzt sich aus jeweils vier Werten dieser Indikatoren zusammen. Somit sind diese insgesamt 8 Elemente lang. Zusammengefasst ergeben bei uns 65536 Featurevektoren inklusive Labels einen Datensatz.
Unsere Datensätze beinhalten also Informationen aus mehr als zehn Jahren.

\begin{dsafigure}
	\begin{center}
		\includegraphics[width=0.5\textwidth]
		{\media Finances_Datensatz}
		\caption{Schematischer Aufbau unserer Featurevektoren (siehe \ref{sssec:Features})}
		\label{fig:DatensatzSchema}
	\end{center}
\end{dsafigure}

\paragraph{Stochastic Oscillator}

Lorem ipsum...

\begin{dsafigure}
	\begin{center}
		\includegraphics[width=0.5\textwidth]
		{\media Finances_Stochastic}
		\caption{Beispielchart des EUR/USD mit dem Stochastic Oscillator (Standardparameter) im unteren Fenster}
		\label{fig:DatensatzSchema}
	\end{center}
\end{dsafigure}

\paragraph{Relative Strength Index}

Lorem ipsum...

\begin{dsafigure}
	\begin{center}
		\includegraphics[width=0.5\textwidth]
		{\media Finances_RSI}
		\caption{Beispielchart des EUR/USD mit dem Relative Strength Index (Standardparameter) im unteren Fenster}
		\label{fig:DatensatzSchema}
	\end{center}
\end{dsafigure}

\newpage
Lorem ipsum...