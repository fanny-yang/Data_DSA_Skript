\subsection{Aufgaben}
\subsubsection{Varianz}
Wir zeigen, dass die Varianz
 $\mathbb{E} [X^2] - \mathbb{E} [X]^2$ auch als $\mathbb{E} [(X- \mathbb{E} [X] )^2 ]$ geschrieben werden kann.	
\begin{align*}
&\mathbb{E} [(X- \mathbb{E} [X] )^2 ] \\
&\quad= \mathbb{E} [X^2 - 2X \mathbb{E} [X] + \mathbb{E} [X]^2 \\
&\quad= \mathbb{E} [X^2] - \mathbb{E} [2X]  \cdot \mathbb{E} [X] + \mathbb{E} [\mathbb{E} [X] \cdot \mathbb{E} [X]]\\
&\quad= \mathbb{E} [X^2] - 2\mathbb{E} [X] \cdot \mathbb{E} [X] +  \mathbb{E} [\mathbb{E} [X] \cdot \mathbb{E} [X] \cdot 1]\\
&\quad= \mathbb{E} [X^2] - 2(\mathbb{E} [X])^2 + (\mathbb{E} [X])^2 \cdot \mathbb{E} [1]\\
&\quad= \mathbb{E} [X^2] - (\mathbb{E} [X])^2
\end{align*}


\subsubsection{Dichtefunktionen}
Wir zeigen, dass eine Dichtefunktion $f$ mittels
\begin{equation*}
\mathbb{P}(\{X\in A\}):=\int_{x\in A}f(x)dx
\end{equation*}
eine Wahrscheinlichkeitsverteilung definiert und infolgedessen alle nötigen Axiome erfüllt.

\begin{enumerate}
\item Aus der Definition einer Dichtefunktion ergibt sich, dass $f(x)\geq 0$ woraus folgt, dass \begin{equation*}\int_{x\in A}f(x)dx\geq 0. \end{equation*}
\item Aus der Definition einer Dichtefunktion ergibt sich, dass \begin{equation*}\int_{-\infty}^{+\infty}f(x)dx= 1 \end{equation*} woraus folgt, dass $\mathbb{P}(A)=1$.
\item Für alle disjunkten $E_i,E_j$ mit $i,j\in \{1,\dots ,n\}$ gilt
\begin{align*}
\mathbb{P}\left(\bigcup_{i=1}^{n} E_i\right)&=\int_{x\in \bigcup_{i=1}^{n}E_i}{f(x)dx}\\
&=\sum_{i=1}^{n}{\int_{x\in E_i}{f(x)dx}}\\
&=\sum_{i=1}^{n}{\mathbb{P}(E_i)}
\end{align*}
\end{enumerate}
Nun zeigen wir, dass im Falle $f(x) = \prod_{i=1}^p f_i(x_i)$ die daraus entstehenden Zufallsvektoren unabhängig sind. Dies folgt aus
\begin{align*}
&\mathbb{P}(\{(X_{1}, ... , X_{p}) \in (A_{1} \times ... \times A_{p})\} )\\ &= \int_{X \in A}{f(x)dx}\\
&= \int_{X \in A}{\prod_{i=1}^{p}{f(x_{i})}} = \prod_{i=1}^{p}{\int_{X_{i} \in A_{i}}{f(x_{i})}}\\
&= \prod_{i=1}^{p}{\mathbb{P} (\{X_{i} \in A_{i}\})}
\end{align*}

\subsubsection{Binomialverteilung}
Wie lässt sich die Wahrscheinlichkeit darstellen dass bei einer Abfolge von $n$ unabhängigen Experimenten, die je mit Wahrscheinlichkeit $p$ erfolgreich sind, $k$ Erfolge auftreten?

Sei $\mathbb{P}(\{X=k\})$ die Wahrscheinlichkeit, dass nach $n$ Experimenten genau $k$ Erfolge auftreten.
Die Wahrscheinlichkeit für einen Durchlauf bestehend aus $n$-Entscheidungen, von denen $k$ erfolgreich sind, die eine fest definierte Abfolge von \enquote{Erfolg} und \enquote{Misserfolg} haben lässt sich wie folgt beschreiben:
$p^k\cdot (1-p)^{n-k}$, falls $p$ die Wahrscheinlichkeit für das eintreten eines Erfolges beschreibt. Da die Anzahl aller möglichen Abfolgen von Erfolgen und Misserfolgen durch $\dbinom{n}{k}$ gegeben ist gilt
\begin{equation*}
\mathbb{P}(\{X=k\})=\dbinom{n}{k}\cdot p^k\cdot (1-p)^{n-k}
\end{equation*}
