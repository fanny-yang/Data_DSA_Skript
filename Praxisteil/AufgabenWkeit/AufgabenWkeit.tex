\documentclass{newlayout}
%Bitte hier den enstprechenden Ort einsetzen z.B. Braunschweig und die Akademienummer
\Akademie{Braunschweig}{2015}{1}

\usepackage[ngerman]{babel}
\usepackage{misc}
\usepackage{multicol}
\usepackage{inputenc}
% \usepackage[ansinew]{inputenc}
\usepackage[utf8]{inputenc} 
\usepackage{graphicx}
	\usepackage{pgf,tikz}
\usepackage{mathrsfs}
\usetikzlibrary{arrows}

%\usepackage{amsmath}%wird automatisch durch newlayout.cls geladen
\usepackage{amsfonts}

\usepackage{url}
\def\UrlBreaks{\do\a\do\b\do\c\do\d\do\e\do\f\do\g\do\h\do\i\do\j\do\k\do\l%
\do\m\do\n\do\o\do\p\do\q\do\r\do\s\do\t\do\u\do\v\do\w\do\x\do\y\do\z\do\0%
\do\1\do\2\do\3\do\4\do\5\do\6\do\7\do\8\do\9\do\-\do\_\do\/\do\%}
\urlstyle{same}

% hinzugefügt, um Fehler 'pdfTeX error (font expansion): auto expansion is only possible with scalable' zu vermeiden
\usepackage{lmodern}
\setkomafont{descriptionlabel}{\normalfont\bfseries}
\addtokomafont{paragraph}{\normalfont}
\usepackage{footnote}
\usepackage[flushmargin,hang,ragged]{footmisc}
\deffootnote{1em}{1em}{%
\textsuperscript{\thefootnotemark\ }
}
%\setlength{\abovedisplayskip}{5pt}
%\setlength{\belowdisplayskip}{5pt}


%%%%%Mathe-Definitionen
\newtheorem{Def}{Definition}
\newtheorem{Sat}{Satz}
\newtheorem{Bew}{Beweis}

\setlength\abovedisplayshortskip{0pt}
\setlength\belowdisplayshortskip{0pt}
\setlength\abovedisplayskip{3pt}
\setlength\belowdisplayskip{3pt}
%%%%Ende Mathe-Definitionen
\begin{document}

\section{Aufgaben Wahrscheinlichkeitstheorie}
\subsection{Übung 1}
Gezeigt werden soll, dass die Varainz:\\
 $\mathbb{E} [X^2] - (\mathbb{E} [X])^2$ auch als $\mathbb{E} [(X- \mathbb{E} [X] )^2 ]$ geschrieben werden kann.	

\begin{align*}
&\mathbb{E} [(X- \mathbb{E} [X] )^2 ] \\
&\quad= \mathbb{E} [X^2 - 2X \mathbb{E} [X] + \mathbb{E} [X]^2 \\
&\quad= \mathbb{E} [X^2] - \mathbb{E} [2X]  \cdot \mathbb{E} [X] + \mathbb{E} [\mathbb{E} [X] \cdot \mathbb{E} [X]]\\
&\quad= \mathbb{E} [X^2] - 2\mathbb{E} [X] \cdot \mathbb{E} [X] +  \mathbb{E} [\mathbb{E} [X] \cdot \mathbb{E} [X] \cdot 1]\\
&\quad= \mathbb{E} [X^2] - 2(\mathbb{E} [X])^2 + (\mathbb{E} [X])^2 \cdot \mathbb{E} [1]\\
&\quad= \mathbb{E} [X^2] - (\mathbb{E} [X])^2
\end{align*}


\subsection{Aufgabe Nr. 2}
Aufgabenteil 1:\\
Zu zeigen ist, dass eine Dichtefunktion $\mathbb{P}(\{X\in A\}):=\int_{x\in A}f(x)dx$ eine Wahrscheinlichkeitsverteilung ist und infolge dessen alle nötigen Axiome erfüllt.\\

\begin{enumerate}
\item Aus der Definition einer Dichtefunktion ergibt sich, dass $f(x)\geq 0$ woraus folgt, dass \begin{equation*}\int_{x\in A}f(x)dx\geq 0\end{equation*}
\item Aus der Definition einer Dichtefunktion ergibt sich, dass \begin{equation*}\int_{-\infty}^{+\infty}f(x)dx= 1\end{equation*}, woraus folgt, dass $\mathb{P}(A)=1$\\
\item Für alle für $i,j\in 1,...,n$ disjunkte $E_i,E_j$gilt:
\begin{align*}
\mathbb{P}(\bigcup_{i=1}^{n} E_i)&=\int_{x\in \bigcup_{i=1}^{n}E_i}{f(x)dx}\\
&=\sum_{i=1}^{n}{\int_{x\in E_i}{f(x)dx}}\\
&=\sum_{i=1}^{n}{\mathbb{P}(E_i)}
\end{align*}
\end{enumerate}
Aufgabenteil 2:\\
Wir sollen zeigen, dass, wenn eine Dichtefunktion faktorisiert wird die daraus entstehenden Zufallsvektoren unabhängig sind.\\
Gegeben ist:
\begin{enumerate}
\item \begin{align*}&\mathbb{P}(\{(X_{1}, ... , X_{p}) \in (A_{1} \times ... \times A_{p})\} )\\ &= \quad \int_{X \in A}{f(x)dx}\end{align*}
\item \begin{equation*}f(x) = \prod_{i=1}^{p}{f(x_{i})}\end{equation*}
\end{enumerate}
Daraus folgt:
\begin{align*}
&\mathbb{P}(\{(X_{1}, ... , X_{p}) \in (A_{1} \times ... \times A_{p})\} ) = \int_{X \in A}{f(x)dx}\\
&= \int_{X \in A}{\prod_{i=1}^{p}{f(x_{i})}} = \prod_{i=1}^{p}{\int_{X_{i} \in A_{i}}{f(x_{i})}}\\
&= \prod_{i=1}^{p}{\mathbb{P} (\{X_{i} \in A_{i}\})}
\end{align*}

\subsection{Aufgabe Nr. 3}
Wie lässt sich die Wahrscheinlichkeit darstellen dass bei einer Abfolge von $n$ unabhängigen Ja- oder nein Experimenten $k$-mal Ja herauskommt?\\
$\mathbb{P}(\{X=K\})$ bezeichnet die Wahrscheinlichkeit, dass nach $n$-Ja, Nein Entscheidungen genau $k$-mal "`Ja"' und $n-k$-mal "`Nein"' geantwortet wurde.\\
Die Wahrscheinlichkeit für einen Durchlauf bestehend aus $n$-Entscheidungen, welche das oben genannte kriterium erfüllen und eine fest definierte Reinfolge haben lässt sich wiefolgt beschreiben:\\
$p^k\cdot (1-p)^{n-k}$, falls $p$ die Wahrscheinlichkeit für das eintreten eines "`Ja"'- Ereignisses beschreibt. Da alle möglichen Kombinationen, von gegebenen "`Ja"', "`Nein"' Ereignissen durch $\dbinom{n}{k}$ gegeben ist gilt für $\mathbb{P}(\{X=K\})=\dbinom{n}{k}p^k\cdot (1-p)^{n-k}$

\end{document}