\section{Einleitung}

Täglich werden in Deutschland rund 64 Millionen Briefe verschickt. Doch habt ihr euch schon einmal gefragt, wie die ganzen Briefe so schnell verarbeitet werden können? Die Vorstellung von mehreren tausend Mitarbeitern, die sich nur mit der Sortierung der Post beschäftigen, erscheint schon aufgrund des Aufwandes unwahrscheinich. Eine bessere Variante stellt die Benutzung von Schrifterkennungsalgorithmen dar, deren sich die Post seit längerer Zeit bedient. Dabei scannt ein Computer die auf den Umschlägen geschriebenen Adressen ein, wertet diese aus und sortiert sie entsprechend. \\
Zu diesem Thema gab es zwei Projektgruppen, die sich dem Problem auf verschiedene Weise näherten. Eine Gruppe nutzte SVMs, die andere ein neuronales Netzwerk.


\section{Schrifterkennung mit SVMs}

\paragraph{Aufbau des Programms}

Da es beim maschinellen Lernen essenziell ist über ein entsprechend großes Datenset zu verfügen, haben wir unseren Algorithmus zunächst einmal mit 60.000 Bildern ausgestattet. 

\begin{verbatim}
	feature, hog_image = feats.hog(sub_pic,
	 orientations = 8,pixels_per_cell = (4,4),
	 cells_per_block = (1,1),visualise = true)
\end{verbatim}

Im ersten Schritt generieren wir die Features sämtlicher Bilder. Features sind Muster in einem Bild, z.B. Kanten oder Farben. Dabei gibt es mehrere Einstellungsmöglichkeiten, beispielsweise zur Variation der Pixelanzahl in einer Zelle (pixels\_per\_cell) und der Genauigkeit zur Erkennung von Konturen (orientations). Dabei gilt: Je kleiner die Zellen sind und je höher die Genauigkeit ist, desto länger braucht der Algorithmus zum berechnen, aber desto präziser ist das Endergebnis. Allerdings muss man auch beachten, dass die Zellengröße nicht zu klein wird, da in sehr kleinen Zellen die Muster eventuell nicht für das Programm erkennbar sind. 

\begin{dsafigure}
\begin{center}
	\includegraphics[width=0.35\textwidth]{\media Kameramann.png}
	\caption{Schwarz-Weiß Bild eines Kameramannes.}
	\label{Kameramann}
\end{center}
\end{dsafigure}

\begin{dsafigure}
\begin{center}
	\includegraphics[width=0.35\textwidth]{\media Kameramann_hogfeat.png}
	\caption{Feature-Bild des voherigen Bildes.}
	\label{Kameramann_hogfeat}
\end{center}
\end{dsafigure}

\begin{verbatim}
clf[:fit](features_trainset', 
 label_train_data[:])
 ...
prob = clf[:predict_proba]
 (features_testset')
\end{verbatim}

Anschließend wird die SVM mit den Bildern trainiert und ist somit einsatzbereit.Um nun die Genauigkeit der SVM zu testen, erzeugen wir ein Testset, dessen Features ebenfalls über den bestehenden Algorithmus generiert werden. Getestet wird die SVM, indem wir das Testset an die SVM übergeben und sie uns aufgrund der vorhandenen Werte eine Schätzung abgibt, was am wahrscheinlichsten auf dem Bild abgebildet ist. Die Ergebnisse werden dann überprüft und die Genauigkeit wird errechnet.

\begin{dsafigure}
\begin{center}
	\includegraphics[width=0.35\textwidth]{\media Zahlen_input.png}
	\caption{Ausschnitt des generierten Bildes, das die SVM analysieren soll.}
	\label{Zahlen_input}
\end{center}
\end{dsafigure}

\begin{dsafigure}
\begin{center}
	\includegraphics[width=0.35\textwidth]{\media Zahlen_hogfeat.png}
	\caption{Feature-Bild des generierten Bildes.}
	\label{Zahlen_hogfeat}
\end{center}
\end{dsafigure}

Daraufhin erzeugen wir ein Bild mit 100 Zahlen Abb. \ref{Zahlen_input}, die jeweils in einer $38 \times 38$ Pixelzelle zufällig verteilt werden. Dieses Bild wird in eine Matrix umgewandelt und ein Algorithmus iteriert über die Matrix, sodass die SVM bestimmen kann, welche 100 Zahlen abgebildet sind.


