\section{Schrifterkennung mit neuronalen Netzen}

Ein funktionsfähiges neuronales Netz zu implementieren bedeutet einen großen Zeitaufwand und viel Feinoptimierung. Da uns diese Zeit nicht zur Verfügung stand und wir ein größeres Interesse an der Anwendung hatten, nutzten wir Caffe. Bei Caffe handelt es sich um eine Software des Berkeley Vision and Learning Centers. Caffe bietet eine Art Bausatz für neuronale Netze. Man kann verschiedene Filter auswählen und sich so sein eigenes neuronales Netz erstellen. Caffe ist in der Lage Bilder sehr schnell auszuwerten, nach Herstellerangaben benötig es je Bild 4 ms in der Trainingsphase und 1 ms in der Testphase.