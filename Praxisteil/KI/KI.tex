\section{Künstliche Intelligenz und die Gesellschaft}
\authors{Florentine Klepel, Katharina Krause}

Computer begegnen uns täglich im Alltag. Die Rechenleistung wächst stark an und ermöglicht somit die Weiterentwicklung im Bereich der Künstlichen Intelligenz. Wird dadurch ermöglicht, dass Computer so intelligent wie Menschen werden? Wer übernimmt die Verantwortung, wenn es soweit kommt?\\
%

Zur Behandlung des Themas Künstliche Intelligenz geben wir zunächst einen Überblick über eine mögliche Definition und die verschiedenen Arten von KIs.
Anschließend werden wir Anwendungsbereiche darstellen und uns mit der Frage, wer die Verantwortung übernimmt, sowie unsere eigene Einstellung zur KI darstellen.
Künstliche Intelligenz (KI) bezeichnet den Versuch, spezifische, menschliche Verhaltensweisen, vor allem die Intelligenz, maschinell nachzuahmen. Im besten Fall den Menschen in seinen Fähigkeiten zu übertreffen. Hierbei verbindet die KI computerwissenschaftliche und kognitionswissenschaftliche Forschung.
Es geht nicht nur um rationale und rationelle Fertigkeiten des Menschen, sondern auch um die Nachahmung menschlichen Empfindens, Erkennens und Wahrnehmens.
Wenn man den Begriff KI näher versucht zu beschreiben, kann man zwischen zwei verschiedenen Arten unterscheiden: der starken und der schwachen KI. Diese Begriffe sind nicht klar definiert \cite{UniversitaetOldenburg}. \\
%

Grob gesehen bezeichnet „schwache KI“ Programme, welche für bestimmte Anwendungsdomänen entwickelt wurden. Ein Computer versucht, Intelligenz zu simulieren. Dies wird in unterschiedlichen Bereichen angewendet. Viele davon begegnen uns regelmäßig im Alltag. Zu den Teilgebieten gehören die Spracherkennung, die Verarbeitung natürlicher Sprachen, vor allem in Form vom Übersetzen, das Computersehen und die Robotik \cite{UniversitaetJena} \cite{Spektrum_1}.
Auf der anderen Seite steht die starke KI, welche den Computern intellektuelle Fähigkeiten, wie z.B. logisches Denken, Treffen von Entscheidungen bei Unsicherheit, Planen, Lernen und Kommunikation in natürlicher Sprache verleihen. Sie soll natürliche Kreativität und Emotionen nachbilden, gleichzeitig aber auch Bewusstsein bzw. Selbstbewusstsein entwickeln. In diesem Zusammenhang stellt sich die Frage, wann man von einer gelungenen starken KI sprechen kann. Die starke KI wird heutzutage durch den Turing-Test nachgewiesen. \\
%

Dieser Test stammt von Alan Turing, welcher durch seinen 1947 veröffentlichten Aufsatz „Intelligent Machinery“ die KI erstmals erwähnt hat. Bei diesem Test sind zwei Menschen und eine Maschine beteiligt. In einem Raum befindet sich eine Person, die nur über Tastatur und Bildschirm entweder mit einer Maschine oder einem Menschen kommuniziert, ohne zu wissen, wann sie mit wem redet. Das Ziel ist es dies durch geschicktes Fragen herauszufinden. Wenn die Wahrscheinlichkeit der korrekten Identifikation bei 50\% liegt, dann gilt die Versuchsperson als durch den Computer getäuscht und der Computer hat den Turing-Test bestanden.  \\
%

Es besteht eine große Diskussion darüber, inwiefern dieser Test wirklich die Intelligenz testet. Dafür spricht, dass in vielerlei Hinsicht eine menschliche Intelligenz getestet werden kann, da es in dem Test nicht nur um Wissen über die Welt, sondern vielmehr auch um soziale und emotionale Reaktionen bis hin zu religiösen Einstellungen geht. Ein weiterer positiver Punkt ist, dass dem Computer bei dem Test Flexibilität und Anpassungsfähigkeit abverlangt werden. Schließlich muss er sich mit dem plötzlichen Wechsel zwischen Themen im Gespräch, wie dem Einstreuen eines Witzes in einer Unterhaltung über das Wetter auseinandersetzen \cite{TheGuardian_1}. \\
%

Jedoch gab es seit Einführung des Tests an Schwierigkeiten. Es ist nämlich relativ einfach möglich Computerprogramme genauso zu programmieren, dass sie auf die meisten Fragen antworten können oder zumindest, wenn sie keine Antwort kennen,  dann mithilfe von Gegenfragen oder Ablenkungsmanövern zu reagieren \cite{NewAtlantis}.  Allerdings würden die meisten das noch nicht als intelligent bezeichnen.
Damit sind wir bei einem grundsätzlichen Problem der Definition des Begriffes der künstlichen Intelligenz. Was bedeutet eigentlich Intelligenz? 
Der Duden definiert Intelligenz als
1. die Fähigkeit abstrakt und vernünftig zu denken und daraus zweckvolles Handeln abzuleiten;
2. die Grundlage der Intellektualität. Dazu gehören wissenschaftliche, künstlerische, religiöse, literarische und journalistische Tätigkeiten. 
Wir halten diese Definition für sinnvoll, da genau diese Fähigkeiten es sind, welche den Mensch von Tieren unterscheiden.\\
%

Zu dem Thema der Verwirklichbarkeit meint Reinhard Furrer  \enquote{Solange wir nicht wissen, wie menschliche Intelligenz zustande kommt, können wir keine künstliche Intelligenz schaffen}. Damit behält er nur bedingt Recht. Der erste Punkt wurde bereits teilweise verwirklicht in Form von einem im Jahre 1997 entwickelten Supercomputer von IBM, der im Schachduell gegen den Weltmeister Garry Kasparov gewann. Schon damals war die Maschine dazu fähig, 200 Positionen pro Sekunde zu berechnen. Damit beweist er, dass es ihm möglich ist, zweckvoll zu handeln. Der Begriff der Intellektualität wurde damals noch nicht betrachtet. \\
%

Im Jahr 2014 schaffte es das erste Computer-Programm, präzise in ganzen Sätzen zu beschreiben, was auf Fotos zu sehen ist \cite{donahue2014}.  Der Nahrungsmittelkonzern Nestlé hat 1000 sprechende Roboter namens Pepper in seinen Kaffeeläden in Japan als Verkäufer eingesetzt.  Auch hier ist die erste Bedingung der Definition von Intelligenz erfüllt: sowohl das Google-Programm, als auch der Roboter benötigt die Fähigkeit, abstrakt zu denken und anschließend zweckvoll zu handeln, um die gestellte Aufgabe zu erfüllen \cite{Spiegel_2}. 
Doch gibt es bereits ein Programm, welches die vollständige Definition von Intelligenz erfüllt? Kann ein Computer intellektuell oder künstlerisch tätig sein \cite{BildDerWissenschaft_1}?\\
%

Ein weiteres Beispiel, auch von Google entwickelt, ist das Programm „Deep Dream“. Dieses Programm ist dazu fähig, bereits vorhandene Bilder zu verändern, sodass sie einen neuen Eindruck vermitteln bzw. bestimmte Stimmungen wiederzugeben.  An dieser Stelle liegt die Frage nahe, inwiefern eine solche Software selbst als Künstler oder Schöpfer angesehen werden kann und damit auch intellektuelle Eigenschaften besitzt. Am Anfang von kreativer Schöpfung liegen Ideen, die ein Computer in dieser Form nicht haben kann. Dem Computer werden Daten vorgegeben, die er so verarbeitet, dass etwas Neues daraus entstehen kann. Es ist einer Software nicht möglich, eine neue Kunstepoche einzuleiten, wie es beispielsweise die Impressionisten getan haben.
Durch diese Programme ist es möglich die Anzahl der Kunstwerke in kurzer Zeit stark zu vergrößern. Jedoch zeugt dies nicht zwingend von Kreativität des Computers. Deshalb ist es noch nicht möglich, das eigentliche Ziel der KI-Forschung, Vorgehensweisen der Informationsaufnahme und -verarbeitung zu entwerfen, die menschlichem Problemlösungsverhalten gleichen, zu verwirklichen. Obwohl die starke KI noch nicht wirklich realisiert wurde, hat sie bereits großen (teilweise guten, teilweise denkwürdigen) Einfluss auf unser Leben.\\
%

Beispielsweise ist es Computern bereits möglich, Straftaten auf Videoaufnahmen zu erkennen. Dieses Programm kann sich sehr gut an Situationen anpassen. Außerdem vermindert es die Zeit, welche ein Mensch zur Problemlösung benötigt. Durch die intelligenten Maschinen kommt es zu einer Zeitersparnis für den Menschen, da die Maschine selbstständig 24 Stunden die Aufsicht übernimmt. Dadurch wird die Leistungsfähigkeit stark erhöht. Trotz der Vorteile, die sich ergeben, sollte man sich darüber bewusst werden, dadurch dass der Mensch den Maschinen wichtige Aufgaben überlässt, einen Teil seine Verantwortung auf den Computer überträgt. Es kann immer sein, dass ein Computer eine Straftat „übersieht“, da er diese noch nicht erlernt hat \cite{UniversitaetHamburg_1}. Es kommt zu einer Art von Kontrollverlust, da der Mensch dem Computer vertraut und nur stichprobenartige Tests zur Analyse des Programmes und dessen Fehlern durchführt. Dieses Programm kann somit von Straftätern manipuliert werden. Der Mensch sollte deshalb bei der Benutzung solche Programme, darauf achten, die damit einhergehenden Folgen nicht zu unterschätzen. Ist es nicht deshalb besser Menschen teilweise diese Arbeit zu überlassen, auch unter Rücksichtnahme auf die vielen Arbeitsplätze, die durch die Programme der künstlichen Intelligenz wegfallen. \\
%

Noch fataler wären Fehlentscheidungen von den lebensrettenden Maschinen der Medizintechnik, welche auf künstlicher Intelligenz basieren. Spätestens hier wird deutlich, dass das Leben mancher Menschen an Maschinen hängt.  In Operationen werden immer häufiger Roboter wie der "da Vinci" des US-Herstellers Intuitive Surgical eingesetzt, die die Ärzte ersetzen. Jede falsche Bewegung kann hier den Tod bedeuten \cite{Spiegel_1}.
In Abhängigkeit mit dem Verstand, stellt sich die Frage, wer verantwortlich gemacht werden sollte, falls KIs Fehlentscheidungen treffen und somit Menschen gefährden oder sogar umbringen. Wer ist dann schuldig – der Programmierer, die Gesellschaft, der Mensch selbst? Auf all diese Fragen gibt es keine eindeutige Antwort. Allein der menschliche Verstand kann die Richtigkeit der erarbeiteten Ergebnisse durch künstliche Intelligenzen beurteilen und überprüfen und somit auch verantworten. \\
%

Zusammenfassend lässt sich sagen, dass der Computer bereits eine starke Entwicklung hinter sich hat. Wir waren beeindruckt von dem ersten Computer, der es schaffte, den Schachweltmeister zu besiegen und sind mitunter nicht minder amüsiert darüber, an was für profane Aufgaben der Computer scheitert. Momentan sind die Maschinen noch weit davon entfernt, einfache Probleme, wie die Mustererkennung lösen zu können.\\
%

Die Medien überspitzen häufig die angebliche Gefahr, welche von Maschinen ausgeht. Selbst wenn die Maschine schneller „denkt“ als wir: Was auch immer sie kann, einer von uns Menschen hat es ihr gegeben. Wir haben der Maschine die Mittel in die Hand gegeben, um zu lernen. Damit tragen wir als Gesellschaft, jeder Einzelne von uns, eine immense Verantwortung, mit der wir nicht spielen dürfen. Dies kann man mit der Erziehung von Kindern vergleichen. Diese können genau wie KIs in Zukunft Veränderungen hervorrufen. Für den Nachwuchs fühlt sich die Gesellschaft verantwortlich. Dieses Gefühl  sollten die Menschen auch gegenüber Computer entwickeln. \\
%

Jeder Schritt, den die Wissenschaft macht, hat Auswirkungen auf die Nachwelt. Die Frage, ob der Wissenschaft Grenzen gesetzt werden dürfen, wird viel diskutiert. Wenn sich jedoch jeder der Tragweite seines Handelns bewusst ist, dann kommt es nicht zu einem Kontrollverlust der Menschen gegenüber den Maschinen und die Frage nach einem Schuldigen, nach jemandem, dem man die Verantwortung im Falle einer Eskalation zuweisen wird, erübrigt sich. Unser Appell: Übernehmt Verantwortung!




