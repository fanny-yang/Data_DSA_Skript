\section{Künstliche Intelligenz}

Computer begegnen uns täglich im Alltag. Die Wissenschaft, die sich mit ihnen beschäftigt, wächst exponentiell. Können Computer so intelligent wie Menschen sein? Wer übernimmt die Verantwortung, wenn es soweit kommt?

Künstliche Intelligenz (KI) bezeichnet den Versuch, spezifische, menschliche Verhaltensweisen, vor allem die Intelligenz, maschinell nachzuahmen. Im besten Fall den Menschen in seinen Fähigkeiten zu übertreffen. Hierbei verbindet die KI computerwissenschaftliche und kognitionswissenschaftliche Forschung.
Es geht nicht nur um rationale und rationelle Fertigkeiten des Menschen, sondern auch um die Nachahmung menschlichen Empfindens, Erkennens und Wahrnehmens.
Wenn man den Begriff KI näher definiert, muss man zwischen zwei verschiedenen Arten unterscheiden: der starken und der schwachen KI. Die schwache KI wurde für bestimmte Anwendungsdomänen entwickelt. Ein Computer probiert, Intelligenz zu simulieren. Im Gegensatz dazu verleiht die starke KI Computern intellektuelle Fähigkeiten, wie z.B. logisches Denken, Treffen von Entscheidungen bei Unsicherheit, Planen, Lernen, Kommunikation in natürlicher Sprache. Für ein als gelungen angesehenes Ergebnis soll eine starke KI all diese Fertigkeiten miteinander verbinden. Er soll natürliche Kreativität und Emotionen nachbilden, gleichzeitig Bewusstsein bzw. Selbstbewusstsein entwickeln.
Die KI wird in unterschiedlichsten Bereichen angewendet. Viele davon begegnen uns regelmäßig im Alltag. Grundsätzlich lassen sich die Anwendungsbereiche in drei Kerndisziplinen aufteilen. Zuerst zu nennen sind Expertensysteme, die für Maschinen in Medizin, Forschung, Militär, Ausbildung und Ingenieurwissenschaften angewandt werden. Außerdem gibt es die Verarbeitung natürlicher Sprachen, welche sich mit dem Übersetzen, Erkennen und Wiedergeben auseinandersetzt. Die 3. große Kerndisziplin ist die Robotik, die den Menschen z.B. in der Industrie teilweise ersetzt.
Die Geburtsstunde der KI liegt in dem 1947 von Alan Turing veröffentlichten Aufsatz „Intelligent Machinery“. Nach ihm ist ein Test zum Nachweis starker künstlicher Intelligenz benannt, dem Turing Test. Bei diesem Test sind zwei Menschen und eine Maschine beteiligt. In einem Raum befindet sich eine Person, die nur über Tastatur und Bildschirm entweder mit einer Maschine oder einem Menschen kommuniziert, ohne zu wissen, wann sie mit wem redet. Das Ziel ist es dies durch geschicktes Fragen herauszufinden. Wenn die Wahrscheinlichkeit der korrekten Identifikation bei 50\% liegt, dann gilt die Versuchsperson als durch den Computer getäuscht und der Computer hat den Turing-Test bestanden.  Es besteht eine große Diskussion darüber, inwiefern dieser Test sinnvoll ist. Dafür spricht, dass in vielerlei Hinsicht eine menschliche Intelligenz getestet werden kann, da eine Vielzahl von Wissensbereichen einbezogen werden. Es geht dabei nicht nur um Wissen über die Welt, sondern vielmehr auch um soziale und emotionale Reaktionen bis hin zu religiösen Einstellungen. Ein weiterer positiver Punkt ist, dass dem Computer bei dem Test Flexibilität und Anpassungsfähigkeit abverlangt werden, schließlich muss er mit dem plötzlichen Wechsel zwischen Situationen z.B. dem Wechsel der Unterhaltung über das Wetter zu Witzen auseinandersetzen. Jedoch gab es von Beginn des Tests an Schwierigkeiten. Es ist möglich Computerprogramme genauso zu programmieren, dass sie auf die meisten Fragen antworten können. Wenn sie keine Antwort kennen, dann antworten sie mit Gegenfragen oder wenden Ablenkungsmanöver an. Damit sind wir bei einem grundsätzlichen Problem der Definition des Begriffes der künstlichen Intelligenz. Was bedeutet eigentlich Intelligenz? Wie kann man sie nachweisen? 
Der Duden definiert Intelligenz als
\begin{enumerate} \item Fähigkeit abstrakt und vernünftig zu denken und daraus zweckvolles Handeln abzuleiten;
\item Grundlage der Intellektualität. Dazu gehören wissenschaftliche, künstlerische, religiöse, literarische und journalistische Tätigkeiten. 
\end{enumerate}
Dass der erste Punkt bereits verwirklichbar ist, zeigte schon im Jahre 1997 der Supercomputer von IBM, der im Schachduell gegen den Weltmeister Garry Kasparov gewann. Schon damals war die Maschine dazu fähig 200 Positionen pro Sekunde zu berechnen. Der Begriff der Intellektualität wurde damals noch nicht betrachtet. 2014 ist die Wissenschaft weiter fortgeschritten. Das Google-Programm GoogleNet beschreibt präzise in ganzen Sätzen, was auf Fotos zu sehen ist.  Der Nahrungsmittelkonzern Nestlé hat 1000 sprechende Roboter namens Pepper in seinen Kaffeeläden in Japan als Verkäufer eingesetzt.  Auch hier ist die erste Bedingung der Definition von Intelligenz erfüllt: sowohl das Google-Programm, als auch der Roboter benötigt die Fähigkeit abstrakt zu denken und anschließend zweckvoll zu handeln, um die gestellte Aufgabe zu erfüllen. Doch erfüllt solch ein Google-Programm auch die 2. Eigenschaft? Kann man es als intellektuell bezeichnen, weil es Kunst erkennt, interpretiert und literarisch auswertet? Ein weiteres Beispiel, auch von Google entwickelt, ist das Programm \glqq Deep Dream\grqq. Dieses Programm ist dazu fähig, bereits vorhandene Bilder zu verändern, sodass sie einen neuen Eindruck vermitteln bzw. bestimmte Stimmungen wiederzugeben.  An dieser Stelle liegt die Frage nahe, inwiefern der Computer selbst als Künstler oder Schöpfer angesehen werden kann und damit auch intellektuelle Eigenschaften besitzt. Am Anfang von kreativer Schöpfung liegen Ideen, die ein Computer in dieser Form nicht haben kann. Dem Computer werden Daten vorgegeben, die er so verarbeitet, dass etwas Neues daraus entstehen kann. Jedoch ist es durch die Computer möglich, die Anzahl der Kunstwerke in kurzer Zeit stark zu vergrößern. Damit ist das eigentliche Ziel der KI-Forschung Vorgehensweisen der Informationsaufnahme und Informationsverarbeitung zu entwerfen, die menschlichem Problemlösungsverhalten näher kommen, und daraus Methoden zur qualitativen Verbesserung und Anreicherung herkömmlicher Systeme der Informatik abzuleiten, noch nicht erfüllt. Zu den bisher erreichten Teilzielen gehören vor allem die Entwicklung von Programmsystemen auf höherem Abstraktionsniveau und damit die Erarbeitung neuer Software-Techniken sowie verbesserter Formen der Mensch-Maschine-Kommunikation. Beispielsweise ist es Computern bereits möglich Straftaten auf Videoaufnahmen zu erkennen. Es ist diesem Programm möglich sich sehr gut an Situationen anzupassen. Außerdem ermöglicht es eine schnellere Problemlösung als es dem Menschen möglich ist. Durch die intelligenten Maschinen kommt es zu einer Zeitersparnis für den Menschen, da die Maschine selbstständig 24 Stunden die Aufsicht übernimmt. Dadurch wird die Leistungsfähigkeit stark erhöht, um nur einige Vorteile solcher Programme, welche durch künstliche Intelligenz funktionieren, zu nennen. Gleichzeitig gilt es aber nicht zu unterschätzen, dass der Mensch dadurch seine Verantwortung auf den Computer überträgt. Es kann immer sein, dass ein Computer eine Straftat \glqq übersieht \grqq, da er diese noch nicht erlernt hat. Der Mensch wird dem Computer vertrauen und nur stichprobenartige Tests zur Analyse des Programmes und dessen Fehler durchführen. Dieses Programm kann somit von Straftätern manipuliert werden. Ist es nicht besser deshalb Menschen diese Arbeit zu überlassen, auch unter Rücksichtnahme auf die vielen Arbeitsplätze, die durch die Programme der künstlichen Intelligenz immer weiter ersetzt werden. Am schwersten überwiegt jedoch das Argument des Kontrollverlustes. Die Menschen sind überfordert mit der sie überflutenden Datenmenge. Der Mensch muss der Chef der Maschinen bleiben, alles andere führt zu psychischen Krankheiten. Viele von uns kennen die Unsicherheit in Anbetracht all der Informationen, die über soziale Netzwerke ohne unser Wissen verbreitet werden. Können wir durch die Möglichkeit des Internets noch wissen, wer was über uns weiß? Die ständige Erreichbarkeit und die unendlichen Optionen überfordern die menschliche Psyche. Hinzu kommt die wachsende Abhängigkeit von den Geräten, die im täglichen Alltag nicht mehr wegzudenken sind. Ein Tag ohne Handy, Computer und Fernsehen – ist das noch möglich? Noch abhängiger sind wir von den lebensrettenden Maschinen der Medizintechnik, welche auf künstlicher Intelligenz basieren. Spätestens hier wird deutlich, dass das Leben mancher Menschen an Maschinen hängt. Dieses Beispiel zeigt, dass KIs eine große Bereicherung für das menschliche Leben sein können. So können sie genauer und objektiver Situationen analysieren. Auf der anderen Seite muss gesagt werden, dass dazu aber vorher das Einspeisen von Daten nötig ist. Der Computer führt lediglich einen Abgleich der momentanen Situation mit den vorgegebenen Daten ab und entscheidet sich für die größte Übereinstimmung. Dies ist keine Art von eigentlichem Denken. Weiterhin stellt es eine schwierige Aufgabe dar, dem Computer emotionale Intelligenz nachzuweisen. Da dem Menschen unklar ist, wie genau unsere Gefühle funktionieren, ist es fraglich ob diese auf ein Programm übertragen werden können. Wenn man jetzt davon ausgeht, dass es trotz dessen möglich wäre, würde dies zum effizienteren Handeln führen. Doch schlussendlich stellt sich die Frage, wer verantwortlich gemacht werden sollte, sollten KIs Fehlentscheidungen treffen und somit Menschen gefährden oder sogar umbringen? Wer ist dann schuldig – der Programmierer, die Gesellschaft, der Mensch selbst? Auf all diese Fragen gibt es keine eindeutige Antwort. Allein der menschliche Verstand kann die Richtigkeit der erarbeiteten Ergebnisse durch künstliche Intelligenzen beurteilen und überprüfen und somit auch verantworten.\\
Zusammenfassend lässt sich sagen, dass der Computer bereits eine starke Entwicklung hinter sich hat. 
Wir waren beeindruckt von dem Computer, der es schaffte, den Schachweltmeister zu besiegen und sind mitunter nicht minder amüsiert darüber, an was für profanen Aufgaben der Computer scheitert.  Denn nur wir können beurteilen, was richtig ist und was falsch. Der Maschine mag es (noch?) an Bewusstsein mangeln. Uns Menschen allerdings mangelt es an der Erkenntnis, wie überlegen wir ihr immer noch sind. Selbst wenn die Maschine schneller \glqq denkt\grqq  als wir: Was auch immer sie kann, einer von uns hat es ihr gegeben. Wir haben der Maschine die Mittel in die Hand gegeben, damit sie lernen kann. Damit tragen wir als Gesellschaft, jeder Einzelne von uns, eine immense Verantwortung, mit der wir nicht spielen dürfen. Jeder Schritt, den die Wissenschaft macht, hat Auswirkungen auf die Nachwelt. Die Frage, ob der Wissenschaft Grenzen gesetzt werden dürfen, wird viel diskutiert. Wenn sich jedoch jeder der Tragweite seines Handelns bewusst ist, dann kommt es nicht zu einem Kontrollverlust der Menschen gegenüber den Maschinen und die Frage nach einem Schuldigen, nach jemandem, dem man die Verantwortung im Falle einer Eskalation zuweisen wird, erübrigt sich. Unser Appell: Übernehmt Verantwortung!
