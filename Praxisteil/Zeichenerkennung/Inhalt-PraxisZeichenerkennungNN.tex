\section{Zeichenerkennung mit neuronalen Netzen}
\authors{Luca Bohn, Lukas Lück, Lukas Kamm}
Nun betrachten wir neuronale Netze im Zusammenhang mit Schrifterkennung. Convolutional Neural Networks eignen sich besonders für die Verarbeitung von Bildern. Eine \emph{Convolution} ist im konkreten Anwendungsfall mit Bildern die Zusammenfassung der Pixel in einem kleinen Bereich, die in ein neues Bild gespeichert werden. Dadurch können beispielsweise kleine Rotationen des Objekts auf einem Foto, das erkannt werden soll, kompensiert werden. Außerdem lassen sich im CNN \emph{Feature Maps} vielseitig anwenden (näheres in Abschnitt \ref{ml:cnn}).
Ein funktionsfähiges neuronales Netz zu implementieren, bedeutet einen großen Zeitaufwand und viel Feinoptimierung. Da uns diese Zeit nicht zur Verfügung steht und wir ein größeres Interesse an der Anwendung haben, nutzen wir Caffe. Bei Caffe handelt es sich um eine Software des Berkeley Vision and Learning Centers. Caffe bietet eine Art Bausatz für neuronale Netze. Man kann verschiedene Filter auswählen und sich so sein eigenes neuronales Netz erstellen. Caffe ist in der Lage, Bilder sehr schnell auszuwerten, nach Herstellerangaben benötigt es je Bild 4 ms in der Trainingsphase und 1 ms in der Testphase.
Als Datensatz für die Zeichen nutzen wir MNIST. Dies ist ein freier Satz aus 28 $\times$ 28 px großen Bildern mit den arabischen Ziffern in Graustufe. Der Testsatz enthält 60.000, der Trainingssatz 10.000 Grafiken. Diese vorgefertigte Zeichendatenbank eignet sich vor allem für Testumgebungen im Bereich des maschinellen Lernens und das Testen von Algorithmen neuronaler Netze. Aufgrund dessen ist sie optimal für beide Schrifterkennungs-Teams geeignet.
\begin{dsafigure}
\begin{center}
	\includegraphics[width=0.40\textwidth]{\media NNFilterMNIST.png}
	\caption{Filter (sog. Feature Maps), spezialisiert für MNIST}
	\label{NNFilterMNIST}
\end{center}
\end{dsafigure}

\subsection{Training}
Voraussetzung für das maschinelle Lernen mit neuronalen Netzen ist das Anlernen mit Hilfe von Trainingsdaten. Hierzu trainieren wir Caffe mit allen 60.000 Grafiken aus dem MNIST-Datensatz.

\subsection{Testen}
Unser Ziel ist es, aus einem Bild mit Handschriftcharakter arabische Ziffern richtig zu erkennen. Hierzu erzeugen wir ein Bild mit $10 \times 10$ zufällig gewählten Ziffern, die jeweils um wenige Pixel von ihrer Zentrumsposition abweichen, um den Handschriftcharakter zu simulieren (vgl. Ausschnitt in Abb. \ref{Zahlen_input}). Anschließend iterieren wir pixelweise, ebenso wie die SVM-Projektgruppe, über das Bild und extrahieren wieder 28 $\times$ 28 px große Grafiken und übermitteln sie an das neuronale Netz.