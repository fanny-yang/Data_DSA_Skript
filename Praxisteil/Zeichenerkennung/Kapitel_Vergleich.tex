\section{Vergleich SVM - Neuronales Netzwerk}

\subsection{Trainingsdauer}

Allgemein sind SVMs schneller zu trainieren als Neuronale Netzwerke, da SVMs nur eine Schicht haben wärend Neuronale Netze aus mehreren Schichten bestehen, in denen die Informationen verarbeitet werden, und deshalb natürlich länger zum berechnen brauchen. In unserem Projekt waren allerdings anfangs die Neuronalen Netze wesentlich schneller, was aber an der schlechteren Implementierung der SVM lag und sie dadurch nicht optimal genutzt wurde. Auch mussten wir bei der SVM zuerst die Features sämtlicher Bilder berechnen bevor wir sie trainieren konnte. Dies war bei dem Neuronalen Netz nicht notwendig, weshalb es insgesamt wesentlich schneller war, denn dieser Vorgang dauert am längsten.

\subsection{Genauigkeit}