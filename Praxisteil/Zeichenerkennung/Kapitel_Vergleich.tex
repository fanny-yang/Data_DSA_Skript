\section{Vergleich SVM und neuronales Netzwerk}

\subsection{Trainingsdauer}

Allgemein sind SVMs schneller zu trainieren als neuronale Netze, da SVMs nur eine Schicht haben während neuronale Netze aus mehreren Schichten bestehen, in denen die Informationen verarbeitet werden, und deshalb natürlich länger zum Berechnen brauchen. In unserem Projekt waren allerdings anfangs die neuronalen Netze wesentlich schneller, was aber an der schlechteren Implementierung der SVM lag und sie dadurch nicht optimal genutzt wurde. Auch mussten wir bei der SVM zuerst die Features sämtlicher Bilder berechnen, bevor wir sie trainieren konnten. Dies war bei dem neuronalen Netz nicht notwendig, weshalb es insgesamt wesentlich schneller war, denn dieser Vorgang dauert am längsten.

\subsection{Genauigkeit}
Im direkten Vergleich der SVM und des neuronalen Netzes lässt sich feststellen, dass die SVM eine Trefferquote von 98~\% aufweist. \\
Leider hatten wir nicht genügend Zeit, einen Testalgorithmus basierend auf einem Convolutional Neural Network zu implementieren und somit direkte Vergleiche hinsichtlich der Genauigkeit beider Algorithmen durchzuführen.